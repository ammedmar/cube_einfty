
\section{Proof of Theorem~\ref{t:comparison}} \label{s:proof2}

Given a cubical set $X$ and a simplicial set $Y$, we need to show the commutativity of the diagram
\begin{equation*}
\begin{tikzcd}
U(\MS)(r) \otimes \cchains(X) \arrow[r] \arrow[d, "\id\; \otimes\, S"'] &
\cchains(X)^{\otimes r} \arrow[d, "S^{\otimes r}"] \\
U(\MS)(r) \otimes \cchains(Y) \arrow[r] &
\cchains(Y)^{\otimes r}.
\end{tikzcd}
\end{equation*}
It suffices to consider elements represented by a surjection-like graph $\Gamma$ since $U(\MS)$ is generated by them.
By naturality, it suffices to consider $X = \cube^n$, $Y = \triangle^n$ and show that
\begin{equation} \label{e:cartan-serre U(MS)-coalgebra map}
S^{\otimes r} \circ \Gamma\big( [0,1]^{\otimes n} \big) = \Gamma \circ S \big( [0,1]^{\otimes n} \big).
\end{equation}

Let us start by verifying that the Cartan-Serre chain map is a morphism of coalgebras.
\begin{lemma}
	If $\Gamma =$ \counit \ or \coproduct \ then \eqref{e:cartan-serre U(MS)-coalgebra map} holds.
\end{lemma}

\begin{proof}
	TBW
\end{proof}

\begin{definition}
	A basis element $x_1 \otimes \cdots \otimes x_n \in \cchains(\cube^n)$ is said to be  \textit{vertex-ordered} if $x_i = [0]$ and $x_j = [1]$ imply $i < j$.
\end{definition}

\begin{lemma}
	Let $\Gamma$ be the surjection-like graph representing the $(r-1)\th$ iterated Serre diagonal. Explicitly,
	\begin{equation*}
	\begin{tikzpicture}[scale=.35]
	\node at (9.5, 1){$\Gamma \, = $};
	\draw (11,.5)--(12,1.5)--(12,2.5);
	\draw (13,.5)--(12,1.5);
	\node at (11,0){$\scriptstyle 1$};
	\node at (12,0){$\scriptstyle \dots$};
	\node at (13,0){$\scriptstyle r$};
	\end{tikzpicture}
	\end{equation*}
	and let $x_1 \otimes \cdots \otimes x_n \in \cchains(\cube^n)$ be a basis element.
	If
	\begin{equation*}	
	\Gamma \big( x_1 \otimes \cdots \otimes x_n \big) =
	\sum \pm \left( x_1^{(1)} \otimes \cdots \otimes x_n^{(1)} \right)
	\otimes \cdots \otimes
	\left( x_1^{(r)} \otimes \cdots \otimes x_n^{(r)} \right)
	\end{equation*}
	with each $x_i^{(k)}$ a basis element of $\cchains(\cube^1)$, then
	\begin{enumerate}
		\item For every $i \in \{1, \dots, n\}$ the element $x_i^{(1)} \otimes \cdots \otimes x_i^{(r)}$ is vertex-ordered.
	
	 	\item For every $k \in \{1, \dots, r\}$ if $x_1 \otimes \cdots \otimes x_n$ is vertex-ordered then $x_1^{(k)} \otimes \cdots \otimes x_n^{(k)}$ is vertex-ordered.
	\end{enumerate}
\end{lemma}

\begin{proof}
	and the claim follows from definition inspecting
	\begin{equation*}
	\Delta([0]) = [0] \otimes [0], \quad \Delta([1]) = [1] \otimes [1], \quad \Delta([0, 1]) = [0] \otimes [0, 1] + [0, 1] \otimes [1].
	\end{equation*}
\end{proof}

\begin{lemma}
	product of vertex-ordered is vertex ordered
\end{lemma}

\begin{lemma}
	commutativity for product of vertex ordered elements.
\end{lemma}