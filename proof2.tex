
\section{Proof of Theorem~\ref{t:comparison}} \label{s:proof2}

Given a cubical set $X$ and a simplicial set $Y$, we need to show the commutativity of the diagram
\begin{equation*}
\begin{tikzcd}
U(\MS)(r) \otimes \cchains(X) \arrow[r] \arrow[d, "\id\; \otimes\, S"'] &
\cchains(X)^{\otimes r} \arrow[d, "S^{\otimes r}"] \\
U(\MS)(r) \otimes \cchains(Y) \arrow[r] &
\cchains(Y)^{\otimes r}.
\end{tikzcd}
\end{equation*}
It suffices to consider elements represented by a surjection-like graph $\Gamma$ since $U(\MS)$ is generated by them.
By naturality, it suffices to consider $X = \cube^n$, $Y = \triangle^n$ and show that
\begin{equation} \label{e:cartan-serre U(MS)-coalgebra map}
S^{\otimes r} \circ \Gamma\big( [0,1]^{\otimes n} \big) = \Gamma \circ S \big( [0,1]^{\otimes n} \big).
\end{equation}

We use triples $(F_0, F_{01}, F_1)$ of subsets of $\{0, \dots, n\}$ to represent faces $F$ of $\gcube^n$ with $F_\epsilon = \{i \mid \forall x \in F, \, x_i = \epsilon\}$ for $\epsilon \in \{0,1\}$ and $F_{01} = \{0, \dots, n\} \setminus F_0 \cup F_1$.

\anibal{improve this below}

Given a subset $U = \{u_1 < \cdots < u_q\}$ of $\{0, \dots, n\}$ we write $d_U$ for $d_{u_1} \! \cdots \, d_{u_q}$.
In this notation, any face of $\gsimplex^n$ can be written as $d_U [0, \dots, n]$ for some $U$.

With this notation we have the following description of the Cartan-Serre chain map.

\begin{lemma}
	On basis elements $S \colon \cchains(\cube^n) \to \chains(\triangle^n)$ is given by
	\begin{equation*}
	(F_0, F_{0,1}, F_1) \mapsto
	\begin{cases}
	d_U [0, \dots, n] & \text{ if } F_{01} \cap \{i \mid i > \min(F_0)\} = \emptyset, \\
	0 & \text{ otherwise},
	\end{cases}
	\end{equation*}
	where $U = \{i-1 \mid i \in F_1 \text{ or } i > \min(F_0)\}$ with the convention $\min(\emptyset) = +\infty$.
\end{lemma}

\begin{proof}
	We assume $n > 0$ since otherwise there is nothing to prove.
	Consider a face $(F_0, F_{01}, F_1)$ and let $M \subseteq \{0, \dots, n\}$ be empty if $F_0$ is empty or be characterized by $i > \min (F_0)$ otherwise.
	Notice in \eqref{e:cartan-serre CW map} that if $x_i = 0$ then $y_j = 0$ for every $j > i$.
	Therefore, the image of $(F_0, F_{01}, F_1)$ in $\simplex^n$ is a face of $d_U[0, \dots, n]$ where $U = \{i-1 \mid i \in M\}$ or, more explicitly, $[0, \dots, n]$ if $F_0$ is empty and $[0, \dots, \min(F_0)]$ otherwise.
	In particular, $S(F_0, F_{01}, F_1)$ is non-zero only if $M \cap F_{01} = \emptyset$, and we can assume without loss of generality that $F_0 = \emptyset$ or $F_0 = \{n\}$.
	
	Notice from \eqref{e:cartan-serre CW map} that $x_i = 1$ if and only if $y_{i-1} = 0$, so $S(\emptyset, F_{01}, F_1) = d_{U} [0, \dots, n]$ where $U = \{i-1 \mid i \in F_1\}$.
\end{proof}

Let us start by verifying that the Cartan-Serre chain map is a morphism of counital coalgebras.
\begin{lemma}
	If $\Gamma =$ \counit \ or \coproduct \ then \eqref{e:cartan-serre U(MS)-coalgebra map} holds.
\end{lemma}

\begin{proof}
	TBW
\end{proof}

\begin{definition}
	A basis element $x_1 \otimes \cdots \otimes x_n \in \cchains(\cube^n)$ is said to be  \textit{vertex-ordered} if $x_i = [0]$ and $x_j = [1]$ imply $i < j$.
\end{definition}

We will consider the basis of $\cchains(\cube^n)$ as a poset defined as follows:
For $n = 1$ we set $[0] < [0,1] < [1]$, and for $n > 1$ we have $(x_1 \otimes \cdots \otimes x_n) \leq (y_1 \otimes \cdots \otimes y_n)$ if $x_i \leq y_i$ for each $i \in \{1, \dots, n\}$. 

\begin{lemma}
	Let
	\begin{equation*}
	\begin{tikzpicture}[scale=.35]
	\node at (9.5, 1){$\Gamma \, = $};
	\draw (11,.5)--(12,1.5)--(12,2.5);
	\draw (13,.5)--(12,1.5);
	\node at (11,0){$\scriptstyle 1$};
	\node at (12,0){$\scriptstyle \dots$};
	\node at (13,0){$\scriptstyle r$};
	\end{tikzpicture}
	\end{equation*}
	be the surjection-like graph representing the $(r-1)\th$ iterated Serre diagonal.
	If
	\begin{equation*}	
	\Gamma\big([0,1]^{\otimes n}\big) =
	\sum \pm \ x^{(1)} \otimes \cdots \otimes x^{(r)}
	\end{equation*}
	with each $x^{(k)} \in \cchains(\cube^n)$ a basis element, then each $x^{(k)}$ is vertex-ordered and $x^{(1)} \leq \cdots \leq x^{(r)}$.
\end{lemma}

\begin{proof}
	The claim follows from definition after inspecting
	\begin{equation*}
	\Delta([0]) = [0] \otimes [0], \quad \Delta([1]) = [1] \otimes [1], \quad \Delta([0, 1]) = [0] \otimes [0, 1] + [0, 1] \otimes [1].
	\end{equation*}
\end{proof}

\begin{lemma}
	Let $x,y,z \in \cchains(\cube^n)$ be basis elements. If $x \leq y \leq z$ and $z$ is vertex-ordered, then $(x \ast y) \leq z$.
\end{lemma}

\begin{lemma}
	commutativity for product of vertex ordered elements.
\end{lemma}










%\begin{lemma}
%	Let
%	\begin{equation*}
%	\begin{tikzpicture}[scale=.35]
%	\node at (9.5, 1){$\Gamma \, = $};
%	\draw (11,.5)--(12,1.5)--(12,2.5);
%	\draw (13,.5)--(12,1.5);
%	\node at (11,0){$\scriptstyle 1$};
%	\node at (12,0){$\scriptstyle \dots$};
%	\node at (13,0){$\scriptstyle r$};
%	\end{tikzpicture}
%	\end{equation*}
%	be the surjection-like graph representing the $(r-1)\th$ iterated Serre diagonal, and $x_1 \otimes \cdots \otimes x_n \in \cchains(\cube^n)$ a basis element.
%	If
%	\begin{equation*}	
%	\Gamma \big( x_1 \otimes \cdots \otimes x_n \big) =
%	\sum \pm \left( x_1^{(1)} \otimes \cdots \otimes x_n^{(1)} \right)
%	\otimes \cdots \otimes
%	\left( x_1^{(r)} \otimes \cdots \otimes x_n^{(r)} \right)
%	\end{equation*}
%	with each $x_i^{(k)}$ a basis element of $\cchains(\cube^1)$, then
%	\begin{enumerate}
%		\item For every $i \in \{1, \dots, n\}$ the element $x_i^{(1)} \otimes \cdots \otimes x_i^{(r)}$ is vertex-ordered.
%	
%	 	\item For every $k \in \{1, \dots, r\}$ if $x_1 \otimes \cdots \otimes x_n$ is vertex-ordered then $x_1^{(k)} \otimes \cdots \otimes x_n^{(k)}$ is vertex-ordered.
%	\end{enumerate}
%\end{lemma}