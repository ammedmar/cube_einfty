\documentclass[12pt]{article}

% Cahiers wants Times New Roman:

\usepackage{times}

% Cahiers wants this page layout:

\usepackage[a4paper,text={128mm,185mm}, centering]{geometry}

\usepackage{changepage}
% Slight modification of section titles (optional):

\usepackage{titlesec}

\titleformat{\section}[hang]%
{\bfseries\large}{\thesection.}{1ex}{}%

\titleformat{\subsection}[hang]%
{\bfseries}{\thesubsection}{1ex}{}%

\usepackage{amsmath, amsthm}
\usepackage{fancyhdr}
\pagestyle{fancy}


%lefthead and rightheads: authors and short title
\lhead{\sc\bfseries R.K. $\&$ A.M-M.}
\rhead{\sc\bfseries An $E_\infty$-structure on cubical cochains}

%title
\title{\vskip 5pt  \bf  \textsc{A combinatorial $E_\infty$-algebra structure on cubical cochains and the Cartan--Serre map}}
% Nota Bene : please, write your title in capital letters, as in the example here.
%authors
\author{
	\itshape\bfseries {Ralph~M.~Kaufmann} \and
	\itshape\bfseries {Anibal~M.~Medina-Mardones}
}

\usepackage{amssymb}
\usepackage{tikz-cd}
\usepackage{csquotes}
\usepackage{bm}
\usepackage{mathbbol}
\DeclareSymbolFontAlphabet{\amsmathbb}{AMSb}

% bibliography
\usepackage[backend=biber,
style=alphabetic,
backref=true,
url=false,
doi=false,
eprint=false]{biblatex}

\renewbibmacro{in:}{}

\addbibresource{aux/bibliography.bib}
\addbibresource{aux/usualpapers.bib}
\addbibresource{aux/mypapers.bib}

\DeclareFieldFormat{title}{\myhref{\mkbibemph{#1}}}
\DeclareFieldFormat
[article,inbook,incollection,inproceedings,patent,thesis,unpublished]
{title}{\myhref{\mkbibquote{#1\isdot}}}

\newcommand{\doiorurl}{%
	\iffieldundef{url}
	{\iffieldundef{eprint}
		{}
		{http://arxiv.org/abs/\strfield{eprint}}}
	{\strfield{url}}%
}

\newcommand{\myhref}[1]{%
	\ifboolexpr{%
		test {\ifhyperref}
		and
		not test {\iftoggle{bbx:eprint}}
		and
		not test {\iftoggle{bbx:url}}
	}
	{\href{\doiorurl}{#1}}
	{#1}%
}

% references
\usepackage[bookmarks=true,
linktocpage=true,
bookmarksnumbered=true,
breaklinks=true,
pdfstartview=FitH,
hyperfigures=false,
plainpages=false,
naturalnames=true,
colorlinks=true,
pagebackref=false,
pdfpagelabels]{hyperref}

\hypersetup{
	colorlinks,
	citecolor=blue,
	filecolor=blue,
	linkcolor=blue,
	urlcolor=blue
}

\usepackage[capitalize, noabbrev]{cleveref}
\let\subsectionSymbol\S
\crefname{subsection}{\subsectionSymbol\!}{subsections}

% layout
\addtolength{\textwidth}{0in}
\addtolength{\textheight}{0in}
\calclayout

% update to MSC2020
\makeatletter
\@namedef{subjclassname@2020}{%
	\textup{2020} Mathematics Subject Classification}
\makeatother

% table of contents
\setcounter{tocdepth}{1}
% environments
\newtheorem{theorem}{Theorem}
\newtheorem{proposition}[theorem]{Proposition}
\newtheorem{lemma}[theorem]{Lemma}
\newtheorem{corollary}[theorem]{Corollary}

\theoremstyle{definition}
\newtheorem{definition}[theorem]{Definition}
\newtheorem*{definition*}{Definition}
\newtheorem{remark}[theorem]{Remark}
\newtheorem*{remark*}{Remark}
\newtheorem{example}[theorem]{Example}
\newtheorem*{example*}{Example}
\newtheorem{convention}[theorem]{Convention}
\newtheorem*{convention*}{Convention}
\newtheorem{notation}[theorem]{Notation}
\newtheorem*{notation*}{Notation}
\newtheorem{question}[theorem]{Question}
\newtheorem*{question*}{Question}

% hyphenation
\hyphenation{co-chain}
\hyphenation{co-chains}
\hyphenation{co-al-ge-bra}
\hyphenation{co-al-ge-bras}
\hyphenation{co-bound-ary}
\hyphenation{co-bound-aries}

% basics
\DeclareMathOperator{\face}{d}
\DeclareMathOperator{\dege}{s}
\DeclareMathOperator{\bd}{\partial}
\DeclareMathOperator{\sign}{sign}
\newcommand{\ot}{\otimes}
\DeclareMathOperator{\EZ}{EZ}
\DeclareMathOperator{\AW}{AW}

% sets and spaces
\newcommand{\N}{\mathbb{N}}
\newcommand{\Z}{\mathbb{Z}}
\newcommand{\Q}{\mathbb{Q}}
\newcommand{\R}{\mathbb{R}}
\renewcommand{\k}{\Bbbk}
\newcommand{\sym}{\mathbb{S}}
\newcommand{\cyc}{\mathbb{C}}
\newcommand{\Ftwo}{{\mathbb{F}_2}}
\newcommand{\Fp}{{\mathbb{F}_p}}
\newcommand{\Cp}{{\cyc_p}}
\newcommand{\gsimplex}{\mathbb{\Delta}}
\newcommand{\gcube}{\mathbb{I}}

% categories
\newcommand{\Cat}{\mathsf{Cat}}
\newcommand{\Fun}{\mathsf{Fun}}
\newcommand{\Set}{\mathsf{Set}}
\newcommand{\Top}{\mathsf{Top}}
\newcommand{\CW}{\mathsf{CW}}
\newcommand{\Ch}{\mathsf{Ch}}
\newcommand{\simplex}{\triangle}
\newcommand{\sSet}{\mathsf{sSet}}
\newcommand{\cube}{\square}
\newcommand{\cSet}{\mathsf{cSet}}
\newcommand{\Alg}{\mathsf{Alg}}
\newcommand{\coAlg}{\mathsf{coAlg}}
\newcommand{\biAlg}{\mathsf{biAlg}}
\newcommand{\sGrp}{\mathsf{sGrp}}
\newcommand{\Mon}{\mathsf{Mon}}
\newcommand{\symMod}{\mathsf{Mod}_{\sym}}
\newcommand{\symBimod}{\mathsf{biMod}_{\sym}}
\newcommand{\operads}{\mathsf{Oper}}
\newcommand{\props}{\mathsf{Prop}}

% operators
\DeclareMathOperator{\free}{F}
\DeclareMathOperator{\forget}{U}
\DeclareMathOperator{\yoneda}{\mathcal{Y}}
\DeclareMathOperator{\Sing}{Sing}
\newcommand{\loops}{\Omega}
\DeclareMathOperator{\cobar}{\mathbf{\Omega}}
\DeclareMathOperator{\proj}{\pi}
\DeclareMathOperator{\incl}{\iota}

% chains
\DeclareMathOperator{\chains}{N}
\DeclareMathOperator{\cochains}{N^{\vee}}
\DeclareMathOperator{\gchains}{C}

% pair delimiters (mathtools)
\DeclarePairedDelimiter\bars{\lvert}{\rvert}
\DeclarePairedDelimiter\norm{\lVert}{\rVert}
\DeclarePairedDelimiter\angles{\langle}{\rangle}
\DeclarePairedDelimiter\set{\{}{\}}

% other
\newcommand{\id}{\mathsf{id}}
\renewcommand{\th}{\mathrm{th}}
\newcommand{\op}{\mathrm{op}}
\DeclareMathOperator*{\colim}{colim}
\DeclareMathOperator{\coker}{coker}
\DeclareMathOperator{\Med}{\mathcal{M}}
\newcommand{\Hom}{\mathrm{Hom}}
\newcommand{\End}{\mathrm{End}}
\newcommand{\coEnd}{\mathrm{coEnd}}
\newcommand{\xla}[1]{\xleftarrow{#1}}
\newcommand{\xra}[1]{\xrightarrow{#1}}
\newcommand{\defeq}{\stackrel{\mathrm{def}}{=}}

% comments
\newcommand{\anibal}[1]{\textcolor{blue}{\underline{Anibal}: #1}}

% mathrm
\newcommand{\rA}{\mathrm{A}}
\newcommand{\rB}{\mathrm{B}}
\newcommand{\rC}{\mathrm{C}}
\newcommand{\rD}{\mathrm{D}}
\newcommand{\rE}{\mathrm{E}}
\newcommand{\rF}{\mathrm{F}}
\newcommand{\rG}{\mathrm{G}}
\newcommand{\rH}{\mathrm{H}}
\newcommand{\rI}{\mathrm{I}}
\newcommand{\rJ}{\mathrm{J}}
\newcommand{\rK}{\mathrm{K}}
\newcommand{\rL}{\mathrm{L}}
\newcommand{\rM}{\mathrm{M}}
\newcommand{\rN}{\mathrm{N}}
\newcommand{\rO}{\mathrm{O}}
\newcommand{\rP}{\mathrm{P}}
\newcommand{\rQ}{\mathrm{Q}}
\newcommand{\rR}{\mathrm{R}}
\newcommand{\rS}{\mathrm{S}}
\newcommand{\rT}{\mathrm{T}}
\newcommand{\rU}{\mathrm{U}}
\newcommand{\rV}{\mathrm{V}}
\newcommand{\rW}{\mathrm{W}}
\newcommand{\rX}{\mathrm{X}}
\newcommand{\rY}{\mathrm{Y}}
\newcommand{\rZ}{\mathrm{Z}}
% mathcal
\newcommand{\cA}{\mathcal{A}}
\newcommand{\cB}{\mathcal{B}}
\newcommand{\cC}{\mathcal{C}}
\newcommand{\cD}{\mathcal{D}}
\newcommand{\cE}{\mathcal{E}}
\newcommand{\cF}{\mathcal{F}}
\newcommand{\cG}{\mathcal{G}}
\newcommand{\cH}{\mathcal{H}}
\newcommand{\cI}{\mathcal{I}}
\newcommand{\cJ}{\mathcal{J}}
\newcommand{\cK}{\mathcal{K}}
\newcommand{\cL}{\mathcal{L}}
\newcommand{\cM}{\mathcal{M}}
\newcommand{\cN}{\mathcal{N}}
\newcommand{\cO}{\mathcal{O}}
\newcommand{\cP}{\mathcal{P}}
\newcommand{\cQ}{\mathcal{Q}}
\newcommand{\cR}{\mathcal{R}}
\newcommand{\cS}{\mathcal{S}}
\newcommand{\cT}{\mathcal{T}}
\newcommand{\cU}{\mathcal{U}}
\newcommand{\cV}{\mathcal{V}}
\newcommand{\cW}{\mathcal{W}}
\newcommand{\cX}{\mathcal{X}}
\newcommand{\cY}{\mathcal{Y}}
\newcommand{\cZ}{\mathcal{Z}}
% mathsf
\newcommand{\sA}{\mathsf{A}}
\newcommand{\sB}{\mathsf{B}}
\newcommand{\sC}{\mathsf{C}}
\newcommand{\sD}{\mathsf{D}}
\newcommand{\sE}{\mathsf{E}}
\newcommand{\sF}{\mathsf{F}}
\newcommand{\sG}{\mathsf{G}}
\newcommand{\sH}{\mathsf{H}}
\newcommand{\sI}{\mathsf{I}}
\newcommand{\sJ}{\mathsf{J}}
\newcommand{\sK}{\mathsf{K}}
\newcommand{\sL}{\mathsf{L}}
\newcommand{\sM}{\mathsf{M}}
\newcommand{\sN}{\mathsf{N}}
\newcommand{\sO}{\mathsf{O}}
\newcommand{\sP}{\mathsf{P}}
\newcommand{\sQ}{\mathsf{Q}}
\newcommand{\sR}{\mathsf{R}}
\newcommand{\sS}{\mathsf{S}}
\newcommand{\sT}{\mathsf{T}}
\newcommand{\sU}{\mathsf{U}}
\newcommand{\sV}{\mathsf{V}}
\newcommand{\sW}{\mathsf{W}}
\newcommand{\sX}{\mathsf{X}}
\newcommand{\sY}{\mathsf{Y}}
\newcommand{\sZ}{\mathsf{Z}}
% mathbb
\newcommand{\bA}{\mathbb{A}}
\newcommand{\bB}{\mathbb{B}}
\newcommand{\bC}{\mathbb{C}}
\newcommand{\bD}{\mathbb{D}}
\newcommand{\bE}{\mathbb{E}}
\newcommand{\bF}{\mathbb{F}}
\newcommand{\bG}{\mathbb{G}}
\newcommand{\bH}{\mathbb{H}}
\newcommand{\bI}{\mathbb{I}}
\newcommand{\bJ}{\mathbb{J}}
\newcommand{\bK}{\mathbb{K}}
\newcommand{\bL}{\mathbb{L}}
\newcommand{\bM}{\mathbb{M}}
\newcommand{\bN}{\mathbb{N}}
\newcommand{\bO}{\mathbb{O}}
\newcommand{\bP}{\mathbb{P}}
\newcommand{\bQ}{\mathbb{Q}}
\newcommand{\bR}{\mathbb{R}}
\newcommand{\bS}{\mathbb{S}}
\newcommand{\bT}{\mathbb{T}}
\newcommand{\bU}{\mathbb{U}}
\newcommand{\bV}{\mathbb{V}}
\newcommand{\bW}{\mathbb{W}}
\newcommand{\bX}{\mathbb{X}}
\newcommand{\bY}{\mathbb{Y}}
\newcommand{\bZ}{\mathbb{Z}}
% mathfrak
\newcommand{\fA}{\mathfrak{A}}
\newcommand{\fB}{\mathfrak{B}}
\newcommand{\fC}{\mathfrak{C}}
\newcommand{\fD}{\mathfrak{D}}
\newcommand{\fE}{\mathfrak{E}}
\newcommand{\fF}{\mathfrak{F}}
\newcommand{\fG}{\mathfrak{G}}
\newcommand{\fH}{\mathfrak{H}}
\newcommand{\fI}{\mathfrak{I}}
\newcommand{\fJ}{\mathfrak{J}}
\newcommand{\fK}{\mathfrak{K}}
\newcommand{\fL}{\mathfrak{L}}
\newcommand{\fM}{\mathfrak{M}}
\newcommand{\fN}{\mathfrak{N}}
\newcommand{\fO}{\mathfrak{O}}
\newcommand{\fP}{\mathfrak{P}}
\newcommand{\fQ}{\mathfrak{Q}}
\newcommand{\fR}{\mathfrak{R}}
\newcommand{\fS}{\mathfrak{S}}
\newcommand{\fT}{\mathfrak{T}}
\newcommand{\fU}{\mathfrak{U}}
\newcommand{\fV}{\mathfrak{V}}
\newcommand{\fW}{\mathfrak{W}}
\newcommand{\fX}{\mathfrak{X}}
\newcommand{\fY}{\mathfrak{Y}}
\newcommand{\fZ}{\mathfrak{Z}}
\addbibresource{aux/usualpapers.bib}
\addbibresource{aux/mypapers.bib}
\usepackage{csquotes}
% !TEX root = ../cube_einfty.tex

\newcommand{\symmod}{\mathsf{Mod}_{\sym}}
\newcommand{\symbimod}{\mathsf{biMod}_{\sym}}
\newcommand{\graphs}{\mathfrak{G}}

\newcommand{\M}{\cM}
\newcommand{\UM}{{\forget(\M)}}
\newcommand{\symL}{{\M_{sl}}}
\newcommand{\USL}{{\forget(\M_{sl})}}
\newcommand{\symurj}{\cX}
\newcommand{\BE}{\cE}

\DeclareMathOperator{\schains}{N^{\simplex}}
\DeclareMathOperator{\cchains}{N^{\cube}}
\DeclareMathOperator{\scochains}{N^{\vee}_{\simplex}}
\DeclareMathOperator{\ccochains}{N^{\vee}_{\cube}}
\DeclareMathOperator{\Schains}{S}
\DeclareMathOperator{\sSchains}{\Schains^{\simplex}}
\DeclareMathOperator{\cSchains}{\Schains^{\cube}}
\DeclareMathOperator{\sScochains}{S^{\vee}_{\simplex}}
\DeclareMathOperator{\cScochains}{S^{\vee}_{\cube}}
\DeclareMathOperator{\sSing}{Sing^{\simplex}}
\DeclareMathOperator{\cSing}{Sing^{\cube}}

\newcommand{\scube}[1]{(\triangle^{\!1})^{\times #1}}
\DeclareMathOperator{\triangulate}{\mathcal{T}}
\DeclareMathOperator{\cubify}{\mathcal{U}}
\DeclareMathOperator{\ez}{\mathfrak{ez}}
\DeclareMathOperator{\sh}{\mathfrak{sh}}
\DeclareMathOperator{\cs}{\mathfrak{cs}}
\DeclareMathOperator{\CS}{CS}
\DeclareMathOperator{\fCS}{\mathfrak{CS}}
\DeclareMathOperator{\ccs}{\mathsf{cs}}
\DeclareMathOperator{\ci}{\mathsf{i}}
\DeclareMathOperator{\gi}{\mathfrak{i}}
\DeclareMathOperator{\CScomp}{\zeta}
\DeclareMathOperator{\subdivide}{\eta}

\newcommand{\fZ}{\mathfrak{Z}} % new cmds here
\addbibresource{aux/bibliography.bib} % new bibs here
\usetikzlibrary{arrows,decorations.markings}
\tikzset{myptr/.style={decoration={markings,mark=at position 1 with %
			{\arrow[scale=2,>=stealth]{>}}},postaction={decorate}}}
		
\newsavebox\preproduct
\begin{lrbox}{\preproduct}
	\begin{tikzpicture}[scale=.3]
	\draw (0,0)--(0,-.8);
	\draw (0,0)--(.5,.5);
	\draw (0,0)--(-.5,.5);
	\end{tikzpicture} 
\end{lrbox}
\newcommand{\product}{% <- this 'right of' is inherited; how to avoid?
	\usebox\preproduct}

\newsavebox\precoproduct
	\begin{lrbox}{\precoproduct}
		\begin{tikzpicture}[scale=.3]
		\draw (0,0)--(0,.8);
		\draw (0,0)--(.5,-.5);
		\draw (0,0)--(-.5,-.5);
		\end{tikzpicture}
	\end{lrbox}
\newcommand{\coproduct}{% <- this 'right of' is inherited; how to avoid?
	\usebox\precoproduct}

\newsavebox\preboundary
\begin{lrbox}{\preboundary}
	\begin{tikzpicture}[scale=.3]
	\draw (0,0)--(0,1.3);
	\draw (.5,0)--(.5,1.3);
	\draw [fill] (0,0) circle [radius=0.1];
	\draw (.9,.6)--(1.3,.6);
	\draw (1.7,0)--(1.7,1.3);
	\draw (2.2,0)--(2.2,1.3);
	\draw [fill] (2.2,0) circle [radius=0.1];
	\end{tikzpicture}
\end{lrbox}
\newcommand{\boundary}{% <- this 'right of' is inherited; how to avoid?
	\usebox\preboundary}

\newsavebox\precoboundary
\begin{lrbox}{\precoboundary}
	\begin{tikzpicture}[scale=.3]
	\draw (0,0)--(0,1.3);
	\draw (.5,0)--(.5,1.3);
	\draw [fill] (0,1.3) circle [radius=0.1];
	\draw (.9,.6)--(1.3,.6);
	\draw (1.7,0)--(1.7,1.3);
	\draw (2.2,0)--(2.2,1.3);
	\draw [fill] (2.2,1.3) circle [radius=0.1];
	\end{tikzpicture}
\end{lrbox}
\newcommand{\coboundary}{% <- this 'right of' is inherited; how to avoid?
	\usebox\precoboundary}

\newsavebox\precounit
\begin{lrbox}{\precounit}
	\begin{tikzpicture}[scale=.3]
	\draw (0,0)--(0,1.3);
	\draw [fill] (0,0) circle [radius=0.1];
	\end{tikzpicture}
\end{lrbox}
\newcommand{\counit}{% <- this 'right of' is inherited; how to avoid?
	\usebox\precounit}

\newsavebox\preidentity
\begin{lrbox}{\preidentity}
	\begin{tikzpicture}[scale=.3]
	\draw (0,0)--(0,1.3);
	\end{tikzpicture}
\end{lrbox}
\newcommand{\identity}{% <- this 'right of' is inherited; how to avoid?
	\usebox\preidentity}

\newsavebox\preunit
\begin{lrbox}{\preunit}
	\begin{tikzpicture}[scale=.3]
	\draw (0,0)--(0,1.3);
	\draw [fill] (0,1.3) circle [radius=0.1];
	\end{tikzpicture}
\end{lrbox}
\newcommand{\unit}{% <- this 'right of' is inherited; how to avoid?
	\usebox\preunit}

\newsavebox\preassociativity
\begin{lrbox}{\preassociativity}
	\begin{tikzpicture}[scale=.2]
	\path[draw] (0,0)--(0,1)--(-1,2);
	\draw (0,1)--(1,2);
	\draw (-.5,1.5)--(0,2);
	\draw (1.2,1)--(1.8,1);
	\path[draw] (3,0)--(3,1)--(2,2);
	\draw (3,1)--(4,2);
	\draw (3.5,1.5)--(3,2);	
	\end{tikzpicture}
\end{lrbox}
\newcommand{\associativity}{% <- this 'right of' is inherited; how to avoid?
	\usebox\preassociativity}

\newsavebox\precoassociativity
\begin{lrbox}{\precoassociativity}
	\begin{tikzpicture}[scale=.2]
	\path[draw] (0,0)--(0,-1)--(-1,-2);
	\draw (0,-1)--(1,-2);
	\draw (-.5,-1.5)--(0,-2);
	\draw (1.2,-1)--(1.8,-1);
	\path[draw] (3,0)--(3,-1)--(2,-2);
	\draw (3,-1)--(4,-2);
	\draw (3.5,-1.5)--(3,-2);
	\end{tikzpicture}
\end{lrbox}
\newcommand{\coassociativity}{% <- this 'right of' is inherited; how to avoid?
	\usebox\precoassociativity}

\newsavebox\preinvolution
\begin{lrbox}{\preinvolution}
	\begin{tikzpicture}[scale=.2]
	\path[draw] (0,0)--(0,.5)--(-.5,1)--(0,1.5)--(0,2);
	\path[draw] (0,.5)--(.5,1)--(0,1.5);
	\end{tikzpicture}
\end{lrbox}
\newcommand{\involution}{% <- this 'right of' is inherited; how to avoid?
	\usebox\preinvolution}

\newsavebox\preleftcounitality
\begin{lrbox}{\preleftcounitality}
	\begin{tikzpicture}[scale=.3]
	\draw (0,0)--(0,.8);
	\draw (0,0)--(.5,-.5);
	\draw (0,0)--(-.5,-.5);
	\draw [fill] (-.5,-.5) circle [radius=0.1];
	\draw (.7,0)--(1.1,0);
	\path[draw] (1.5,-.5)--(1.5,.8);
	\end{tikzpicture}
\end{lrbox}
\newcommand{\leftcounitality}{% <- this 'right of' is inherited; how to avoid?
	\usebox\preleftcounitality}

\newsavebox\preleftcounitcoproduct
\begin{lrbox}{\preleftcounitcoproduct}
	\begin{tikzpicture}[scale=.3]
	\draw (0,0)--(0,.8);
	\draw (0,0)--(.5,-.5);
	\draw (0,0)--(-.5,-.5);
	\draw [fill] (-.5,-.5) circle [radius=0.1];
	\end{tikzpicture}
\end{lrbox}
\newcommand{\leftcounitcoproduct}{% <- this 'right of' is inherited; how to avoid?
	\usebox\preleftcounitcoproduct}

\newsavebox\prerightcounitality
\begin{lrbox}{\prerightcounitality}
	\begin{tikzpicture}[scale=.3]
	\draw (0,0)--(0,.8);
	\draw (0,0)--(.5,-.5);
	\draw (0,0)--(-.5,-.5);
	\draw [fill] (.5,-.5) circle [radius=0.1];
	\draw (-.7,0)--(-1.1,0);
	\path[draw] (-1.5,-.5)--(-1.5,.8);
	\end{tikzpicture}
\end{lrbox}
\newcommand{\rightcounitality}{% <- this 'right of' is inherited; how to avoid?
	\usebox\prerightcounitality}

\newsavebox\prerightcounitcoproduct
\begin{lrbox}{\prerightcounitcoproduct}
	\begin{tikzpicture}[scale=.3]
	\draw (0,0)--(0,.8);
	\draw (0,0)--(.5,-.5);
	\draw (0,0)--(-.5,-.5);
	\draw [fill] (.5,-.5) circle [radius=0.1];
	\end{tikzpicture}
\end{lrbox}
\newcommand{\rightcounitcoproduct}{% <- this 'right of' is inherited; how to avoid?
	\usebox\prerightcounitcoproduct}

\newsavebox\preleftunitality
\begin{lrbox}{\preleftunitality}
	\begin{tikzpicture}[scale=.3]
	\draw (0,0)--(0,-.8);
	\draw (0,0)--(-.5,.5);
	\draw (0,0)--(.5,.5);
	\draw [fill] (.5,.5) circle [radius=0.1];
	\draw (.7,0)--(1.1,0);
	\path[draw] (1.5,.5)--(1.5,-.8);
	\end{tikzpicture}
\end{lrbox}
\newcommand{\leftunitality}{% <- this 'right of' is inherited; how to avoid?
	\usebox\preleftunitality}

\newsavebox\prerightunitality
\begin{lrbox}{\prerightunitality}
	\begin{tikzpicture}[scale=.3]
	\draw (0,0)--(0,-.8);
	\draw (0,0)--(-.5,.5);
	\draw (0,0)--(.5,.5);
	\draw [fill] (-.5,.5) circle [radius=0.1];
	\draw (-.7,0)--(-1.1,0);
	\path[draw] (-1.5,.5)--(-1.5,-.8);
	\end{tikzpicture}
\end{lrbox}
\newcommand{\rightunitality}{% <- this 'right of' is inherited; how to avoid?
	\usebox\prerightunitality}

\newsavebox\preproductcounit
\begin{lrbox}{\preproductcounit}
	\begin{tikzpicture}[scale=.3]
	\draw (0,0)--(0,-.8);
	\draw (0,0)--(.5,.5);
	\draw (0,0)--(-.5,.5);
	\draw [fill] (0,-.8) circle [radius=0.1];
	\end{tikzpicture}
\end{lrbox}
\newcommand{\productcounit}{% <- this 'right of' is inherited; how to avoid?
	\usebox\preproductcounit}

\newsavebox\preunitcoproduct
\begin{lrbox}{\preunitcoproduct}
	\begin{tikzpicture}[scale=.3]
	\draw (0,0)--(0,.8);
	\draw (0,0)--(.5,-.5);
	\draw (0,0)--(-.5,-.5);
	\draw [fill] (0,.8) circle [radius=0.1];
	\end{tikzpicture}
\end{lrbox}
\newcommand{\unitcoproduct}{% <- this 'right of' is inherited; how to avoid?
	\usebox\preunitcoproduct}

\newsavebox\preleibniz
\begin{lrbox}{\preleibniz}
	\begin{tikzpicture}[scale=.245]
	\draw (0,.3)--(0,-.3);
	\draw (0,.3)--(.5,.8);
	\draw (0,.3)--(-.5,.8);
	\draw (0,-.3)--(0,.3);
	\draw (0,-.3)--(.5,-.8);
	\draw (0,-.3)--(-.5,-.8);
	
	\draw (.7,0)--(1.1,0);
	\draw (2.1,.8)--(2.1,.3)--(1.7,0)--(1.7,-.8);
	\draw (2.1,.3)--(2.9,-.3);
	\draw (3.3,.8)--(3.3,0)--(2.9,-.3)--(2.9,-.8);
	
	\draw (3.8,0)--(4.1,0);
	\draw (4.6,.8)--(4.6,0)--(5,-.3)--(5,-.8);
	\draw (5,-.3)--(5.8,.3);
	\draw (5.8,.8)--(5.8,.3)--(6.2,0)--(6.2,-.8);	
	\end{tikzpicture}
\end{lrbox}
\newcommand{\leibniz}{% <- this 'right of' is inherited; how to avoid?
	\usebox\preleibniz}

\newsavebox\prebialgebra
\begin{lrbox}{\prebialgebra}
	\begin{tikzpicture}[scale=.5]
	\draw (0,.3)--(0,-.3);
	\draw (0,.3)--(.5,.8);
	\draw (0,.3)--(-.5,.8);
	\draw (0,-.3)--(0,.3);
	\draw (0,-.3)--(.5,-.8);
	\draw (0,-.3)--(-.5,-.8);
	
	\draw (.8,0)--(1.2,0);
	
	\draw (2.1,.8)--(2.1,.3)--(1.7,0)--(2.1,-.3)--(2.1,-.8);
	
	\draw (3.1,.8)--(3.1,.3)--(3.5,0)--(3.1,-.3)--(3.1,-.8);

	\draw (2.1,.3)--(3.1,-.3);
	\draw (2.1,-.3)--(2.5,-.06);
	\draw (3.1,.3)--(2.7,.06);	
	\end{tikzpicture}
\end{lrbox}
\newcommand{\bialgebra}{% <- this 'right of' is inherited; how to avoid?
	\usebox\prebialgebra}

\newsavebox\precommutativity
\begin{lrbox}{\precommutativity}
	\begin{tikzpicture}[scale=.27]
	\draw (.3,0)--(.3,-1);
	\draw (.3,0)--(.8,.5);
	\draw (.3,0)--(-.2,.5);
	
	\draw (1.2,-.4)--(1.8,-.4);
	
	\draw (3,-.3)--(3,-1);
	\draw (2.5,0)--(3,-.3);
	\draw (3.5,0)--(3,-.3);
	\draw (2.46,0)--(2.9,.18);
	\draw (3.1,.3)--(3.5,.5);
	\draw (3.5,0)--(2.5,.5);
		\end{tikzpicture} 
	\end{lrbox}
	\newcommand{\commutativity}{% <- this 'right of' is inherited; how to avoid?
		\usebox\precommutativity}	

\begin{document}
	\maketitle
	%Nota bene:  the next two commands erase pages numbers (the right page numbers will be add by the editors on the pdf)
	\cfoot{}
	\thispagestyle{empty}
	%%
	\vskip 25pt
	% Abstract in French and in English, followed by Keywords and MSC:
	\begin{adjustwidth}{0.5cm}{0.5cm}
		{\small
			{\bf R\'esum\'e.} Les cochaînes cubiques sont munies d'un produit associatif, dual à la diagonale de Serre, relevant la structure commutative graduée en cohomologie.
			Dans ce travail, nous introduisons par des méthodes combinatoires explicites une extension de ce produit à une structure $E_\infty$.
			Comme application, nous prouvons que l'application de Cartan--Serre, qui relie les cochaînes singulières cubiques et simpliciales d'espaces, est un quasi-isomorphisme de $E_\infty$-algèbres. \\
			{\bf Abstract.} Cubical cochains are equipped with an associative product, dual to the Serre diagonal, lifting the graded commutative structure in cohomology.
			In this work we introduce through explicit combinatorial methods an extension of this product to a full $E_\infty$-structure.
			As an application we prove that the Cartan--Serre map, which relates the cubical and simplicial singular cochains of spaces, is a quasi-isomorphism of $E_\infty$-algebras. \\
			{\bf Keywords.} Cubical sets, cochain complex, cup product, Cartan--Serre map, $E_\infty$-algebras, operads. \\
			{\bf Mathematics Subject Classification (2020).} 55N45, 18M70, 18M85.
		}
	\end{adjustwidth}

	% Content
	
\section{Introduction} \label{s:introduction}

Instead of simplices, in his groundbreaking work on fibered spaces Serre considered cubes as the basic shapes used to define cohomology, stating that:

\begin{displaycquote}[p.431]{serre1951homologie}
	Il est en effet evident que ces derniers se pretent mieux que les simplexes a l'etude des produits directs, et, a fortiori, des espaces fibres qui en sont la generalisation.
\end{displaycquote}

Cubical sets, a model for the homotopy category, were also considered by Kan \cite{kan1955abstract, kan1956abstract} before introducing simplicial sets, and have become important in Voevodsky's program for univalent foundations and homotopy type theory \cite{kapulkin2020straightening, mortberg2017cubical}, and nonabelian algebraic topology \cite{brown2011nonabelian}.

Other areas where cubical methods are important are applied topology, where cubical complexes are ubiquitous \cite{tomasz2004computational}, and geometric group theory where actions on certain cube complexes characterized combinatorially have central relevance \cite{gromov1987hyperbolic, agol2013haken}.

Cubical cochains are equipped with the \textit{Serre algebra structure}, a lift to the cochain level of the graded ring structure in cohomology.
Using an acyclic carrier argument it can be shown that this product is commutative up to coherent homotopies in a non-canonical way.
The goal of this work is to introduce an effective description of this derived structure in the form of an explicit $E_\infty$-algebra structure naturally extending the Serre product.
We use the combinatorial model of the $E_\infty$-operad $\UM$ obtained from the finitely presented prop $\M$ introduced in \cite{medina2020prop1}.
The resulting $\UM$-algebra structure on cubical cochains is induced from a natural $\M$-bialgebra structure on the cochains of standard cubes, which is determined by only three linear maps.
To our knowledge, this is the first effective construction of an $E_\infty$-algebra structure on cubical cochains.
Non-constructively, this result could be obtained using a lifting argument based on the cofibrancy of the reduced version of the operad $\UM$ in the category of operads, but this existence statement misses the rich combinatorial structure present in our effective construction.

As described in \cite{medina2020prop1}, the operad $\UM$ also acts on simplicial cochains extending the Alexander--Whitney algebra structure.
We use a construction of Cartan and Serre to relate these cubical and simplicial $E_\infty$-structures.
More specifically, in \cite[p. 442]{serre1951homologie}, Serre describes for any space $Z$ a natural quasi-isomorphism
\[
\CScomp_Z^\vee \colon \cochains(\cSing Z) \to \cochains(\sSing Z)
\]
between its simplicial and cubical singular cochains.
Furthermore, he states this to be a quasi-isomorphism of algebras with respect to the Serre and Alexander--Whitney structures.
In the present work we deduce from a statement at the level of general simplicial sets that $\CScomp_Z^\vee$ is in fact a quasi-isomorphisms of $E_\infty$-algebras.
More specifically, let $\cubify$ be the right adjoint to the triangulation functor from cubical to simplicial sets.
We construct a natural quasi-isomorphism of $E_\infty$-algebras
\[
\CScomp_Y^\vee \colon \ccochains(\cubify Y) \to \scochains(Y)
\]
for any simplicial set $Y$, which factor $\CScomp_Z^\vee$ when $Y = \sSing Z$.

We now mention three application of the contributions in this paper.
For every prime $p$, the mod $p$ cohomology of a space is equipped with natural stable endomorphisms known as Steenrod operations \cite{steenrod1962cohomology}.
Following an operadic viewpoint developed by May \cite{may1970general}, in \cite{medina2020maysteenrod} we effectively described a May--Steenrod structure on $\UM$, i.e., a compatible choice of elements in $\UM$ that represent Steenrod operations on the mod~$p$ homology of $\UM$-algebras.
Since, as proven in this article, cubical cochains are equipped with a $\UM$-structure, we use this May--Steenrod structure to extend the cubical cup-$i$ products of \cite{kadeishvili2003cupi} and \cite{pilarczyk2016cubical} to a family of cochain level representatives of Steenrod operations at every prime.
Furthermore, the effective nature of these constructions permitted the implementation of these Steenrod products in the computer algebra system \texttt{ComCh} \cite{medina2021computer}.

For a closed smooth manifold $M$, in \cite{medina2021flowing} we compared a cochain complex generated by manifolds with corners over $M$, and the complex of cubical cochains defined by a choice of cubulation of $M$.
We used a canonical vector field associated to the cubulation to compare multiplicatively these two models of ordinary cohomology, whose product structures are respectively given by transverse intersection and the Serre product.
With the explicit description introduced here of an $E_\infty$-structure on cubical cochains, we expect to build on this multiplicative comparison and, using a coherent family of vector fields, describe the corresponding $E_\infty$-structure on geometric cochains extending the transverse intersection product.
For more details regarding this geometric model of cohomology please consult \cite{medina2021foundations}.

Our construction of an $E_\infty$-algebra structure on cubical cochains is obtained by dualizing an $E_\infty$-coalgebra structure on cubical chains.
In the fifties, Adams introduced in \cite{adams1956cobar} a comparison map
\[
\theta_Z \colon \cobar S^\simplex(Z,z) \to S^\cube(\loops_z Z)
\]
from his cobar construction on the simplicial singular chains of a pointed space $(Z, z)$ to the cubical singular chains on its based loops space $\loops_z Z$.
This comparison map is a quasi-isomorphism of algebras, which was shown by Baues \cite{baues1998hopf} to be one of bialgebras by considering Serre's cubical coproduct.
In \cite{medina2021cobar} we generalize Baues result by constructing an $E_{\infty}$-bialgebra structure on the cobar construction of the coalgebra of singular chains, and proving that Adams' comparison map is a quasi-isomorphism of $E_{\infty}$-bialgebras.

\subsection*{Outline}

The required concepts from the theory of operads and props is reviewed in \cref{s:operads and props}, including the definition of the operad $\UM$.
\cref{s:action} contains our main contribution, where we define a natural $\M$-bialgebra structure on the chains of standard cubes, and from it a natural $\UM$-coalgebra structure on the chains of cubical sets.
The comparison between simplicial and cubical cochains is presented in \cref{s:the cartan-serre comparison map}, where we show that the Cartan--Serre comparison map is a morphism of $E_\infty$-algebras.
	
\section*{Acknowledgment}

We thank Clemens Berger, Greg Friedman, Chris Kapulkin, Peter May, Manuel Rivera, Paolo Salvatore, Dev Sinha, Dennis Sullivan, and Bruno Vallette for insightful discussion related to this project.

A.M-M. acknowledges financial support from Innosuisse grant 32875.1~IP-ICT-1 and the hospitality of the \textit{Laboratory for Topology and Neuroscience} at EPFL.


	
\section{Conventions and preliminaries} \label{s:preliminaries}

\subsection{Chain complexes}

Throughout this article $\k$ denotes a commutative and unital ring and we work over its associated symmetric monoidal category of chain complexes $(\Ch, \otimes, \k)$.
We denote the chain complex of $\k$-linear maps between two such by $\Hom(C, C^\prime)$ referring to the functor $\Hom(-, \k)$ as \textit{linear duality}.

\subsection{Kan and Yoneda extensions}

Given categories $\mathsf{B}$ and $\cC$ we denote their associated \textit{functor category} by $\Fun(\mathsf{B}, \cC)$.
Recall that a category is said to be \textit{small} if its objects and morphisms form sets.
We denote the category of small categories by $\Cat$.
A category is said to be \textit{cocomplete} if any functor to it from a small category has a colimit.
If $\mathsf{A}$ is small and $\mathsf{C}$ cocomplete, then the \textit{(left) Kan extension of $g$ along $f$} exists for any pair of functors $f$ and $g$ in the diagram below, and it is the initial object in $\Fun(\mathsf{B}, \mathsf{C})$ making
\begin{equation*}
\begin{tikzcd}[column sep=normal, row sep=normal]
\mathsf{A} \arrow[d, "f"'] \arrow[r, "g"] & \mathsf{C} \\
\mathsf{B} \arrow[dashed, ur, bend right] & \quad
\end{tikzcd}
\end{equation*}
commute.
Recall the \textit{Yoneda embedding}, the functor $\yoneda \colon \mathsf{A} \to \Fun(\mathsf{A}^\op, \Set)$ induced by the assignment
\[
a \mapsto \big( a^\prime \mapsto \mathsf{A}(a^\prime, a) \big).
\]
A Kan extension along the Yoneda embedding is referred to as a \textit{Yoneda extension}.
	
\section{Operads, props and \pdfEinfty-structures} \label{s:operads and props}

We now review the definition of the finitely presented prop $\M$ introduced in \cite{medina2020prop1} and whose associated operad is a model for the  $E_\infty$ operad.
Given its small number of generators and relations, is well suited to define $E_\infty$-structures.
In the next section we use this model to define natural $E_\infty$-structures on cubical chains and cochains.
We start by reviewing the basic material in the theory of operads and props, referring the reader to, for example, \cite{markl2008props} for a more complete treatment.

\subsection{Symmetric modules and bimodules}

Let $\S$ be the category whose objects are the natural numbers and whose set of morphisms between $m$ and $n$ is empty if $m \neq n$ and is otherwise the symmetric group $\S_n$.
A \textit{left $\S$-module} (resp. \textit{right} $\S$-\textit{module} or $\S$-\textit{bimodule}) is a functor from $\S$ (resp. $\S^\op$ or $\S \times \S^\op$) to $\Ch$.
In this paper we prioritize left module structures over their right counterparts.
As usual, taking inverses makes both perspectives equivalent.
We respectively denote by $\Smod$ and $\Sbimod$ the categories of left $\S$-modules and of $\S$-bimodules with morphisms given by natural transformations.

The group homomorphisms $\S_n \to \S_n \times \S_1$ induce a forgetful functor \[
\forget \colon \Sbimod \to \Smod
\]
defined explicitly on an object $\cP$ by $\forget(\cP)(r) = \cP(1, r)$ for $r \in \N$.
The similarly defined forgetful functor to right $\S$-modules will not be considered.

\subsection{Composition structures}

We can define \textit{operads} and \textit{props} by enriching $\S$-modules and $\S$-bimodules with certain composition structures.
For a complete presentation of these concepts we refer to Definition~11 and 54 of \cite{markl2008props}.
Intuitively, using examples defined in the next subsection, operads and props can be understood by abstracting the composition structure naturally present in the left $\S$-module $\End^C$ (or right $\S$-module $\End_C$), naturally an operad, and the $\S$-bimodule $\End^C_C$, naturally a prop.
We remark that the prop structure on $\cP$ restricts to an operad structure on $\forget(\cP)$.

\subsection{Representations}

Given a chain complex $C$ define
\begin{gather*}
\End^C(r) = \Hom(C, C^{\otimes r}), \qquad
\End_C(r) = \Hom(C^{\otimes r}, C), \\
\End^C_C(r, s) = \Hom(C^{\otimes r}, C^{\otimes s}),
\end{gather*}
for $r,s \in \N$, with their natural operad and prop structures respectively.
We remark that the forgetful functor $\forget$ sends $\End^C_C$ to $\End^C$.

Let $C$ be a chain complex, $\cO$ an operad, and $\cP$ a prop.
An $\cO$-\textit{coalgebra} (resp. $\cO$-\textit{algebra} or $\cP$-\textit{bialgebra}) structure on $C$ is a structure preserving morphism $\cO \to \End^C$ (resp. $\cO \to \End_C$ or $\cP \to \End_C^C$).

\subsection{\pdfEinfty-operads}

Recall that a \textit{free $\S_r$-resolution} of a chain complex $C$ is a quasi-isomorphism $R \to C$ from a chain complex $R$ of free $\k[\S_r]$-modules.

An $\S$-module $M$ is said to be $E_\infty$ if there exists a morphism of $\S$-modules $M \to \underline{\k}$ inducing for each $r \in \N$ a free $\S_r$-resolution $M(r) \to \k$.
For example, we can obtain one such $\S$-module by using the functor of singular chains and the set $\{\mathrm{E} \S_r \to \ast\}_{n \in \N}$ of maps to the terminal space from models of the universal bundle.

An operad is said to be $E_\infty$ if its underlying $\S$-module is $E_\infty$.

\subsection{Free prop construction} \label{ss:free prop}

\begin{figure}
	\boxed{\begin{tikzpicture}[scale=.6]
\draw (1,3.7) to (1,3); 

\draw (1,3) to [out=205, in=90] (0,0);

\draw [shorten >= 0cm] (.6,2.73) to [out=-100, in=90] (2,0);

\draw [shorten >= .15cm] (1,3) to [out=-25, in=30, distance=1.1cm] (1,1.5);
\draw [shorten <= .1cm] (1,1.5) to [out=210, in=20] (0,1);

\node at (1,3.9){};
\node at (0,-.32){};
\node at (2,-.32){};

\node at (3,1.5){$\sim$\ \ \ };
\end{tikzpicture}
\begin{tikzpicture}[scale=.6]
\draw (1,3.7) to (1,3); 

\draw [->](1,3) to [out=205, in=90] (0,0);

\draw [shorten >= 0cm,->] (.6,2.73) to [out=-100, in=90] (2,0);

\draw [shorten >= .15cm] (1,3) to [out=-25, in=30, distance=1.1cm] (1,1.5);
\draw [shorten <= .1cm] (1,1.5) to [out=210, in=20] (0,1);


\def\x{.8}

\node[scale=\x] at (1,3.9){$\scriptstyle 1$};

\node[scale=\x] at (.7,3.05){$\scriptstyle 1$};
\node[scale=\x] at (1.35,3.05){$\scriptstyle 2$};

\node[scale=\x] at (.1,2.3){$\scriptstyle 1$};
\node[scale=\x] at (.8,2.3){$\scriptstyle 2$};

\node[scale=\x] at (-.15,1.3){$\scriptstyle 1$};
\node[scale=\x] at (.3,1.3){$\scriptstyle 2$};

\node[scale=\x] at (0,-.3){$\scriptstyle 1$};
\node[scale=\x] at (2,-.3){$\scriptstyle 2$};
\end{tikzpicture}}
	\caption{Immersed graphs represent labeled directed graphs with the direction implicitly given from top to bottom and the labeling from left to right.}
	\label{f:immersion}
\end{figure}

The \textit{free prop} $\free(M)$ generated by an $\S$-bimodule $M$ is constructed using isomorphism classes of directed graphs with no directed loops that are enriched with the following labeling structure.
We think of each directed edge as built from two compatibly directed half-edges.
For each vertex $v$ of a directed graph $\Gamma$, we have the sets $in(v)$ and $out(v)$ of half-edges that are respectively incoming to and outgoing from $v$.
Half-edges that do not belong to $in(v)$ or $out(v)$ for any $v$ are divided into the disjoint sets $in(\Gamma)$ and $out(\Gamma)$ of incoming and outgoing external half-edges.
For any positive integer $n$ let $\overline{n} = \{1, \dots, n\}$ and set $\overline{0} = \emptyset$.
For any finite set $S$, denote the cardinality of $S$ by $|S|$.
The labeling is given by bijections
\[
\overline{|in(\Gamma)|}\to in(\Gamma), \qquad
\overline{|out(\Gamma)|}\to out(\Gamma),
\]
and
\[
\overline{|in(v)|}\to in(v), \qquad
\overline{|out(v)|}\to out(v),
\]
for every vertex $v$.
We refer to the isomorphism classes of such labeled directed graphs with no directed loops as $(n,m)$\textit{-graphs} denoting the set of these by $\graphs(m,n)$.
We use graphs immersed in the plane to represent elements in $\graphs(m,n)$, please see \cref{f:immersion}.
We consider the right action of $\S_n$ and the left action of $\S_m$ on a $(n,m)$-graph given respectively by permuting the labels of $in(\Gamma)$ and $out(\Gamma)$.
This action defines the $\S$-bimodule structure on the free prop
\begin{equation} \label{e:free prop}
\free(M)(m,n) \ = \bigoplus_{\Gamma \in \graphs(m,n)} \bigotimes_{v \in Vert(\Gamma)} out(v) \otimes_{\S_q} M(p, q) \otimes_{\S_p} in(v),
\end{equation}
where we simplified the notation writing $p$ and $q$ for $\overline{|in(v)|}$ and $\overline{|out(v)|}$ respectively.
The composition structure is defined by (relabeled) grafting and disjoint union.

\subsection{The prop $\M$}

We now recall the model of $E_\infty$ that is central to our constructions.

\begin{definition}
	Let $\M$ be the prop generated by
	\begin{equation} \label{e:generators of M}
	\counit\,, \hspace*{.6cm} \coproduct\,, \hspace*{.6cm} \product,
	\end{equation}
	in degrees $0$, $0$ and $1$ respectively, and boundaries
	\begin{equation} \label{e:boundary of M}
	\partial\ \counit = 0,
	\hspace*{.6cm}
	\partial\, \coproduct = 0,
	\hspace*{.6cm}
	\partial \product = \ \boundary\,,
	\end{equation}
	modulo the prop ideal generated by
	\begin{equation} \label{e:relations of M}
	\leftcounitality\,, \hspace*{.6cm} \rightcounitality\,, \hspace*{.6cm} \productcounit.
	\end{equation}
\end{definition}

Explicitly, any element in $\M(m,n)$ can be written as a linear combination of the $(m,n)$-graphs generated by those in \eqref{e:generators of M} via grafting, disjoint union and relabeling, modulo the prop ideal generated by the relations in \eqref{e:relations of M}. Its boundary is determined, using \eqref{e:free prop}, by \eqref{e:boundary of M}.

As proven in \cite[Theorem 3.3]{medina2020prop1} we have the following.

\begin{proposition}
	The operad $\UM$ is $E_\infty$.
\end{proposition}

We remark that, as proven in \cite{medina2018prop2}, this prop is obtained from applying the functor of cellular chains to a finitely presented prop over the category of CW-complexes.

	
\section{An $E_\infty$ structure on cubical chains} \label{s:action}

In this section we construct a natural $\M$-bialgebra structure on the chains of standard cubes $\chains(\cube^n)$.
These are determined by three natural linear maps satisfying the relations defining $\mathcal M$.
A Kan extension argument then provides the chains of any cubical set with a natural $U(\M)$ coalgebra structure.

\subsection{Cellular chains}

The chain complex $\chains(\cube^n)$ is by definition equal to $\chains(\cube^1)^{\otimes n}$, which is isomorphic to the chain complex $C(\gcube)^{\otimes n}$, where $C(\gcube)$ is the complex of cellular chains on the standard interval.
We will use the notation $x_1 \otimes \cdots \otimes x_n$ with $x_i \in \{[0], [0,1], [1]\}$ for the elements in its basis, and we remark that this chain complex is also isomorphic to $\gchains(\gcube^n)$, the cellular chains of the geometric $n$-cube with its standard CW structure.

\subsection{Counit, coproduct and product}

For $n \in \mathbb{N}$, define: \vspace*{5pt} \\
(1) The \textit{counit} $\epsilon \in \Hom(\chains(\square^n), \Z)$ known as the \textit{augmentation} by
\begin{equation*}
\epsilon \left( x_1 \otimes \cdots \otimes x_d \right) = \epsilon(x_1) \, \cdots \, \epsilon(x_n),
\end{equation*}
where
\begin{equation*}
\epsilon([0]) = \epsilon([1]) = 1, \qquad \epsilon([0, 1]) = 0.
\end{equation*} \vspace*{-6pt} \\
(2) The \textit{coproduct} $\Delta \in \Hom \left( \chains(\square^n), \chains(\square^n)^{\otimes 2} \right)$ known as the \textit{Serre coproduct} by
\begin{equation*}	
\Delta (x_1 \otimes \cdots \otimes x_n) = 	
\sum \pm \left( x_1^{(1)} \otimes \cdots \otimes x_n^{(1)} \right) \otimes 	
\left( x_1^{(2)} \otimes \cdots \otimes x_n^{(2)} \right),	
\end{equation*}	
where the sign is determined using the Koszul convention, and we are using Sweedler's notation
\begin{equation*}	
\Delta(x_i) = \sum x_i^{(1)} \otimes x_i^{(2)}
\end{equation*}
for the chain map $\Delta \colon \chains(\square^1) \to \chains(\square^1)^{\otimes 2}$ defined by
\begin{equation*}
\Delta([0]) = [0] \otimes [0], \quad \Delta([1]) = [1] \otimes [1], \quad \Delta([0, 1]) = [0] \otimes [0, 1] + [0, 1] \otimes [1].
\end{equation*}
Using that $\chains(\square^n) = \chains(\square^1)^{\otimes n}$, $\Delta$ is the composition
\begin{equation*}
\begin{tikzcd}
\chains(\square^1)^{\otimes n} \arrow[r, "\Delta^{\otimes n}"] &[3pt] \left( \chains(\square^1)^{\otimes 2}  \right)^{\otimes n} \arrow[r, "sh"] &[-5pt] \left( \chains(\square^1)^{\otimes n} \right)^{\otimes 2}
\end{tikzcd}
\end{equation*}
where $sh$ is the shuffle map that places tensor factors in odd position first. \vspace*{5pt} \\
(3) The \textit{product} $\ast \in \Hom(\chains(\square^n)^{\otimes 2}, \chains(\square^n))$ by
\begin{align*}
(x_1 \otimes \cdots \otimes x_n) \ast (y_1 \otimes \cdots \otimes y_n) =
(-1)^{|x|} \sum_{i=1}^n x_{<i}\, \epsilon(y_{<i}) \otimes x_i \ast y_i \otimes \epsilon(x_{>i}) \, y_{>i},
\end{align*}
where
\begin{align*}
x_{<i} & = x_1 \otimes \cdots \otimes x_{i-1}, &
y_{<i} & = y_1 \otimes \cdots \otimes y_{i-1}, \\
x_{>i} & = x_{i+1} \otimes \cdots \otimes x_n, & 
y_{>i} & = y_{i+1} \otimes \cdots \otimes y_n,
\end{align*}
with the convention
\begin{equation*}
x_{<1} = y_{<1} = x_{>n} = y_{>n} = 1 \in \Z,
\end{equation*}
and the only non-zero values of $x_i \ast y_i$ are
\begin{equation*}
\ast([0] \otimes [1]) = [0, 1], \qquad  \ast([1] \otimes [0]) = -[0, 1].
\end{equation*}

\subsection{Main construction} \label{ss:main construction}

The following is the main technical result of this paper.
Its proof is given in \cref{ss:proof action}.

\begin{lemma} \label{l:cubical chain bialgebra}
	The assignment
	\begin{equation*}
	\counit \mapsto \epsilon, \quad \coproduct \mapsto \Delta, \quad \product \mapsto \ast,
	\end{equation*}
	induces natural $\mathcal M$-bialgebra structure on $\chains(\square^n)$ for every $n \in \mathbb{N}$, or, equivalently, a functor $\cube \to \biAlg_{\M}$.
\end{lemma}

The category of bialgebras over a prop is in general not cocomplete, but those of algebras and coalgebras over operads are.
So we have the following result, the main contribution of this paper.

\begin{theorem} \label{t:lift to e infinity coalgebras}
	Composing the functor defined in \cref{l:cubical chain bialgebra} with the forgetful functor $\biAlg_{\M} \to \coAlg_{U(\M)}$ defines a functor $\cube \to \coAlg_{U(\M)}$ whose Kan extension endows the chains of a cubical set with a natural $E_\infty$ coalgebra structure extending the Serre coproduct.	
\end{theorem}

By linear duality, the same argument can be used to define a natural $E_\infty$ algebra structure on cubical cochains extending the Serre product.
	\section{The Cartan--Serre map} \label{s:comparison}

Let us consider, with their usual CW structures, the topological simplex $\gsimplex^n$ and the topological cube $\gcube^n$.
In \cite[p. 442]{serre1951homologie}, Serre described a quasi-isomorphism of coalgebras between the simplicial and cubical singular chains of a topological space.
It is given by precomposing with a canonical cellular map $\cs \colon \gcube^n \to \gsimplex^n$ also considered in \cite[p.199]{eilenberg1953acyclic} where it is attributed to Cartan.

The goal of this section is to deduce from a more general categorical statement that this comparison map between singular chains of a space is a quasi-isomorphism of $E_\infty$-coalgebras.

\subsection{Simplicial sets} \label{ss:simplicial sets}

We denote the \textit{simplex category} by $\simplex$, the category $\Fun(\simplex^\op, \Set)$ of \textit{simplicial sets} by $\sSet$, and the representable simplicial set $\yoneda \big( [n] \big)$ by $\simplex^n$.
As usual, we denote an element in $\simplex^n_m$ by a non-decreasing tuple $[v_0, \dots, v_m]$ with $v_i \in \{0, \dots, n\}$.
The \textit{Cartesian product} of simplicial sets is defined by the product of functors.
The \textit{simplicial $n$-cube} $\scube{n}$ is the $n^\th$-fold Cartesian product of $\simplex^1$ with itself.

We will use the following model of the topological $n$-simplex:
\[
\gsimplex^n = \big\{ (y_1, \dots, y_n) \in \gcube^n \mid i \leq j \Rightarrow y_i \geq y_j \big\},
\]
whose cell structure associates $[v_0, \dots, v_m]$ with the subset
\begin{equation} \label{e:cell structure of gsimplex}
	\Big\{ \big( \underbrace{1, \dots, 1}_{v_0}, \underbrace{y_1, \dots y_1}_{v_1-v_0}, \dots, \underbrace{y_m, \dots y_m}_{v_m-v_{m-1}}, \underbrace{0, \dots, 0}_{n-v_m} \big) \mid y_1 \geq \dots \geq y_m \Big\}.
\end{equation}
The spaces $\gsimplex^n$ define a functor $\simplex \to \CW$ with
\begin{align*}
	\sigma_i(x_1, \dots, x_n) &= (x_1, \dots, \widehat x_i, \dots, x_n) \\
	\delta_0(x_1, \dots, x_n) &= (1, x_1, \dots, x_n), \\
	\delta_i(x_1, \dots, x_n) &= (x_1, \dots, x_i, x_i, \dots, x_n), \\
	\delta_n(x_1, \dots, x_n) &= (x_1, \dots, x_n, 0).
\end{align*}
Its Yoneda extension is the \textit{geometric realization} functor.
It has a right adjoint $\sSing \colon \Top \to \sSet$ referred to as the \textit{simplicial singular complex} satisfying
\[
\sSing(\fZ)_n = \Top(\gsimplex^n, \fZ)
\]
for any topological space $\fZ$.

The functor of (\textit{normalized}) \textit{chains} $\schains \colon \sSet \to \Ch$ is the composition of the geometric realization functor and that of cellular chains.
We denote the composition $\schains \circ \sSing$ by $\sSchains$ and omit the superscript $\simplex$ if no confusion may result from doing so.
For any $n \in \N$, the \textit{Alexander--Whitney coalgebra structure} on $\chains(\simplex^n)$ is given by
\[
\Delta \big( [v_0, \dots, v_m] \big) = \sum_{i=0}^m [v_0, \dots, v_i] \ot [v_i, \dots, v_m],
\]
and
\[
\epsilon \big( [v_0, \dots, v_m] \big) =
\begin{cases}
	1 & \text{ if } m = 0, \\ 0 & \text{ if } m>0.
\end{cases}
\]
The degree 1 product $\ast \colon \chains(\simplex^n)^{\ot 2} \to \chains(\simplex^n)$ is defined by
\begin{align*}
	\begin{autobreak}
		\left[v_0, \dots, v_p \right] \ast
		\left[v_{p+1}, \dots, v_m\right] =
		{\begin{cases}
			(-1)^{p+|\sigma|} \left[v_{\sigma(0)}, \dots, v_{\sigma(m)}\right] &
			\text{ if } v_i \neq v_j \text{ for } i \neq j, \\
			0 & \text{ if not},
		\end{cases}}
	\end{autobreak}
\end{align*}
where $\sigma$ is the permutation that orders the totally ordered set of vertices and $(-1)^{|\sigma|}$ is its sign.
As shown in \cite[Theorem 4.2]{medina2020prop1} the assignment
\[
\counit \mapsto \epsilon, \quad \coproduct \mapsto \Delta, \quad \product \mapsto \ast,
\]
defines a natural $\M$-bialgebra on the chains of representable simplicial sets, and, by forgetting structure, also a natural $\UM$-coalgebra.
For any simplicial set, a natural $\UM$-coalgebra structure on its chains is defined by a Yoneda extension.

\subsection{The Eilenberg--Zilber maps} \label{ss:ez}

For any permutation $\sigma \in \sym_n$ let
\[
\gi_\sigma \colon \gsimplex^n \to \gcube^n
\]
be the inclusion defined by $(x_1, \dots, x_n) \mapsto (x_{\sigma(1)}, \dots, x_{\sigma(n)})$.
If $e$ is the identity permutation, we denote $\mathfrak{i}_{e}$ simply as $\mathfrak{i}$.
%and notice that it satisfies $\cs \circ\, \mathfrak{i} = \id_{\gsimplex^n}$.
The maps $\set{\mathfrak{i}_\sigma}_{\sigma \in \sym_n}$ define a subdivision of $\gcube^n$ making it isomorphic to $\bars[\big]{\scube{n}}$ in $\CW$.
Using this identification, the identity map induces a cellular map
\[
\ez \colon \gcube^n \to \bars[\big]{\scube{n}}.
\]
We denote the induced chain map by
\[
\EZ \colon \chains(\cube^n) \to \chains\big(\scube{n}\big).
\]
For any topological space $\fZ$, the cubical map
\[
\cubify\sSing(\fZ) \to \cSing(\fZ)
\]
is defined, using the adjunction isomorphism
\[
\sSet\big(\scube{n}, \sSing(\fZ)\big) \cong
\Top\big(\bars{\scube{n}}, \fZ\big),
\]
by the assignment
\[
\big( \bars{\scube{n}} \xra{f} \fZ \big) \mapsto
\big( \gcube^n \xra{\ez} \bars{\scube{n}} \xra{f} \fZ \big).
\]
We denote the induced chain map by
\[
\EZ_{\Schains(\fZ)} \colon
\cchains\!\big(\cubify\sSing(\fZ)\big) \to \cSchains(\fZ).
\]
%Please note that $\nu_\fZ$ is a cubical map given the naturality of $\ez$.

%% POSSIBLE REMARK
%Using the isomorphism $\chains(\cube^n) \cong \chains(\cube^1)^{\ot n} \cong \chains(\simplex^1)^{\ot n}$, this map agrees with the usual Eilenberg--Zilber maps $\chains(\simplex^1)^{\ot n} \to \chains \big( \scube{n} \big)$.

%% EXPLORE LATER
%For any topological space $\fZ$ we have a quasi-isomorphism
%\begin{equation} \label{e:ezz}
%	\begin{tikzcd} [column sep=small, row sep=0]
%		\EZ_{\Schains(\fZ)} \colon &[-20pt] \cSchains(\fZ) \arrow[r] & \sSchains (\fZ) \\
%		& (\gcube^n \to \fZ) \arrow[r, mapsto] &
%		\sum\limits_{\mathclap{\sigma \in \sym_n}} \sign(\sigma) \big( \gsimplex^n \xra{\gi_\sigma} \gcube^n \to \fZ \big).
%	\end{tikzcd}
%\end{equation}

\subsection{The Cartan--Serre maps}

The cellular map
\[
\cs \colon \gcube^n \to \gsimplex^n
\]
is defined by
\[
\cs(x_1, \dots, x_n) =
(x_1,\ x_1 x_2, \, \dots \, , \ x_1 x_2 \dotsm x_n).
\]
We denote its induced chain map by
\[
\CS \colon \chains(\cube^n) \to \chains(\simplex^n).
\]
%by applying the functor of cellular chains to $\cs \colon \gcube^n \to \gsimplex^n$.
The chain map
\[
\CS_{\Schains(\fZ)} \colon \sSchains(\fZ) \to \cSchains(\fZ)
\]
between the singular chain complexes of a topological space $\fZ$ is defined by
\[
\CS_{\Schains(\fZ)} (\gsimplex^n \to \fZ) =
(\gcube^n \xra{\cs} \gsimplex^n \to \fZ).
\]
%\[
%\begin{tikzcd} [column sep=small, row sep=0]
%	\CS_{\Schains(\fZ)} \colon &[-25] \sSchains(\fZ) \arrow[r] & \cSchains (\fZ) \\ &
%	(\gsimplex^n \to \fZ) \arrow[r, mapsto] & (\gcube^n \xra{\cs} \gsimplex^n \to \fZ).
%\end{tikzcd}
%\]

These maps were considered in \cite[p. 442]{serre1951homologie} where it was stated that $\CS_{\Schains(\fZ)}$ is a natural quasi-isomorphisms of coalgebras.
We will prove this in \cref{ss:e-infty preservation} showing in fact that it is a quasi-isomorphism of $E_\infty$-coalgebras.

\subsection{No-go results}

Since $\CS$ is shown to be a coalgebra map in \cref{l:cs coalgebra map} and $\EZ$ is well known to be one, one may hope for higher structures to be preserved by these maps.
We now provide some examples constraining the scope of these expectations.

\begin{example*}
	We will show that $\EZ$ does not preserve $\UM$-structures.
	More specifically, that in general
	\[
	\EZ^{\ot 2} \circ \, \Delta_1 \neq \Delta_1 \circ \EZ
	\]
	where
	\[
	\Delta_1 = (\ast \ot \id) \circ (\id \ot (12) \Delta) \circ \Delta
	\]
	is the cup-1 coproduct presented in \cref{e:closed cup-i}.
	Up to signs, on one hand we have
	\begin{align*}
		\begin{autobreak}
			\Delta_1 \big( [01] [01] \big)
			= [01][01] \ot [1][01]
			\ +\ [01][1] \ot [01][01]
			\ +\ [0][01] \ot [01][01]
			\ +\ [01][01] \ot [01][0].
		\end{autobreak}
	\end{align*}
%	\begin{multline*}
%		\Delta_1 \big( [01] [01] \big) = \\
%		[01][01] \ot [1][01] + [01][1] \ot [01][01] + [0][01] \ot [01][01] + [01][01] \ot [01][0].
%	\end{multline*}
	Therefore,
	\begin{align*}
		\begin{autobreak}
			\EZ^{\ot 2} \circ \, \Delta_1 \big( [01][01] \big)
			= \big( 011\times001 + 001\times011 \big) \ot 11 \times 01
			\ +\ 01\times11 \ot \big( 011\times001+ 001\times011 \big)
			\ +\ 00\times01 \ot \big( 011\times001 + 001\times011 \big)
			\ +\ \big( 011\times001 + 001\times011 \big) \ot 01\times00.
		\end{autobreak}
	\end{align*}
%	\begin{multline*}
%		\EZ^{\ot 2} \circ \, \Delta_1 \big( [01][01] \big) =
%		\big( 011\times001 + 001\times011 \big) \ot 11 \times 01 \ +\
%		01\times11 \ot \big( 011\times001 + 001\times011 \big) \ +\ \\
%		00\times01 \ot \big( 011\times001 + 001\times011 \big) \ +\
%		\big( 011\times001 + 001\times011 \big) \ot 01\times00.
%	\end{multline*}
	On the other hand, we have
	\[
	\Delta_1 [0,1,2] = [0,1,2] \ot [0,1] + [0,2] \ot [0,1,2] + [0,1,2] \ot [1,2].
	\]
	Therefore,
%	\begin{multline*}
%		\Delta_1 \circ \EZ \big( [01] [01] \big) = \Delta_1 \big( 011\times001 + 001\times011 \big) = \\
%		011\times001 \ot 01\times00 \ + \
%		01\times01 \ot 011\times001 \ + \
%		011\times001 \ot 11\times01 \ + \ \\
%		001\times011 \ot 00\times01 \ + \
%		01\times01 \ot 001\times011 \ + \
%		001\times011 \ot 01\times11.
%	\end{multline*}
	\begin{align*}
		\begin{autobreak}
			\Delta_1 \circ \EZ \big( [01] [01] \big)
			= \Delta_1 \big( 011\times001 + 001\times011 \big)
			= 011\times001 \ot 01\times00
			\ + \ 01\times01 \ot 011\times001
			\ + \ 011\times001 \ot 11\times01
			\ + \ 001\times011 \ot 00\times01
			\ + \ 01\times01 \ot 001\times011
			\ + \ 001\times011 \ot 01\times11.
		\end{autobreak}
	\end{align*}
	We conclude that
	\[
	\EZ^{\ot 2} \circ \, \Delta_1 \big( [01][01] \big) \neq \Delta_1 \circ \EZ \big( [01] [01] \big)
	\]
	since, for example, the basis element $01\times11 \ot 011\times001$ appears in the left sum but not in the right one.
\end{example*}

\begin{example*}
	We will show that the Cartan--Serre map does not preserve $\M$-structures.
	More specifically, that in general
	\[
	\CS(x \ast y) \neq \CS(x) \ast \CS(y).
	\]
	Consider $x = [1][1]$ and $y = [0][01]$.
	On one hand we have that
	\[
	\CS \big( [1][1] \big) \ast \CS \big( [0][01] \big) = 0
	\]
	since $\CS \big( [0][01] \big) = 0$.
	On the other hand we have, up to a signs, that
	\[
	\CS \Big( ([1][1]) \ast ([0][01]) \Big) =
	\CS \Big( [01][01] \Big) = [012],
	\]
	which establishes the claim.
\end{example*}

The reason for this incompatibility is that $\ast$ in the simplicial context is commutative, which is not the case in the cubical one.

\begin{example*}
	We will show that the Cartan--Serre map does not preserve $\UM$-structures.
	More specifically, that in general
	\[
	\CS \circ\, \widetilde\Delta_1 \neq
	\widetilde\Delta \circ \CS
	\]
	where
	\[
	\widetilde\Delta_1 =
	(\ast \ot \id) \circ (12) (\id \ot (12) \Delta) \circ \Delta.
	\]
	On one hand we have that
	\[
	\CS\Big( \widetilde\Delta_1 \big( [01][01] \big) \Big) =
	T \Delta_1 \big( [012] \big),
	\]
	and on the other that
	\[
	\widetilde\Delta_1 \circ \CS \big( [01][01] \big) =
	\Delta_1 \big( [012] \big),
	\]
	which establishes the claim.
\end{example*}

In \cref{ss:e-infty preservation} we will show that $\CS$ is a morphism of $E_\infty$-coalgebras.
To do so we now introduce an $E_\infty$-suboperad of $\UM$ where the incompatibility resulting from the lack of commutativity of $\ast$ in the cubical context is dealt with.

\subsection{Shuffle graphs} \label{ss:shuffle graphs}

Consider $k = k_1+\dots+k_r$.
A $(k_1,\dots,k_r)$-shuffle $\sigma$ is a permutation in $\sym_{k}$ satisfying
\begin{align*}
	&\sigma(1) < \dots < \sigma(k_1), \\
	&\sigma(k_1+1) < \dots < \sigma(k_1+k_2), \\
	&\qquad \vdots \\
	&\sigma(k-k_r+1) < \dots < \sigma(k).
\end{align*}
The (\textit{left comb}) \textit{shuffle graph} associated to such $\sigma$ is the $(1,k)$-graph
\[
\boxed{\documentclass{standalone}
\usepackage{tikz}
\usepackage{amsmath}

\begin{document}
	\begin{tikzpicture}[scale=1.5]
		\coordinate (o) at (0,0);
		\draw (o)--(0,-.4) node[below, scale=.7]{$1$};
		\draw (o)--(-.6,.6) node[above, scale=.7]{$1$\hspace{0pt}};
		\draw (-.4,.4)--(-.2,.6) node[above, scale=.7]{$2$\hspace{0pt}};
		\draw (o)--(.6,.6) node[above, scale=.7]{\hspace{0pt} $k_1$};
		\node[scale=.7] at (.1,.5){...};
		\node[scale=.7] at (.2,.72){...};
		\node at (1.1,0){$\dots$};
	\end{tikzpicture}
	\hspace*{-16pt}
	\begin{tikzpicture}[scale=1.5]
		\coordinate (o) at (0,0);
		\draw (o)--(0,-.4) node[below, scale=.7]{$r$};
		\draw (o)--(-.6,.6) node[above, scale=.7]{$k-k_r+1$\hspace*{11pt}};
		\draw (-.15,.15)--(.3,.6) node[above, scale=.7]{$k-1$\hspace*{1pt}};
		\draw (o)--(.6,.6) node[above, scale=.7]{\hspace*{5pt} $k$};
		\node[scale=.7] at (-.15,.5){...};
		\node[scale=.7] at (-.1,.74){...};
		\node at (1,.05){$\circ$};
	\end{tikzpicture}
	\begin{tikzpicture}[scale=1.5]
		\coordinate (o) at (0,0);
		\draw (o)--(0,.4) node[scale=.7, above]{$1$};
		\draw (o)--(-1,-.5) node[scale=.7, below]{$\sigma^{-1}(1)$\hspace*{8pt}};
		\draw (-.75,-.375)--(-.5,-.5) node[scale=.7, below]{\hspace*{2pt}$\sigma^{-1}(2)$};
		\draw (-.5,-.25)--(0,-.5) node[scale=.7, below]{\hspace*{10pt}$\sigma^{-1}(3)$};
		\draw (o)--(1,-.5) node[scale=.7, below]{$\sigma^{-1}(k)$};
		\node[scale=.7] at (.15,-.3){...};
		\node[scale=.7] at (.55,-.65){...};
	\end{tikzpicture}
\end{document}}
\]
presented as a composition of (left comb) self-graftings of the generators $\product$ and $\coproduct$.
With the notation introduced in \cref{e:iterated comb}, the $\UM$-coalgebra sends the shuffle graph associated to $\sigma$ to
\[
(\ast^{k_1} \ot \dotsb \ot \ast^{k_r}) \circ \sigma^{-1} \Delta^{k-1}.
\]

\begin{example*}
	All the graphs in \cref{f:cup-i} are shuffle graphs.
	In fact, all the cup-$i$ coproducts presented in \cref{e:closed cup-i} are induced from shuffle graphs, whereas
	\begin{align*}
		\widetilde\Delta_1 &=
		(\ast \ot \id) \circ (12) (\id \ot (12) \Delta) \circ \Delta \\ &=
		(\ast \ot \id) \circ (123) \Delta^2,
	\end{align*}
	used in the previous section to probe the limits of the structure preserving properties of $\CS$, is not.
\end{example*}

The operad $\UMsh$ is defined as the suboperad of $\UM$ (freely) generated by shuffle graphs.
Explicitly, any element in $\UMsh(r)$ is represented by a linear combination of $(1,r)$-graphs obtained by grafting these.
The same proof used in \cite[p.5]{medina2020prop1} to show that $\UM$ is an $E_\infty$-operad can be used to prove the same for $\UMsh$.

\subsection{$E_\infty$-coalgebra preservation} \label{ss:e-infty preservation}

We devote this subsection to the proof of the following key result.

\begin{theorem} \label{t:main local}
	The chain map $\CS \colon \chains(\cube^n) \to \chains(\simplex^n)$ is a quasi-isomorphism of $\UMsh$-coalgebras.
\end{theorem}

We start by stating an alternative description of the $\CS$ map.

\begin{lemma} \label{l:cs explicit}
	Let $x = x_1 \ot \cdots \ot x_n \in \chains(\cube^n)_m$ be a basis element with $x_{q_i} = [0,1]$ for all $\{q_1 < \dots < q_m\}$.
	If there is $x_\ell = [0]$ with $\ell < q_m$ then $\CS(x) = 0$, otherwise
	\[
	\CS(x) = \big[ q_1-1, \ \dots \, , \ q_m-1, \ p(x)-1 \big]
	\]
	where $p(x) = \min \set[\big]{\ell \mid x_\ell = [0]}$ or $p(x) = n+1$ if this set is empty.
\end{lemma}

\begin{proof}
	This can be directly verified using the cell structure of $\gsimplex^n$ described in \cref{e:cell structure of gsimplex}.
%	the fact that the image of $\cs \circ \, \delta^0_\ell \colon \gcube^{n-1} \to \gsimplex^n$ for $\ell \in \set{1,\dots,n-1}$ is in a lower dimensional skeleton of $\gsimplex^n$, and the identities
%	\[
%	\cs \circ \, \delta_\ell^1 = \delta_{i-1} \circ \cs,
%	\qquad
%	\cs \circ \, \delta_n^0 = \delta_n \circ \cs,
%	\]
%	for $\ell \in \set{1, \dots, n}$.
\end{proof}

\begin{lemma} \label{l:cs coalgebra map}
	The chain map $\CS \colon \chains(\cube^n) \to \chains(\simplex^n)$ is a quasi-isomorphism of coalgebras.
\end{lemma}

\begin{proof}
	The chain map $\CS$ is a quasi-isomorphism compatible with the counit since it is induced from a cellular map between contractible spaces.
	We need to show it preserves coproducts.
	By naturality it suffices to verify this on $[0,1]^{\ot n}$.
	Recall from \cref{l:coproduct description} that
	\[
	\Delta \big( [0,1]^{\ot n} \big) =
	\sum_{\lambda \in \Lambda} (-1)^{\ind \lambda} \
	\Big(x_1^{(\lambda)} \ot \cdots \ot x_n^{(\lambda)}\Big) \ot
	\Big(y_1^{(\lambda)} \ot \cdots \ot y_n^{(\lambda)}\Big),
	\]
	where the sum is over all choices for each $i \in \{1,\dots,n\}$ of
	\begin{align*}
		x_i^{(\lambda)} &= [0,1],&&\text{or} & x_i^{(\lambda)} &= [0], \\
		y_i^{(\lambda)} &= [1],  && & y_i^{(\lambda)} &= [0,1].
	\end{align*}
	By \cref{l:cs explicit}, the summands above not sent to $0$ by $\CS \ot \CS$ are those basis elements for which $x_i^{(\lambda)} = [0]$ implies $x_j^{(\lambda)} = [0]$ for all $i < j$.
	For any one such summand, its sign is positive and its image by $\CS \ot \CS$ is $[0, \dots, k] \ot [k, \dots, n]$ where $k+1 = \min\set[\big]{i \mid x_i^{(\lambda)} = [0]}$ or $k = n$ if this set is empty.
	The summands $[0, \dots, k] \ot [k,\dots,n]$ are precisely those appearing when applying the Alexander--Whitney coproduct to $\CS \big([0,1]^{\ot n}\big) = [0,\dots,n]$.
	This concludes the proof.
\end{proof}

We will consider the basis of $\chains(\cube^n)$ as a poset with
\[
(x_1 \ot \cdots \ot x_n) \leq (y_1 \ot \cdots \ot y_n)
\]
if and only if $x_\ell \leq y_\ell$ for each $\ell \in \{1, \dots, n\}$ with respect to
\[
[0] < [0,1] < [1].
\]
As we prove next, an example of ordered elements are the tensor factors of each summand in the iterated Serre diagonal.

\begin{lemma} \label{l:order iterated coproduct}
	Writing
	\[
	\Delta^{k-1} \big([0,1]^{\ot n}\big) =
	\sum \pm \ x^{(1)} \ot \cdots \ot x^{(k)}
	\]
	with each $x^{(\ell)}$ a basis element of $\chains(\cube^n)$, we have
	\[
	x^{(1)} \leq \cdots \leq x^{(k)}
	\]
	for every summand.
\end{lemma}

\begin{proof}
%	For $k = 2$, recall from \cref{l:coproduct description} that
%	\[
%	\Delta \big( [0,1]^{\ot n} \big) =
%	\sum_{\lambda \in \Lambda} (-1)^{\ind \lambda} \
%	\Big(x_1^{(\lambda)} \ot \cdots \ot x_n^{(\lambda)}\Big) \ot
%	\Big(y_1^{(\lambda)} \ot \cdots \ot y_n^{(\lambda)}\Big),
%	\]
%	where the sum is over all choices for each $i \in \{1,\dots,n\}$ of
%	\begin{align*}
%		x_i^{(\lambda)} &= [0,1],&&\text{or} & x_i^{(\lambda)} &= [0], \\
%		y_i^{(\lambda)} &= [1],  && & y_i^{(\lambda)} &= [0,1].
%	\end{align*}
%
%
%	and any $\ell \in \{1, \dots, n\}$ we have that
%	\begin{align*}
%		x^{(1)}_\ell = [0]  & \iff x^{(2)}_\ell = [0,1], \\
%		x^{(1)}_\ell = [0,1] & \iff x^{(2)}_\ell = [1],
%	\end{align*}
%	and that neither $x^{(1)}_\ell = [1]$ or $x^{(2)}_\ell = [0]$ can occur, hence $x^{(1)} \leq x^{(2)}$.
%	The claim for $k > 2$ follows from a straightforward induction argument.
	This can be proven using a straightforward induction argument whose base case follows from inspecting \cref{l:coproduct description}.
\end{proof}

\begin{lemma}
	Let $x$, $y$ and $z$ be basis elements of $\chains(\cube^n)$.
	If both $x \leq z$ and $y \leq z$ then either $(x \ast y) = 0$ or every summand in $(x \ast y)$ is $\leq z$.
\end{lemma}

\begin{proof}
	Recall that
	\begin{multline*}
		(x_1 \ot \cdots \ot x_n) \ast (y_1 \ot \cdots \ot y_n)
		=
		(-1)^{|x|} \sum_{\ell=1}^n x_{<\ell}\, \epsilon(y_{<\ell}) \ot x_\ell \ast y_\ell \ot \epsilon(x_{>\ell}) \, y_{>\ell}.
	\end{multline*}
	By assumption $x_{<\ell} \leq z_{<\ell}$ and $y_{>\ell} \leq z_{>\ell}$ for every $\ell \in \{1, \dots, n\}$.
	If $x_\ell \ast y_\ell \neq 0$ then $x_\ell \ast y_\ell = [0,1]$ and either $x_\ell = [1]$ or $y_\ell = [1]$ which implies $z_\ell = [1]$ as well, so $x_\ell \ast y_\ell \leq z_\ell$.
\end{proof}

\begin{lemma}
	If $x$ and $y$ are basis elements of $\chains(\cube^n)$ satisfying $x \leq y$ then
	\begin{equation} \label{e:cs collapse as algebra map}
		\CS(x \ast y) = \CS(x) \ast \CS(y).
	\end{equation}
\end{lemma}

\begin{proof}
	We present this proof in the form of three claims.
	We use \cref{l:cs explicit}, the assumption $x \leq y$, and the fact that the join of basis elements in $\chains(\simplex^n)$ sharing a vertex is $0$ without explicit mention.

	\medskip\noindent \textit{Claim 1}.
	If $\CS(x) = 0$ or $\CS(y) = 0$ then for every $i \in \{1, \dots, n\}$
	\begin{equation} \label{e:zero for join}
		\CS \big( x_{<i}\, \epsilon(y_{<i}) \ot x_i \ast y_i \ot \epsilon(x_{>i}) \, y_{>i} \big) = 0.
	\end{equation}
	Assume $\CS(x) = 0$, that is, there exists a pair $p < q$ such that $x_p = [0]$ and $x_q = [0,1]$, then \eqref{e:zero for join} holds since:
	\begin{enumerate}
		\item If $i > q$, then $x_p$ and $x_q$ are part of $x_{<i}$.
		\item If $i = q$, then $x_q \ast y_q = 0$ for any $y_q$.
		\item If $i < q$, then $\epsilon(x_{>i}) = 0$.
	\end{enumerate}
	Similarly, if there is a pair $p < q$ such that $y_p = [0]$ and $y_q = [0,1]$, then \eqref{e:zero for join} holds since:
	\begin{enumerate}
		\item If $i < p$, then $y_p$ and $y_q$ are part of $y_{>i}$.
		\item If $i = p$, then $x_i = [0]$ and $x_i \ast y_i = 0$.
		\item If $i > p$, then either $x_i \ast y_i = 0$ or $x_i \ast y_i = [0,1]$ and $x_p = [0]$.
	\end{enumerate}
	This proves the first claim and identity \eqref{e:cs collapse as algebra map} under its hypothesis.

	\medskip\noindent \textit{Claim 2}.
	If $\CS(x) \neq 0$ and $\CS(y) \neq 0$ then
	\[
	\CS(x \ast y) =
	\CS \big( x_{<p_x} \epsilon(y_{<p_x}) \ot \, x_{p_x} \! \ast y_{p_x} \ot \epsilon(x_{>p_x}) \, y_{>p_x} \big)
	\]
	if $p_x = \min \big\{ i \mid x_i = [0] \big\}$ is well-defined and $x \ast y = 0$ if not.

	\medskip Assume $p_x$ is not well-defined, i.e., $x_i \neq [0]$ for all $i \in \{1, \dots, n\}$.
	Given that $x \leq y$ we have that $[0] < x_i$ implies $x_i \ast y_i = 0$, and the claim follows in this case.

	Assume $p_x$ is well-defined.
	We will show that for all $i \in \{1,\dots,n\}$ with the possible exception of $i = p_x$ we have
	\begin{equation} \label{e:case main lemma third claim}
		\CS \big( x_{<i} \, \epsilon(y_{<i}) \ot \, x_{i} \! \ast y_{i} \ot \epsilon(x_{>i}) \, y_{>i} \big) = 0
	\end{equation}
	This follows from:
	\begin{enumerate}
		\item If $i < p_x$ and $x_i = [1]$ then $y_i = [1]$ and $x_i \ast y_i = 0$.
		\item If $i < p_x$ and $x_i = [0,1]$ then $x_i \ast y_i = 0$ for any $y_i$.
		\item If $i > p_x$ then \cref{l:cs explicit} implies the claim since $x_{p_x} = [0]$ and $x_i \ast y_i \neq 0$ iff $x_i \ast y_i = [0,1]$.
	\end{enumerate}

	\noindent \textit{Claim 3}.
	If $\CS(x) \neq 0$ and $\CS(y) \neq 0$ then \eqref{e:cs collapse as algebra map} holds.

	\medskip Let us assume that $\big\{ i \mid x_i = [0] \big\}$ is empty, which implies the analogous statement for $y$ since $x \leq y$.
	Since neither of $x$ nor $y$ have a factor $[0]$ in them, \cref{l:cs explicit} implies that the vertex $[n]$ is in both $\CS(x)$ and $\CS(y)$, which implies $\CS(x) \ast \CS(y) = 0$ as claimed.

	Assume now that $p_x = \big\{ i \mid x_i = [0] \big\}$ is well defined, and let $\{q_1 < \dots < q_m\}$ with $x_{q_i} = [0,1]$ for $i \in \{1,\dots,m\}$.
	Since $\CS(x) \neq 0$ \cref{l:cs explicit} implies that $p_x > q_m$, so $\epsilon(x_{>p_x}) = 1$ and Claim 2 implies
	\[
	\CS(x \ast y) =
	\CS \big( x_{<p_x} \epsilon(y_{<p_x}) \ot \, x_{p_x} \! \ast y_{p_x} \ot y_{>p_x} \big).
	\]
	We have the following cases:
	\begin{enumerate}
		\item If $\epsilon(y_{<p_x}) = 0$ then there is $q_i$ such that $y_{q_i} = [0,1]$ so $[q_i-1]$ is in both $\CS(x)$ and $\CS(y)$.
		\item If $\epsilon(y_{p_x}) \neq 0$ and $y_{p_x} \in \{[0], [0,1]\}$ then $x_{p_x} \ast y_{p_x} = 0$ and $[p_x-1]$ is in both $\CS(x)$ and $\CS(y)$.
		\item If $\epsilon(y_{p_x}) \neq 0$ and $y_{p_x} = [1]$ let $\{\ell_1 < \dots < \ell_k\}$ be such that $y_{\ell_j} = [0,1]$ and let $p_y > \ell_k$ be either $n+1$ or $\min\{j \mid y_j = \{0\}\}$ then
		\begin{align*}
			\CS(x \ast y) & =
			\CS \big( x_{< p_x} \ot x_{p_x} \ast y_{p_x} \ot y_{> p_y} \big) \\ & =
			[q_1-1, \dots, q_m-1, p_x-1, \ell_1-1, \dots, \ell_k-1, p_y-1] \\ & =
			\CS(x) \ast \CS(y).
		\end{align*}
	\end{enumerate}
	This concludes the proof.
\end{proof}

Combining the previous two lemmas we obtain the following.

\begin{lemma} \label{l:cs product order}
	Let $x^{(1)} \leq \dots \leq x^{(k)}$ be basis elements of $\chains(\cube^n)$.
	Then,
	\[
	\CS \, \circ \ast^{k-1} \big( x^{(1)} \ot \dotsb \ot x^{(k)} \big)
	=
	\ast^{k-1} \circ \CS^{\ot k} \big( x^{(1)} \ot \dotsb \ot x^{(k)} \big).
	\]
\end{lemma}

We are now ready to present the argument establishing that $\CS$ is an $E_\infty$-coalgebra map.

\begin{proof}[Proof of \cref{t:main local}]
	Since $\UMsh$ is generated by elements represented by shuffle graphs, we only need to show that for any $(k_1,\dots,k_r)$-shuffle $\sigma$ with $k = k_1+\dots+k_r$ the following holds
	\[
	\CS^{\ot r}(\ast^{k_1} \ot \dotsb \ot \ast^{k_r}) \circ \sigma^{-1} \Delta^{k-1} =
	(\ast^{k_1} \ot \dotsb \ot \ast^{k_r}) \circ \sigma^{-1} \Delta^{k-1} \circ \CS.
	\]
	By naturality, it suffices to prove this identity for $[0,1]^{\ot n}$.
	According to \cref{l:order iterated coproduct}
	\[
	x^{(1)} \leq \cdots \leq x^{(k)}
	\]
	for every summand in
	\[
	\Delta^{k-1} \big([0,1]^{\ot n}\big) =
	\sum \pm \ x^{(1)} \ot \cdots \ot x^{(k)}.
	\]
	Since $\sigma$ is a shuffle permutation, \cref{l:cs product order} implies that
	\begin{multline*}
		\CS^{\ot r}(\ast^{k_1} \ot \dotsb \ot \ast^{k_r}) \circ \sigma^{-1} \Delta^{k-1} \big([0,1]^{\ot n}\big) \\ =
		(\ast^{k_1} \ot \dotsb \ot \ast^{k_r}) \circ \sigma^{-1} \CS^{\ot k} \circ \, \Delta^{k-1} \big([0,1]^{\ot n}\big).
	\end{multline*}
	As proven in \cref{l:cs coalgebra map}, $\CS$ is a coalgebra map, which concludes the proof.
\end{proof}

\subsection{Categorical reformulation} \label{ss:e infty preservation}

%In this subsection we will use \cref{t:main local} to deduce from a more general categorical statement that, for any topological space $\fZ$, the chain map $\CS_{\Schains(\fZ)} \colon \sSchains(\fZ) \to \cSchains(\fZ)$ is a quasi-isomorphism of $E_\infty$-coalgebras, specifically, of $\UMsh$-coalgebras.
%We will also construct a zig-zag of quasi-isomorphisms of $E_\infty$-coalgebras between the chains of a cubical set and those of its triangulation.

The assignment $2^n \mapsto \scube{n}$ defines a functor $\cube \to \sSet$ with
\begin{align*}
	&\delta_i^\varepsilon \colon \scube{n} \to \scube{(n+1)} \\
	&\sigma_i \colon \scube{(n+1)} \to \scube{n}
\end{align*}
given by inserting $[\varepsilon, \dots, \varepsilon]$ as the $i^\th$ factor and removing the $i^\th$ factor respectively.
Its Yoneda extension, referred to as triangulation functor, is denoted by
\[
\triangulate \colon \cSet \to \sSet.
\]
This functor admits a right adjoint
\[
\cubify \colon \sSet \to \cSet
\]
defined, as usual, by the expression
\[
\cubify(Y)(2^n) = \sSet \big( \scube{n}, \, Y \big).
\]
We mention that, as proven in \cite[\S~8.4.30]{cisinski2006presheaves}, the pair $(\triangulate,\, \cubify)$ defines a Quillen equivalence when $\sSet$ and $\cSet$ are considered as model categories.

% TO BE EXPLORED LATER
%\subsubsection{Eilenberg--Zilber maps}
%
%We have the following generalization of the map $\EZ \colon \chains(\cube^n) \to \chains(\scube{n})$.
%
%\begin{definition}
%	Let $X$ be a cubical set.
%	The \textit{Eilenberg--Zilber subdivision}
%	\[
%	\EZ_X \colon \cchains(X) \to \schains(\triangulate(X))
%	\]
%	is the chain map induced by the natural cellular map
%	\[
%	\bars{X} \defeq
%	\colim_{\cube^n \downarrow X} \gcube^n \xra{\ez}
%	\colim_{\cube^n \downarrow X} \bars[\big]{\scube{n}} \defeq
%	\bars[\big]{\triangulate(X)}.
%	\]
%\end{definition}
%
%\begin{proposition}
%	For any cubical set $X$ the Eilenberg--Zilber map $\EZ \colon \cchains(X) \to \schains(\triangulate X)$ is a quasi-isomorphism of coalgebras.
%\end{proposition}
%
%\begin{proof}
%	This follows directly from \cref{p:ez is coalgebra map]}.
%\end{proof}

% TO BE EXPLORED LATER
%\begin{proposition}
%	The map $\EZ_{\Schains(\fZ)}$ factors through $\EZ_{\cSing(\fZ)}$.
%	Explicitly,
%	\[
%	\begin{tikzcd}[row sep=0, column sep=small]
%		\cSchains(\fZ) \arrow[r] &
%		\schains \big( \triangulate \cSing(\fZ) \big) \arrow[r] &
%		\sSchains(\fZ) \\
%		(\gcube^n \xra{f} \fZ) \arrow[r, mapsto] &
%		\big( \bars{\scube{n}} \xra{f} \fZ \big) \arrow[r] &
%		\sum\limits_{\mathclap{\sigma \in \sym_n}} \sign(\sigma) \big( \gsimplex^n \xra{\gi_\sigma} \bars{\scube{n}} \xra{f} \fZ \big).
%	\end{tikzcd}
%	\]
%\end{proposition}
%
%\begin{proposition}
%	The map ... coalgebra q-i.
%\end{proposition}

\begin{definition}
	The simplicial map $\ccs \colon \scube{n} \to \simplex^n$ is defined by
	\[
	[\varepsilon_0^1, \dots, \varepsilon_m^1]
	\times \dots \times
	[\varepsilon_0^n, \dots, \varepsilon_m^n]
	\mapsto
	[v_0, \dots, v_m]
	\]
	where $v_i = \varepsilon_i^1 + \varepsilon_i^1 \varepsilon_i^2 + \dots + \varepsilon_i^1 \dotsm \varepsilon_i^n$.
\end{definition}

Please observe that the maps $\cs$ and $\bars{\ccs} \circ \ez$ agree.

\begin{definition}
	Let $Y$ be a simplicial set.
	The map
	\[
	\CS_Y \colon \schains(Y) \to \cchains(\cubify Y)
	\]
	is the linear map induced by sending a simplex $y \in Y_n$ to the composition
	\[
	\scube{n} \xra{\ccs} \simplex^n \xra{\xi_y} Y
	\]
	where $\xi_y \colon \simplex^n \to Y$ is the simplicial map determined by $\xi_y \big( [n] \big) = y$.
\end{definition}

\begin{theorem} \label{t:main comparison}
	For any simplicial set $Y$ the map $\CS_Y \colon \schains(Y) \to \cchains(\cubify Y)$ is a quasi-isomorphism of $\UMsh$-coalgebras which extend respectively the Alexander--Whitney and Serre coalgebra structures.
\end{theorem}

\begin{proof}
	This is a direct consequence of \cref{t:main local} following from a standard category theory argument, which we now present.
	Consider the isomorphism
	\[
	\chains(\cubify Y) \cong \displaystyle\bigoplus_{n\in\N} \chains(\cube^n) \ot \k\set[\Big]{\sSet \big( \scube{n}, \simplex^n \big)} \Big/ \sim \,
	\]
	and the canonical linear inclusions:
	\[
	\begin{tikzcd} [row sep=-7pt, column sep=small]
		\chains(\cube^n) \arrow[r] &
		\displaystyle\bigoplus_{m \in \N} \Hom \big( \chains(\cube^m),\, \chains(\cube^n) \big) \\
		(2^m \xra{\delta} 2^n) \arrow[r, mapsto] &
		\big(\chains(\cube^m) \xra{\chains(\delta)} \chains(\cube^n)\big)
	\end{tikzcd}
	\vspace*{-7pt}
	\]
	and
	\[
	\begin{tikzcd} [row sep=-5pt, column sep=small]
		\displaystyle \bigoplus_{n \in \N}
		\k\set[\Big]{\sSet\big(\scube{n}, \simplex^n\big)} \arrow[r] &
		\displaystyle \bigoplus_{n \in \N}
		\Hom \Big( \chains \big( \scube{n} \big),\, \chains(\simplex^n) \Big) \\
		\big( \scube{n} \xra{f} \simplex^n \big) \arrow[r, mapsto] &
		\Big( \chains \big( \scube{n} \big) \xra{\chains(f)} \chains(\simplex^n) \Big).
	\end{tikzcd}
	\]
	We can use these and the naturality of $\EZ$ to construct the following chain map which is an isomorphism onto its image.
	\[
	\begin{tikzcd}[column sep=small, row sep=0]
		\chains(\cubify Y) \arrow[r] &
		\displaystyle\bigoplus_{n\in\N} \Hom \big( \chains(\cube^n),\, \chains(Y)\, \big) \\
		(\delta \ot f) \arrow[r, mapsto]& \big( \chains(f) \circ \EZ \circ \chains(\delta) \, \big).
	\end{tikzcd}
	\]
	Let $\Gamma$ be an element in $\UMsh(r)$ and denote by $\Gamma^\cube \colon \chains(\cubify Y) \to \chains(\cubify Y)^{\ot r}$ and $\Gamma^\simplex \colon \chains(Y) \to \chains(Y)^{\ot r}$ its image in the respective endomorphism operads.
	Using the naturality of $\Gamma^\square$, we have that
	$\Gamma^\square(\delta \ot \!f)$ corresponds to $(\chains(f) \circ \EZ)^{\ot r} \circ \Gamma^\square \circ \chains(\delta)$.
	On the other hand, the map $\CS_Y$ corresponds to
	\[
	\begin{tikzcd}[column sep=small, row sep=0]
		\chains(Y)_n \arrow[r] & \chains(\cubify Y)_n \\
		y \arrow[r,mapsto]& \big( \chains(\xi_y) \circ \CS \big)
	\end{tikzcd}
	\]
	where $\xi_y \colon \simplex^n \to Y$ is determined by $\xi_y \big( [n] \big) = y$, and we used that $\CS = \chains(\ccs) \circ \EZ$ to ensure the above assignment is well defined.
	The image of $\Gamma^\triangle(y)$ corresponds to $\chains(\xi_y)^{\ot r} \circ \Gamma^\triangle \circ \CS$.
	So the claim follows from the identity
	\begin{align*}
		\Gamma^\cube \big( 2^n \ot (\xi_y \circ \ccs) \big) &=
		(\chains(\xi_y \circ \ccs) \circ \EZ)^{\ot r} \circ \Gamma^\square \\ &=
		\chains(\xi_y)^{\ot r} \circ \CS^{\ot r} \circ \, \Gamma^\square \\ &=
		\chains(\xi_y)^{\ot r} \circ \, \Gamma^\triangle \circ \CS
	\end{align*}
	where we used that $\CS^{\ot r} \circ \, \Gamma^\square = \Gamma^\triangle \circ \CS$ as proven in \cref{t:main local}.
\end{proof}

\begin{corollary} \label{c:zig-zag}
	For any cubical set $X$
	\[
	\cchains(X) \xra{\cchains(\xi_X)}
	\cchains(\cubify{\triangulate X})
	\xla{\CS_{\cT X}} \schains(\triangulate X),
	\]
	where $\xi$ is the unit of adjunction, is a natural zig-zag of quasi-isomorphisms of $\UMsh$-coalgebras which extend respectively the Serre and Alexander--Whitney coalgebra structures.
\end{corollary}

\begin{proof}
	The map $\CS_{\cT X}$ is a quasi-isomorphism of $\UMsh$-coalgebras by \cref{t:main comparison}, whereas $\cchains(\xi_X)$ is also one since it is induced from a cubical map that is a weak-equivalence.
\end{proof}

\begin{corollary} \label{c:cs e infty}
	The singular simplicial and cubical chains of a topological space $\fZ$ are quasi-isomorphic as $\UMsh$-coalgebras which extend respectively the Alexander--Whitney and Serre coalgebra structures.
	More specifically, the map
	\[
	\CS_{\Schains(\fZ)} \colon \sSchains(\fZ) \to \cSchains(\fZ)
	\]
	is a quasi-isomorphism of $\UMsh$-coalgebras.
\end{corollary}

\begin{proof}
	It can be verified using that $\cs = \bars{\ccs} \circ \ez$ that this map factors as
	\[
	\CS_{\Schains(\fZ)} \colon \sSchains(\fZ) \xra{\CS_{\sSing(\fZ)}} \cchains\!\big(\cubify\sSing(\fZ)\big) \xra{\EZ_{\Schains(\fZ)}} \cSchains(\fZ)
	\]
	where the first map was proven in \cref{t:main comparison} to be a quasi-isomorphism of $\UMsh$-coalgebras, and the second, introduced in \cref{ss:ez}, is also one since it is induced from a cubical map whose geometric realization is a homeomorphism.
\end{proof}

% TO BE EXPLORED LATER
%\subsubsection{The other map}
%
%\begin{definition}
%	The simplicial map $\ci \colon \simplex^n \to \scube{n}$ is defined by
%	\[
%	[0, \dots, n] \mapsto \varepsilon^1 \times \dots \times \varepsilon^n
%	\]
%	where $\varepsilon^j = [\varepsilon^j_0, \dots, \varepsilon^j_n]$ with
%	\[
%	\varepsilon^j_i =
%	\begin{cases}
%		0, & i \leq j, \\
%		1, & i > j. \\
%	\end{cases}
%	\]
%\end{definition}
%
%\begin{proposition}
%	Up to cellular isomorphisms the maps $\mathfrak{i}$ and $\bars{\ci}$ agree.
%\end{proposition}
%
%It has a left inverse
%\[
%\rI_{\,\cubify Y} \colon \cchains(\cubify Y) \to \schains(Y)
%\]
%induced by sending a cube $g \in \cubify Y$ to the image of the identity $[n]$ by the composition
%\[
%\simplex^n \xra{\incl} \scube{n} \xra{g} Y.
%\]


	% Bibliography
	\sloppy
	\printbibliography

	% Addresses
	\noindent \textit{Ralph M. Kaufmann} \\
	Purdue University \\
	\href{mailto:rkaufman@purdue.edu}{rkaufman@purdue.edu} \\
	\newline
	\noindent \textit{Anibal~M.~Medina-Mardones} \\
	Max Plank Institute for Mathematics and University of Notre Dame \\
	\href{mailto:ammedmar@mpim-bonn.mpg.de}{ammedmar@mpim-bonn.mpg.de}
\end{document}