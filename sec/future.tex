\section{Future work} \label{s:future}

In the fifties, Adams introduced in \cite{adams1956cobar} a comparison map
\[
\cobar \sSchains(\fZ,z) \to \cSchains(\loops_z \fZ)
\]
from his cobar construction on the simplicial singular chains of a pointed space $(\fZ,z)$ to the cubical singular chains on its based loop space $\loops_z \fZ$.
This comparison map is a quasi-isomorphism of algebras, which was shown by Baues \cite{baues1998hopf} to be one of bialgebras by considering Serre's cubical coproduct.
In \cite{medina2021cobar} the $E_\infty$-coalgebra structure defined here is used to generalize Baues' result, by showing that Adams' comparison map is a quasi-isomorphism of $E_\infty$-bialgebras or, more precisely, of monoids in the category of $\UM$-coalgebras.

For a closed smooth manifold $M$, in \cite{medina2021flowing} a canonical vector field was used to compare multiplicatively two models of ordinary cohomology.
On one hand, a cochain complex generated by manifolds with corners over $M$, with partially defined intersection; on the other, the cubical cochains of a cubulation of $M$ with the Serre product.
With the explicit description introduced here of an $E_\infty$-structure on cubical cochains, we expect to build on this multiplicative comparison and enhance geometric cochains \cite{medina2022foundations} with compatible representations of further derived structure.