
\section{The \texorpdfstring{$E_\infty$}{E-infty}-prop \texorpdfstring{$\M$}{M}} \label{s:operads and props}

We now review the definition of the finitely presented $E_\infty$-prop $\M$ introduced in \cite{medina2020prop1} which, given its small number of generators and relations, is well suited to define $E_\infty$-structures.
In the next section we use this model to define a natural $E_\infty$-structure on cubical chains and cochains.
We start by reviewing the basic material in the theory of operads and props, referring the reader to, for example, \cite{markl2008props} for a more complete treatment.

\subsection{Symmetric modules and bimodules}

Let $\S$ be the category whose objects are the natural numbers and whose set of morphisms between $m$ and $n$ is empty if $m \neq n$ and is otherwise the symmetric group $\S_n$.
A \textit{left $\S$-module} (resp. \textit{right} $\S$-\textit{module} or $\S$-\textit{bimodule}) is a functor from $\S$ (resp. $\S^\op$ or $\S \times \S^\op$) to $\Ch$.
In this paper we prioritize left module structures over their right counterparts.
As usual, taking inverses makes both perspectives equivalent.
We respectively denote by $\smod$ and $\sbimod$ the categories of left $\S$-modules and of $\S$-bimodules with morphisms given by natural transformations.

The group homomorphisms $\S_n \to \S_n \times \S_1$ induce a forgetful functor $\forget \colon \sbimod \to \smod$.
Explicitly, $\forget(\P)(r) = \P(1, r)$ for $r \in \N$.
The similarly defined forgetful functor to right $\S$-modules will not be used.

\subsection{Composition structures}

We can define \textit{operads} and \textit{props} by enriching $\S$-modules and $\S$-bimodules with certain composition structures.
For a complete presentation of these concepts we refer to Definition~11 and 54 of \cite{markl2008props}.
Intuitively, using examples defined in the next subsection, operads and props can be understood by abstracting the composition structure naturally present in the left $\S$-module $\End^C$ (or right $\S$-module $\End_C$), naturally an operad, and the $\S$-bimodule $\End^C_C$, naturally a prop.
We remark that the prop structure on $\P$ restricts to an operad structure on $\forget(\P)$.

\subsection{Representations}

Given a chain complex $C$ define $\End^C$, $\End_C$ and $\End_C^C$ by
\begin{align*}
\End^C(r) &= \Hom(C, C^{\otimes r}),
& \End_C(r) &= \Hom(C^{\otimes r}, C),
& \End^C_C(r, s) &= \Hom(C^{\otimes r}, C^{\otimes s}),
\end{align*}
with their natural operad and prop structures respectively.
We remark that the forgetful functor $U$ sends $\End^C_C$ to $\End^C$.

Let $C$ be a chain complex, $\O$ an operad, and $\P$ a prop.
An $\O$-\textit{coalgebra} (resp. $\O$-\textit{algebra} or $\P$-\textit{bialgebra}) structure on $C$ is a structure preserving morphism $\O \to \End^C$ (resp. $\O \to \End_C$ or $\P \to \End_C^C$).

\subsection{\texorpdfstring{$E_\infty$}{E-infty}-operads and -props}

Recall that a \textit{free $\S_r$-resolution} of a chain complex $C$ is a quasi-isomorphism $R \to C$ from a chain complex of free $\k[\S_r]$-modules.

An $\S$-module $M$ is said to be $E_{\infty}$ if there exists a morphism of $\S$-modules $M \to \underline{\k}$ inducing for each $r \in \N$ a free $\S_r$-resolution $M(r) \to \k$.
For example, we can obtain one such $\S$-module by using the functor of singular chains and the set, parameterized by $r \in \N$, of maps to the terminal space from models of the universal bundle $\mathrm{E} \S_r$.

An operad is said to be $E_{\infty}$ if its underlying $\S$-module is $E_\infty$, and, following Boardman-Vogt \cite{boardman1973homotopy}, a prop $\P$ is said to be an $E_\infty$-prop if $U(\P)$ is an $E_\infty$-operad.

\subsection{Free prop construction} \label{ss:free prop}

\begin{figure}
	\input{aux/immersion}
	\caption{Immersed graphs represent labeled directed graphs with the direction implicitly given from top to bottom and the labeling from left to right.}
	\label{f:immersion}
\end{figure}

The \textit{free prop} $\free(M)$ generated by an $\S$-bimodule $M$ is constructed using directed graphs with no directed loops that are enriched with a labeling described next.
We think of each directed edge as built from two compatibly directed half-edges.
For each vertex $v$ of a directed graph $G$, we have the sets $in(v)$ and $out(v)$ of half-edges that are respectively incoming to and outgoing from $v$.
Half-edges that do not belong to $in(v)$ or $out(v)$ for any $v$ are divided into the disjoint sets $in(G)$ and $out(G)$ of incoming and outgoing external half-edges.
For any positive integer $n$ let $\overline{n} = \{1, \dots, n\}$ and set $\overline{0} = \emptyset$.
For any finite set $S$, denote the cardinality of $S$ by $|S|$.
The labeling is given by bijections
\[
\overline{|in(G)|}\to in(G), \qquad
\overline{|out(G)|}\to out(G),
\]
and
\[
\overline{|in(v)|}\to in(v), \qquad
\overline{|out(v)|}\to out(v),
\]
for every vertex $v$.
We refer to the isomorphism classes of such labeled directed graphs with no directed loops as $(n,m)$\textit{-graphs} denoting the set of these by $\G(m,n)$.
We use graphs immersed in the plane to represent elements in $\G(m,n)$, please see \cref{f:immersion}.
We consider the right action of $\S_n$ and the left action of $\S_m$ on a $(n,m)$-graph given respectively by permuting the labels of $in(G)$ and $out(G)$.
This action defines the $\S$-bimodule structure on the free prop
\begin{equation} \label{e:free prop}
\free(M)(m,n) \ = \bigoplus_{\Gamma \in \G(m,n)} \bigotimes_{v \in Vert(\Gamma)} out(v) \otimes_{\S_q} M(p, q) \otimes_{\S_p} in(v),
\end{equation}
where we simplified the notation writing $p$ and $q$ for $\overline{|in(v)|}$ and $\overline{|out(v)|}$ respectively.
The composition structure is defined by (relabeled) grafting and disjoint union.

\subsection{The prop $\M$}

We now recall the model of $E_\infty$ that is central to our constructions.

\begin{definition}
	Let $\M$ be the prop generated by
	\begin{equation} \label{e:generators of M}
	\counit\,, \hspace*{.6cm} \coproduct\,, \hspace*{.6cm} \product,
	\end{equation}
	in degrees $0$, $0$ and $1$ respectively, and boundaries
	\begin{equation} \label{e:boundary of M}
	\partial\ \counit = 0,
	\hspace*{.6cm}
	\partial\, \coproduct = 0,
	\hspace*{.6cm}
	\partial \product = \ \boundary\,,
	\end{equation}
	modulo the prop ideal generated by
	\begin{equation} \label{e:relations of M}
	\leftcounitality\,, \hspace*{.6cm} \rightcounitality\,, \hspace*{.6cm} \productcounit.
	\end{equation}
\end{definition}

Explicitly, any element in $\M(m,n)$ can be written as a linear combination of the $(m,n)$-graphs generated by those in \eqref{e:generators of M} via grafting, disjoint union and relabeling, modulo the prop ideal generated by the relations in \eqref{e:relations of M}, and its boundary is determined, using \eqref{e:free prop}, by \eqref{e:boundary of M}.

The same proof given in \cite[Theorem 3.3]{medina2020prop1} establishes the following.

\begin{proposition}
	The prop $\M$ is $E_{\infty}$.
\end{proposition}

We remark that, as proven in \cite{medina2018prop2}, this prop is obtained from applying the functor of cellular chains to a finitely presented prop over the category of CW-complexes.
