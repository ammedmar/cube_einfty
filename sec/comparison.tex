
\section{The Cartan-Serre comparison map} \label{s:the cartan-serre comparison map}

More specifically, in \cite[p. 442]{serre1951homologie}, Serre described for any topological space a quasi-isomorphism between its simplicial and cubical singular cochains using a canonical cellular map $\gcube^n \to \gsimplex^n$ also considered in \cite[p.199]{eilenberg1953acyclic}, where it is attributed to Cartan.

Let $Z$ be a topological space.
The goal of this section is to prove a generalization of the comultiplicativity of the natural quasi-isomorphisms $\chains(\sSing Z) \to \chains(\cSing Z)$ studied by Cartan and Serre.
Specifically, we show that this map is a morphism of $E_\infty$-coalgebras in \cref{ss:extended comultiplicativity}.
More generally, we construct an $E_\infty$-coalgebra map $\chains(Y) \to \chains(\cubify Y)$ for any simplicial set $Y$, where $\cubify$ is the right adjoint to the triangulation functor from cubical sets to simplicial sets.

We review from \cite{medina2021cubical} the relationship between the $E_\infty$-structures defined on simplicial chains in \cref{ss:e-infty on simplicial} and on cubical chains in \cref{ss:e-infty on cubical}.

%In \cite[p.442]{serre1951homologie}, the \textit{Serre--Cartan collapse} $\gcube^n \to \gsimplex^n$ was introduced and used to define a natural quasi-isomorphism of coalgebras $\sS(Z) \to \cS(Z)$ for any topological space $Z$.
This map factors as the composition
\[
\schains(\sSing(Z)) \xra{\CS_{\sSing(Z)}}
\chains(\cubify \sSing(Z)) \to
\chains(\cSing(Z)),
\]
where the first map $\CS_X$, referred to as the \textit{Serre--Cartan comparison map}, is a quasi-isomorphism defined for any simplicial set $X$, and the second map is induced from a morphism of cubical sets.

Although both $\schains(X)$ and $\chains(\cubify X)$ have natural $\UM$-structures, the map $\CS_X$ is not a morphism of $\UM$-coalgebras for a generic simplicial set $X$.
Nevertheless, after restriction of their $\UM$-structures via an inclusion of $E_\infty$-operads $\USL \to \UM$, the Cartan-Serre comparison map becomes a morphism of $E_\infty$-coalgebras.

\newpage

\subsection{Simplicial chains}

We denote the \textit{simplex category} by $\simplex$, the category of \textit{simplicial sets} by $\sSet = \Fun(\simplex^\op, \Set)$ and the standard $n$-simplex by $\simplex^n$.
As usual, we denote an element in $\simplex^n_m$ by a non-decreasing tuples $[v_0, \dots, v_m]$ with $v_i \in \{0, \dots, n\}$.
The \textit{Cartesian product} of simplicial sets is defined object-wise.
For example,
\[
\big(\simplex^n \times \simplex^{n^\prime}\big)_m = \simplex^n_m \times \simplex^{n^\prime}_m
\]
consists of pairs of non-decreasing tuples $[v_0, \dots, v_m] \times [w_0, \dots, w_m]$ of appropriate integers.

The \textit{simplicial singular complex} functor is denoted by $\sSing \colon \Top \to \sSet$ and the functor of \textit{(normalized) chains} by $\schains \colon \sSet \to \Ch$.
We omit the superscript $\simplex$ from either of these if no confusion may result from doing so.

The \textit{Alexander--Whitney coalgebra} functor is defined by a Kan extension argument based on the following natural maps.
For any $n \in \N$, define $\epsilon \colon \chains(\simplex^n) \to \k$ by
\[
\epsilon \big( [v_0, \dots, v_q] \big) = \begin{cases} 1 & \text{ if } q = 0, \\ 0 & \text{ if } q>0, \end{cases}
\]
and $\Delta \colon \chains(\simplex^n) \to \chains(\simplex^n)^{\otimes2}$ by
\[
\Delta \big( [v_0, \dots, v_q] \big) = \sum_{i=0}^q [v_0, \dots, v_i] \otimes [v_i, \dots, v_q].
\]

\subsection{$E_\infty$-structure on simplicial chains} \label{ss:e infinity structures}

In \cite{medina2020prop1}, a similar construction to the one introduced in \cref{s:action} provides the chains of simplicial sets with a natural $\UM$-coalgebra structure.
It is also induced from a natural $\M$-bialgebra structure on the chains of representables presheaves, i.e., of standard simplices in this case.
This $\M$-bialgebra structure on $\chains(\simplex^{\!n})$ is defined by the assignment
\[
\counit \mapsto \epsilon, \quad \coproduct \mapsto \Delta, \quad \product \mapsto \ast,
\]
where $\epsilon$ and $\Delta$ define the Alexander--Whitney coalgebra structure on simplicial chains, and
\[
\ast \colon \chains(\simplex^{\!n})^{\otimes 2} \to \chains(\simplex^{\!n})
\]
is an algebraic version of the \textit{join} defined by
\[
\left[v_0, \dots, v_p \right] \ast \left[v_{p+1}, \dots, v_q\right] = \begin{cases} (-1)^{p+|\pi|} \left[v_{\pi(0)}, \dots, v_{\pi(q)}\right] & \text{ if } v_i \neq v_j \text{ for } i \neq j, \\
0 & \text{ if not}, \end{cases}
\]
where $\pi$ is the permutation that orders the totally ordered set of vertices and $(-1)^{|\pi|}$ is its sign.

\subsection{The simplicial cube} \label{ss:simplicial cube}

The \textit{simplicial $n$-cube} is the $n^\th$-fold Cartesian product $\scube{n}$.
We introduce the following notation for its elements.
For $n = 1$ and $m \in \N$ we write
\[
\langle k \rangle \defeq [\, \overbrace{0, \dots, 0}^{m+1-k}, \overbrace{1, \dots, 1}^{k}\,]
\]
for $k \in \{0, \dots, m+1\}$, and
\[
\scube{n}_m = \big\{ \angles{k_1} \times \dots \times \angles{k_n} \mid \angles{k_i} \in \simplex^1_m \big\}.
\]
Additionally, we define for each $n \in \N$ natural simplicial maps
\[
\inclusion \colon \simplex^n \leftrightarrows \scube{n} :\! \projection
\]
satisfying $\projection \circ \inclusion = \id_{\simplex^{\!n}}$ by
\[
\inclusion \colon [0, \dots, n] \mapsto \angles{n} \times \dots \times \angles{1},
\]
and
\[
\projection \colon [\varepsilon_0^1, \dots, \varepsilon_m^1] \times \dots \times [ \varepsilon_0^n, \dots, \varepsilon_m^n] \mapsto
[\eta_0^1, \dots, \eta_m^1] \times \dots \times [ \eta_0^n, \dots, \eta_m^n]
\]
with $\eta_i^k = \varepsilon_i^1 \dots \varepsilon_i^k$.

\subsection{Triangulation and its right adjoint}

The assignment $2^n \mapsto \scube{n}$ defines a functor $\cube \to \sSet$ with $\delta_i^\varepsilon \colon \scube{n} \to \scube{(n+1)}$ inserting $[\varepsilon, \dots, \varepsilon]$ as $i^\th$ factor and $\sigma_i \colon \scube{(n+1)} \to \scube{n}$ removing the $i^\th$ factor.
Its left Kan extension
\[
\triangulate \colon \cSet \to \sSet
\]
is referred to as the \textit{triangulation} functor.
It admits a right adjoint
\[
\cubify \colon \sSet \to \cSet
\]
defined by
\[
\cubify(X)(2^m) = \sSet \big( \scube{n}, \, X \big).
\]

As proven in \cite[8.4.30]{cisinski2006presheaves}, the pair $(\triangulate,\, \cubify)$ define a Quillen equivalence when simplicial and cubical sets are considered as model categories.

\subsection{The Cartan-Serre comparison chain map} \label{ss:comparison map}

The projection
\[
\projection \colon \scube{n} \to \simplex^{\!n}
\]
defined in \cref{ss:simplicial cube} induces for any simplicial set $Y$ a natural morphism of graded sets
\[
Y \to \cubify Y
\]
defined on a standard simplex $\simplex^n$ by
\[
\big( [m] \xra{\sigma} [n] \big) \mapsto
\big( \scube{m} \xra{\pi} \simplex^m \xra{\sigma_\ast} \simplex^n \big).
\]
It induces a chain map
\[
\CS_Y \colon \schains(Y) \to \cchains(\cubify Y)
\]
which we refer to as the \textit{Cartan-Serre comparison chain map}, remarking that it is a quasi-isomorphism by an acyclic carrier argument \cite{eilenberg1953acyclic}.

\subsection{The Cartan-Serre chain map} \label{ss:the cartan-serre chain map}

Let $\psi \colon \gchains(\gcube^n) \to \gchains(\gsimplex^{\!n})$ be the chain map induced by the Cartan-Serre map $\gcube^n \to \gsimplex^{\!n}$.
Using the identification of the cellular chains of $\gcube^n$ and $\gsimplex^{\!n}$ with $\chains(\cube^n)$ and $\chains(\simplex^{\!n})$
provides the domain and target of $\psi$ with an $\Sl$-coalgebra structure.
We have the following key result whose proof will be given in  \cref{ss:comparison proof}.

\begin{lemma} \label{l:cartan-serre is e infinity}
	The map $\psi$ is a morphism of $\Sl$-coalgebras.
\end{lemma}

It is easy to verify that $\psi$ under this identification is equal to the composition
\[
\begin{tikzcd}[column sep=normal]
\chains(\cube^n) \arrow[r, "\chains(\triangulate)"] &[-3pt]
\chains \big( \scube{n} \big) \arrow[r, "\chains(s)"] &[-3pt]
\chains(\simplex^{\!n}),
\end{tikzcd}
\]
which will be use to establish the extended comultiplicativity of the comparison map we introduce next.

\subsection{Proof of \cref{t:extended comultiplicativity}} \label{ss:proof extended comultiplicativity}

For any topological space $Z$, we need to show that the chain map $\chains(\sSing Z) \to \chains(\cSing Z)$ induced by precomposing each basis element with the Cartan-Serre map $\gcube^n \to \gsimplex^{\!n}$ is an $\Sl$ morphism.
This map factors as a composition
\[
\chains(\sSing Z) \to \chains(\cubify \sSing Z) \to \chains(\cSing Z)
\]
where the first map is the comparison map of \cref{ss:comparison map} and the second is induced from a morphism of cubical set.

The first is a morphism of $\Sl$ coalgebras by \cref{t:comparison map is e infinity}, and the second by naturality.

\subsection{The operad of surjection-like graphs} \label{ss:surjection-like graphs}

The operad $\USL$ is generated as a suboperad of $\UM$ by all immerse connected graphs of the form
\begin{equation*}
\begin{tikzpicture}[scale=1]
\draw (0,0)--(0,-.6) node[below, scale=.75]{$1$};
\draw (0,0)--(.5,.5);
\draw (-.3, .3)-- (-.2,.5) node[scale=.75] at (-.2,.7) {\qquad $1\, \ \ 2\ \ ...\ \ k_1$};
\draw (-.5,.5)--(0,0);
\node[scale=.75] at (.11,.4){$..$.};

\node[scale=.75] at (1,0){$\cdots$};
\node[scale=.75] at (1,-.9){$\cdots$};

\draw (2,0)--(2,-.68) node[scale=.75, below]{$r$};
\draw (2,0)--(2.5,.5);
\draw (1.7, .3)--(1.8,.5) node[scale=.75] at (1.78,.7) {\qquad $1\, \ \ 2\ \ ...\ \ k_r$};
\draw (1.5,.5)--(2,0);
\node[scale=.75] at (2.11,.4){$..$.};

\draw (1,2.5)--(1,3) node[scale=.75, above]{$1$};
\draw (1,2.5)--(0,2) node[scale=.75, below]{$1$};
\draw (.25,2.125)--(.5,2) node[scale=.75, below]{$2$};
\draw (.5,2.25)--(1,2) node[scale=.75, below]{$3$};
\draw (1,2.5)--(2,2) node[scale=.75, below]{\ \quad $n + r$};
\node[scale=.75] at (1.5,1.75){$\cdots$};

\node[scale=.75] at (1,1.3) {$\vdots$};

\node at (2.85,0){};
\end{tikzpicture}
\end{equation*}
where there are no hidden vertices and the strands are joined so that the associated maps $\{1, \dots, k_j\} \to \{1, \dots, n+k\}$ are order-preserving.
From \cite{medina2021cubical} we have the following.

\begin{proposition} \label{p:simplicialandcubical}
	The suboperad $\USL$ of $\UM$ is $E_\infty$.
\end{proposition}

\subsection{The Cartan-Serre collapse map} \label{ss:the cartan-serre collapse map}

Let us consider, with their usual CW structures, the topological simplex
\[
\gsimplex^{\!n} = \{(y_0, \dots, y_n) \mid y_i \in [0,1], \ \textstyle{\sum} \, y_i = 1\}
\]
and the topological cube $\gcube^n$ recalled in \eqref{e:topological cube}.

The \textit{Cartan-Serre collapse map} $\gcube^n \to \gsimplex^{\!n}$ is the cellular map defined by
\begin{equation} \label{e:cartan-serre CW map}
\begin{split}
&y_0 = 1 - x_1, \\
&y_1 = x_1(1 - x_2), \\
&\ \vdots \\
&y_{n-1} = x_1 x_2 \dots x_{n-1}(1-x_n), \\
&y_{n} = x_1 x_2 \dots x_n.
\end{split}
\end{equation}

For later use, notice that coordinates in the domain that are $0$ play an important role.
Specifically, if $x_i = 0$ then $y_{j} = 0$ for every $j \geq i$.


\begin{theorem} \label{t:comparison map is e infinity}
	The comparison map for any simplicial set $Y$ is a morphism of $\Sl$ coalgebras.
\end{theorem}

\begin{proof}
	This follows from \cref{l:cartan-serre is e infinity} and the identification of $\psi$ with $\chains(s \circ \mathcal T)$
\end{proof}

\subsection{Extended comultiplicativity} \label{ss:extended comultiplicativity}

Let $Y$ be a simplicial set and $X$ a cubical set.
By restriction of their $\UM$ coalgebra structures, both $\chains(Y)$ and $\chains(X)$ are $\Sl$ coalgebras, and we have the following generalization of the Serre's comultiplicativity.

\begin{theorem} \label{t:extended comultiplicativity}
	For any topological space $Z$, the quasi-isomorphism $\chains(\sSing Z) \to \chains(\cSing Z)$ induced by precomposing with the Serre-Cartan map $\gcube^n \to \gsimplex^{\!n}$ is a morphisms of $E_\infty$-coalgebras, where the domain structure extends the Alexander--Whitney coproduct and the target the Serre coproduct.
\end{theorem}

%\subsection{The Cartan-Serre comparison chain map} \label{ss:comparison map}
%
%For any simplicial set $Y$ the map $\projection \colon \scube{n} \to \simplex^{\!n}$ induces a map
%\[
%\begin{tikzcd}[column sep = small, row sep=0pt]
%\sSet(\simplex^{\!n}, Y) \arrow[r] &
%\sSet(\triangulate \cube^n, Y) \\
%\big( \simplex^{\!n} \xra{g} Y \big) \arrow[r, maps to] &
%\big(\scube{n} \xra{\projection} \simplex^{\!n} \xra{g} Y \big),
%\end{tikzcd}
%\]
%for every $n \geq 0$, which, composing with the adjunction isomorphism, defines a morphism
%\[
%\begin{tikzcd}[column sep = small, row sep=0pt]
%\sSet(\simplex^{\!n}, Y) \arrow[r] &
%\cSet(\cube^n,\, \cubify Y) \\
%\big( \simplex^{\!n} \xra{g} Y \big) \arrow[r, maps to] &
%\Big(\cube^n \to \big(\scube{n} \xra{g \circ \projection} Y \big)\Big)
%\end{tikzcd}
%\]
%of graded sets inducing a chain map $\CS_Y \colon \chains(Y) \to \chains(\cubify Y)$.
%We refer to $\CS_Y$ as the \textit{Cartan-Serre comparison chain map} for $Y$ and remark that it is a quasi-isomorphism by an acyclic carrier argument \cite{eilenberg1953acyclic}.
