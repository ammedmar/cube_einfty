
\section{The Cartan-Serre map} \label{s:the cartan-serre map}

Let $Z$ be a topological space.
The goal of this section is to prove a generalization of the comultiplicativity of the natural quasi-isomorphisms $\chains(\sSing Z) \to \chains(\cSing Z)$ defined by precomposing with the chain map induced by the Cartan-Serre map defined in \cref{ss:the cartan-serre map}.
Specifically, we show that this map is a morphism of $E_\infty$ coalgebras in \cref{ss:extended comultiplicativity}.
More generally, we construct an $E_\infty$ coalgebra map $\chains(Y) \to \chains(\U Y)$ for any simplicial set $Y$, where $\U$ is the right adjoint to the triangulation functor from cubical sets to simplicial sets.

\subsection{The Cartan-Serre map} \label{ss:the cartan-serre map}

Let us consider the topological simplex
\begin{equation*}
\gsimplex^{\!n} = \{(y_0, \dots, y_n) \mid y_i \in [0,1], \ \textstyle{\sum} \, y_i = 1\}
\end{equation*}
and the topological cube
\begin{equation*}
\gcube^{n} = \{(x_1, \dots, x_n) \mid x_i \in [0,1]\}
\end{equation*}
with their usual CW structures.
The \textit{Cartan-Serre} \textit{map} $\gcube^n \to \gsimplex^{\!n}$ is defined by
\begin{equation} \label{e:cartan-serre CW map}
\begin{split}
&y_0 = 1 - x_1, \\
&y_1 = x_1(1 - x_2), \\
&\ \vdots \\
&y_{n-1} = x_1 x_2 \cdots x_{n-1}(1-x_n), \\
&y_{n} = x_1 x_2 \cdots x_n.
\end{split}
\end{equation}
It is a cellular map and we notice that coordinates in the domain that are $0$ play an important role.
More specifically, if $x_i = 0$ then for every $j \geq i$ we have $y_{j} = 0$.

\subsection{Simplicial sets} \label{ss:simplicial sets}

Let us now review some basic notions from the theory of simplicial sets.
The \textit{simplex category} $\simplex$ is the full subcategory of $\Cat$ with objects
\begin{equation*}
[n] = 0 \to 1 \to \cdots \to n.
\end{equation*}
Its morphisms are generated by the well known \textit{coface} $\delta_i$ and \textit{codegeneracy $\sigma_i$ functors}, and we denote by $\cube_{\deg}([m], [n])$ the subset of morphism in $\cube([m], [n])$ of the form $\sigma_i \circ \tau$ with $\tau \in \cube([m], [n+1])$.


The category of \textit{simplicial sets} is the functor category $\sSet = \Fun(\simplex^{\!\op}, \Set)$ whose \textit{product} is defined by
\begin{equation*}
(X \times Y)[n] = X[n] \times Y[n], \qquad
(X \times Y)(\tau) = X(\tau) \times Y(\tau).
\end{equation*}
For elements $x \in X$ and $y \in Y$, we write $x \times y$ instead of $(x, y)$ for the associated element in $X \times Y$.

The \textit{standard $n$-simplex} is the simplicial set $\simplex^{\!n} = \simplex(-, [n])$.
The \textit{Yoneda embedding} $\simplex \to \sSet$ is the functor induced by $[n] \mapsto \simplex^{\!n}$.
For any simplicial set $X$ we have
\begin{equation*}
X[n] \cong \colim_{\simplex^{\!n} \to X} \simplex^{\!n}.
\end{equation*}

\subsection{Simplicial singular complex}

The assignment $[n] \to \gsimplex^{\!n}$ defines a functor $\simplex \to \Top$ whose Kan extension is known as \textit{geometric realization}.
It has a right adjoint $\sSing \colon \Top \to \sSet$ given by
\begin{equation*}
Z \to \Big([n] \to \Top(\gsimplex^{\!n}, Z)\Big),
\end{equation*}
and referred to as the \textit{simplicial singular complex} of the topological space $Z$.

\subsection{Simplicial chains} \label{ss:simplicial chains}

The functor of \textit{chains} $\chains \colon \sSet \to \Ch$ is the Kan extension along the Yoneda embedding of the functor $\simplex \to \Ch$ assigning to $[n]$ the chain complex whose degree $m$ part is the $R$ module
\begin{equation*}
\frac{R\{\simplex([m], [n])\}}{R\{\simplex_{\deg}([m], [n])\}}
\end{equation*}
and whose boundary is defined by
\begin{equation*}
\partial (\id_{[n]}) = \sum_{i=0}^{n} (-1)^i \ \id_{[n]} \circ \delta_i.
\end{equation*}
We remark that $\chains(\simplex^{\!n})$ is isomorphic to the cellular chains on $\gsimplex^{\!n}$.
For a topological space $Z$, the chain complex $\chains(\sSing Z)$ is referred to as the \textit{simplicial singular chains} of $Z$.

\subsection{$E_\infty$ structure} \label{ss:e infinity structures}

In \cite{medina2020prop1}, a similar construction to the one introduced in \cref{s:action} provides the chains of simplicial sets with a natural $U(\M)$-coalgebra structure.
It is also induced from a natural $\M$-bialgebra structure on the chains of standard simplices.
This $\M$-bialgebra structure on $\chains(\simplex^{\!n})$ is defined by the assignment
\begin{equation*}
\counit \mapsto \epsilon, \quad \coproduct \mapsto \Delta, \quad \product \mapsto \ast,
\end{equation*}
where
\begin{equation*}
\varepsilon \big( [v_0, \dots, v_q] \big) = \begin{cases} 1 & \text{ if } q = 0, \\ 0 & \text{ if } q > 0, \end{cases}
\end{equation*}
is the \textit{augmentation map},
\begin{equation*}
\Delta \big( [v_0, \dots, v_q] \big) = \sum_{i=0}^q [v_0, \dots, v_i] \otimes [v_i, \dots, v_q],
\end{equation*}
is the \textit{Alexander-Whitney diagonal}, and the following is an algebraic version of the \textit{join}:
\begin{equation*}
\left[v_0, \dots, v_p \right] \ast \left[v_{p+1}, \dots, v_q\right] = \begin{cases} (-1)^{p+|\pi|} \left[v_{\pi(0)}, \dots, v_{\pi(q)}\right] & \text{ if } v_i \neq v_j \text{ for } i \neq j, \\
0 & \text{ if not}, \end{cases}
\end{equation*}
where $\pi$ is the permutation that orders the totally ordered set of vertices and $(-1)^{|\pi|}$ is its sign.

\subsection{Surjection-like graphs} \label{ss:surjection-like graphs}

A \textit{surjection-like graph} is either the $(1,0)$-graph \counit\ or a $(1, r)$-graph obtained from
\begin{equation*}
	\def\r{.8}
	\boxed{\begin{tikzpicture}[scale=.4]
	\node[scale=\r] at (5,8){$1$};
	
	\draw (3,5.5)--(5,6.5)--(5,7.5);
	\draw (7,5.5)--(5,6.5);
	\draw (4,5.5)--(3.5,5.75);
	
	\node[scale=\r] at (3,5){$1$};
	\node[scale=\r] at (4,5){$2$};
	\node[scale=\r] at (5.3,5){$\dots$};
	\node[scale=\r] at (7,5){$r+d$};
	
	\node at (5,4){$\vdots$};
	
	\node[scale=\r] at (1.8, 2.5){$1$};
	\node[scale=\r] at (2.6, 2.5){$2$};
	\node[scale=\r*.7] at (3.4, 2.5){$\dots$};
	\node[scale=\r] at (4.2, 2.5){$k_1$};
	
	\draw (2,2)--(3,1)--(3,0);
	\draw (4,2)--(3,1);
	\draw (2.6,2)--(2.4,1.6);
	
	\node[scale=\r] at (3,-.5){$1$};
	
	\node at (5,1){\ $\cdots$};
	
	\node[scale=\r] at (5.8, 2.5){$1$};
	\node[scale=\r] at (6.6, 2.5){$2$};
	\node[scale=\r*.7] at (7.4, 2.5){$\dots$};
	\node[scale=\r] at (8.2, 2.5){$k_r$};
	
	\draw (6,2)--(7,1)--(7,0);
	\draw (8,2)--(7,1);
	\draw (6.6,2)--(6.4,1.6);
	
	\node[scale=\r] at (7,-.5){$r$};	
	\end{tikzpicture}}
\end{equation*}
by gluing the strands of the $(1,n+d)$-graph at the top to the input strands of the $(k_i,1)$-graphs at the bottom so that for each $i = 1, \dots, r$ the induced map 
\begin{equation*}
\{1, \dots, k_i\} \to \{1, \dots, r+d\}
\end{equation*}
is order preserving.

\begin{definition}
	The operad $\Sl$ is the suboperad of $U(\M)$ generated by elements represented by surjection-like graphs.
\end{definition}

The same proof used in \cite[Theorem 3.3.]{medina2020prop1} shows $\Sl$ is an $E_\infty$ operad.
We remark that this operad was used in \cite[Theorem A.11.]{medina2020prop1} to compare the action of the surjection operad of McClure-Smith \cite{mcclure2003multivariable} and Berger-Fresse \cite{berger2004combinatorial} on simplicial sets and that of $U(\M)$.

\subsection{Extended comultiplicativity} \label{ss:extended comultiplicativity}

Let $Y$ be a simplicial set and $X$ a cubical set.
By restriction of their $U(\M)$ coalgebra structures, both $\chains(Y)$ and $\chains(X)$ are $\Sl$ coalgebras, and we have the following generalization of the Serre's comultiplicativity. 

\begin{theorem} \label{t:extended comultiplicativity}
	For any topological space $Z$, the quasi-isomorphism $\chains(\sSing Z) \to \chains(\cSing Z)$ induced by precomposing with the Serre-Cartan map $\gcube^n \to \gsimplex^{\!n}$ is a morphisms of $E_\infty$ coalgebras, where the domain structure extends the Alexander-Whitney coproduct and the target the Serre coproduct.
\end{theorem}

We will obtain this result from a more general categorical statement.

\subsection{The simplicial cube} \label{ss:the simplicial cube}

For non-negative integers $n$ and $m$, we identify the set $\simplex^n_m$ with that of non-decreasing sequences $[v_0, \dots, v_m]$ with each $v_i \in \{0, \dots, n\}$.
For $\simplex^1_m = \simplex([m], [1])$ we reduce this notation writing
\begin{equation*}
\langle k \rangle = [\, \overbrace{0, \dots, 0}^{m+1-k}, \overbrace{1, \dots, 1}^{k}\,]
\end{equation*}
for $k \in \{0, \dots, m+1\}$.

Let us now consider the product of the simplicial interval with itself, given explicitly by
\begin{equation*}
\scube{n}_m = \big\{ \angles{k_1} \times \cdots \times \angles{k_n} \mid \angles{k_i} \in \simplex^1_m \big\}.
\end{equation*}
The simplicial sets $\simplex^{\!n}$ and $\scube{n}$ are isomorphic when $n$ is either $0$ or $1$.
For $n > 1$ we consider the inclusion
\begin{equation*}
\begin{tikzcd}[row sep=-3pt, column sep=small,
/tikz/column 1/.append style={anchor=base east},
/tikz/column 2/.append style={anchor=base west}]
\simplex^{\!n} \arrow[r, "\iota"] & \scube{n} \\
{[0, \dots, n]} \arrow[r, mapsto] & \angles{n} \times \cdots \times \angles{1},
\end{tikzcd}
\end{equation*}
and the projection $s \colon \scube{n} \to \iota(\simplex^{\!n})$ defined by
\begin{equation*}
[ \varepsilon_0^1, \dots, \varepsilon_m^1] \times \cdots \times [ \varepsilon_0^n, \dots, \varepsilon_m^n] \mapsto [ \eta_0^1, \dots, \eta_m^1] \times \cdots \times [ \eta_0^n, \dots, \eta_m^n]
\end{equation*}
with $\eta_i^k = \varepsilon_i^1 \cdots \varepsilon_i^k$.
These are maps of simplicial sets satisfying $s \circ \iota = \id_{\simplex^{\!n}}$.

\subsection{Triangulation and its right adjoint} \label{triangulation and its right adjoint}

The assignment $2^n \mapsto \scube{n}$ defines a functor $\cube \to \sSet$ with $\delta_i^\varepsilon$ inserting $[\varepsilon, \dots, \varepsilon]$ in the $i\th$ position, and $\sigma_i$ removing the $i\th$ factor.
The left Kan extension of this functor $\T \colon \cSet \to \sSet$ is referred to as \textit{triangulation}.
It admits a right adjoint $\U \colon \sSet \to \cSet$ defined by
\begin{equation*}
(\U X)(2^m) = \sSet \big( \scube{n}, \, X \big).
\end{equation*}
Although we do not use this fact, we mention that $\T$ and $\U$ define a Quillen equivalence when simplicial and cubical sets are considered as model categories \cite[8.4.30]{cisinski2006presheaves}.

\subsection{The Cartan-Serre chain map} \label{ss:the cartan-serre chain map}

Let $\psi \colon \gchains(\gcube^n) \to \gchains(\gsimplex^{\!n})$ be the chain map induced by the Cartan-Serre map $\gcube^n \to \gsimplex^{\!n}$.
Using the identification of the cellular chains of $\gcube^n$ and $\gsimplex^{\!n}$ with $\chains(\cube^n)$ and $\chains(\simplex^{\!n})$
provides the domain and target of $\psi$ with an $\Sl$-coalgebra structure.
We have the following key result whose proof will be given in  \cref{ss:comparison proof}.

\begin{lemma} \label{l:cartan-serre is e infinity}
	The map $\psi$ is a morphism of $\Sl$-coalgebras.
\end{lemma}

It is easy to verify that $\psi$ under this identification is equal to the composition
\begin{equation*}
\begin{tikzcd}[column sep=normal]
\chains(\cube^n) \arrow[r, "\chains(\T)"] &[-3pt]
\chains \big( \scube{n} \big) \arrow[r, "\chains(s)"] &[-3pt]
\chains(\simplex^{\!n}),
\end{tikzcd}
\end{equation*}
which will be use to establish the extended comultiplicativity of the comparison map we introduce next.

\subsection{Comparison map} \label{ss:comparison map}

For any simplicial set $Y$ the map $s \colon \scube{n} \to \simplex^{\!n}$ induces a map
\begin{equation*}
\begin{tikzcd}[column sep = small, row sep=tiny]
\sSet(\simplex^{\!n}, Y) \arrow[r] &
\sSet(\T \cube^n, Y) \\
\big( \simplex^{\!n} \xrightarrow{g} Y \big) \arrow[r, maps to] &
\big(\scube{n} \xrightarrow{s} \simplex^{\!n} \xrightarrow{g} Y \big),
\end{tikzcd}
\end{equation*}
for every $n \geq 0$, which, composing with the adjunction isomorphism, gives a map
\begin{equation*}
\begin{tikzcd}[column sep = small, row sep=tiny]
\sSet(\simplex^{\!n}, Y) \arrow[r] &
\cSet(\cube^n,\, \U Y) \\
\big( \simplex^{\!n} \xrightarrow{g} Y \big) \arrow[r, maps to] &
\Big(\cube^n \xrightarrow{\T} \big(\scube{n} \xrightarrow{g \circ s} Y \big)\Big).
\end{tikzcd}
\end{equation*}
These define a morphism of graded sets inducing a chain map $S_Y \colon \chains(Y) \to \chains(\U Y)$, referred to as the comparison map for $Y$, which is a quasi-isomorphism by an acyclic carrier argument.

\begin{theorem} \label{t:comparison map is e infinity}
	The comparison map for any simplicial set $Y$ is a morphism of $\Sl$ coalgebras.
\end{theorem}

\begin{proof}
	This follows from \cref{l:cartan-serre is e infinity} and the identification of $\psi$ with $\chains(s \circ \mathcal T)$
\end{proof}

\subsection{Proof of \cref{t:extended comultiplicativity}} \label{ss:proof extended comultiplicativity}

For any topological space $Z$, we need to show that the chain map $\chains(\sSing Z) \to \chains(\cSing Z)$ induced by precomposing each basis element with the Cartan-Serre map $\gcube^n \to \gsimplex^{\!n}$ is an $\Sl$ morphism.
This map factors as a composition
\begin{equation*}
\chains(\sSing Z) \to \chains(\U \sSing Z) \to \chains(\cSing Z)
\end{equation*}
where the first map is the comparison map of \cref{ss:comparison map} and the second is induced from a morphism of cubical set.
The first is a morphism of $\Sl$ coalgebras by \cref{t:comparison map is e infinity}, and the second by naturality.