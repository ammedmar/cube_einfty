% !TEX root = ../cube_einfty.tex

\section{The Cartan--Serre map} \label{s:comparison}

Let us consider, with their usual CW structures, the topological simplex $\gsimplex^n$ and the topological cube $\gcube^n$.
In \cite[p. 442]{serre1951homologie}, Serre described a quasi-isomorphism of algebras between the cubical and simplicial singular cochains of a topological space.
It is given by precomposing with a canonical cellular map $\cs \colon \gcube^n \to \gsimplex^n$ also considered in \cite[p.199]{eilenberg1953acyclic} where it is attributed to Cartan.

The goal of this section is to deduce from a more general categorical statement that this comparison map between singular cochains of a space is a quasi-isomorphism of $E_\infty$-algebras.

\subsection{Simplicial sets} \label{ss:simplicial sets}

\subsubsection{Categorical aspects}

We denote the \textit{simplex category} by $\simplex$, the category of \textit{simplicial sets} $\Fun(\simplex^\op, \Set)$ by $\sSet$, and the standard $n$-simplex $\yoneda \big( [n] \big)$ by $\simplex^n$.
As usual, we denote an element in $\simplex^n_m$ by a non-decreasing tuple $[v_0, \dots, v_m]$ with $v_i \in \{0, \dots, n\}$.
The \textit{Cartesian product} of simplicial sets is defined by the product of functors.
The \textit{simplicial $n$-cube} $\scube{n}$ is the $n^\th$-fold Cartesian product of $\simplex^1$.

\subsubsection{Topological aspects}

We will use the following model of the topological $n$-simplex:
\[
\gsimplex^n = \big\{ (y_1, \dots, y_n) \in \gcube^n \mid i \leq j \Rightarrow y_i \geq y_j \big\},
\]
whose cell structure associates $[v_0, \dots, v_m]$ with the subset
\begin{equation} \label{e:cell structure of gsimplex}
	\Big\{ \big( \underbrace{1, \dots, 1}_{v_0}, \underbrace{y'_1, \dots y'_1}_{v_1-v_0}, \dots, \underbrace{y'_m, \dots y'_m}_{v_m-v_{m-1}}, \underbrace{0, \dots, 0}_{n-v_m} \big) \mid y'_1 \geq \dots \geq y'_m \Big\}.
\end{equation}
The spaces $\gsimplex^n$ define a functor $\gsimplex^{(-)} \colon \simplex \to \CW $ with
\begin{align*}
	\sigma_i(x_1, \dots, x_n) &= (x_1, \dots, \widehat x_i, \dots, x_n) \\
	\delta_0(x_1, \dots, x_n) &= (1, x_1, \dots, x_n), \\
	\delta_i(x_1, \dots, x_n) &= (x_1, \dots, x_i, x_i, \dots, x_n), \\
	\delta_n(x_1, \dots, x_n) &= (x_1, \dots, x_n, 0).
\end{align*}
Its Yoneda extension is the \textit{geometric realization} functor.
It has a right adjoint $\sSing \colon \Top \to \sSet$ referred to as the \textit{simplicial singular complex} satisfying
\[
\sSing(\fZ)_n = \Top(\gsimplex^n, \fZ)
\]
for any topological space $\fZ$.

\subsubsection{Algebraic aspects}

The functor of (\textit{normalized}) \textit{chains} $\schains \colon \sSet \to \Ch$ is the composition of the geometric realization functor and that of cellular chains.
We denote the composition $\schains \circ \sSing$ by $\sSchains$ and omit the superscript $\simplex$ if no confusion may result from doing so.
For any $n \in \N$, define $\epsilon \colon \chains(\simplex^n) \to \k$ by
\[
\epsilon \big( [v_0, \dots, v_m] \big) =
\begin{cases}
	1 & \text{ if } m = 0, \\ 0 & \text{ if } m>0,
\end{cases}
\]
$\Delta \colon \chains(\simplex^n) \to \chains(\simplex^n)^{\ot 2}$ by
\[
\Delta \big( [v_0, \dots, v_m] \big) = \sum_{i=0}^m [v_0, \dots, v_i] \ot [v_i, \dots, v_m],
\]
and $\ast \colon \chains(\simplex^n)^{\ot 2} \to \chains(\simplex^n)$ by
\[
\left[v_0, \dots, v_p \right] \ast \left[v_{p+1}, \dots, v_m\right] = \begin{cases} (-1)^{p+|\sigma|} \left[v_{\sigma(0)}, \dots, v_{\sigma(m)}\right] & \text{ if } v_i \neq v_j \text{ for } i \neq j, \\
	0 & \text{ if not}, \end{cases}
\]
where $\sigma$ is the permutation that orders the totally ordered set of vertices and $(-1)^{|\sigma|}$ is its sign.
As shown in \cite[Theorem 4.2]{medina2020prop1} the assignment
\[
\counit \mapsto \epsilon, \quad \coproduct \mapsto \Delta, \quad \product \mapsto \ast,
\]
defines a natural $\Med$-bialgebra on the chains of standard simplicial sets.

\subsection{The cellular maps} \label{ss:comparison collapse map}

\subsubsection{The Eilenberg--Zilber maps}

For any permutation $\sigma \in \Sym_n$ let
\[
\gi_\sigma \colon \gsimplex^n \to \gcube^n
\]
be the inclusion defined by $(x_1, \dots, x_n) \mapsto (x_{\sigma(1)}, \dots, x_{\sigma(n)})$.
If $e$ is the identity permutation, we denote $\mathfrak{i}_{e}$ simply as $\mathfrak{i}$.
%and notice that it satisfies $\cs \circ\, \mathfrak{i} = \id_{\gsimplex^n}$.
The maps $\set{\mathfrak{i}_\sigma}_{\sigma \in \Sym_n}$ define a subdivision of $\gcube^n$ making it isomorphic to $\bars[\big]{\scube{n}}$ in $\CW$.
Using this identification, the identity map induces a cellular map
\[
\ez \colon \gcube^n \to \bars[\big]{\scube{n}}
\]
whose induced chain map is denoted by
\[
\EZ \colon \chains(\cube^n) \to \chains \big( \scube{n} \big).
\]
Using the isomorphism $\chains(\cube^n) \cong \chains(\cube^1)^{\ot n} \cong \chains(\simplex^1)^{\ot n}$, this map agrees with the usual Eilenberg--Zilber maps $\chains(\simplex^1)^{\ot n} \to \chains \big( \scube{n} \big)$.

For any topological space $\fZ$ we have a quasi-isomorphism
\begin{equation} \label{e:ezz}
	\begin{tikzcd} [column sep=small, row sep=0]
		\EZ_{\Schains(\fZ)} \colon &[-20pt] \cSchains(\fZ) \arrow[r] & \sSchains (\fZ) \\
		& (\gcube^n \to \fZ) \arrow[r, mapsto] &
		\sum\limits_{\mathclap{\sigma \in \Sym_n}} \sign(\sigma) \big( \gsimplex^n \xra{\gi_\sigma} \gcube^n \to \fZ \big).
	\end{tikzcd}
\end{equation}

\subsubsection{Cartan-Serre maps}

Let
\[
\cs \colon \gcube^n \to \gsimplex^n
\]
be the cellular map defined by $\cs(x_1, \dots, x_n) = (x_1,\ x_1 x_2, \, \dots \, , \ x_1 x_2 \dotsm x_n)$.
Its induced chain map is denoted by
\[
\CS \colon \chains(\cube^n) \to \chains(\simplex^n).
\]
For any topological space $\fZ$ we have a cellular map
\[
\begin{tikzcd} [column sep=small, row sep=0]
	\sSing(\fZ) \arrow[r] & \cSing (\fZ) \\
	(\gsimplex^n \to \fZ) \arrow[r, mapsto] & (\gcube^n \xra{\cs} \gsimplex^n \to \fZ)
\end{tikzcd}
\]
whose induced chain map is denoted
\[
\CS_{\Schains(\fZ)} \colon \sSchains(\fZ) \to \cSchains(\fZ).
\]

The map $\cs$ was considered in \cite[p. 442]{serre1951homologie} where it was stated that $\CS_{\Schains(\fZ)}$ is a natural quasi-isomorphisms of coalgebras.
We remark that in this reference the topological $n$-simplex is modeled as a subset of $\R^{n+1}$ instead of $\R^n$.

%\begin{lemma}
%	The map $\cs \colon \gcube^n \to \gsimplex^n$ satisfies:
%	\begin{align*}
%		\cs \circ \, \delta_i^1 = \delta_{i-1} \circ \cs
%	\end{align*}
%	for $i \in \{1, \dots, n\}$,
%	\[
%	\cs \circ \, \delta_i^0 \big( [01] \times \dots \times[01] \big) \subseteq \gsimplex^n_{n-1}
%	\]
%	for $i \in \{1, \dots, n-1\}$, and
%	\[
%	\cs \circ \, \delta_n^0 = \delta_n \circ \cs.
%	\]
%\end{lemma}
%
%The chain map induced by the Cartan--Serre projection is denoted by
%\[
%\CS \colon \chains(\cube^n) \to \chains(\simplex^n).
%\]

\subsection{The cellular maps and algebraic structures}

\subsubsection{Coalgebra structures}

We begin by stating the following classical result.

\begin{proposition} \label{p:ez is coalgebra map]}
	The chain map $\EZ \colon \chains(\cube^n) \to \chains \big( \scube{n} \big)$ is a quasi-isomorphism of coalgebras.
\end{proposition}

We also have the following.

\begin{proposition}
	The chain map $\CS \colon \chains(\cube^n) \to \chains(\simplex^n)$ is a quasi-isomorphism of coalgebras.
\end{proposition}

\begin{proof}
	Since $\CS$ is induced from a cellular map between contractible spaces it is a quasi-isomorphism compatible with the counit.
	To study the compatibility of $\CS$ with coproducts consider $n > 0$, since the case $n = 0$ is immediate, and let
	\[
	\Delta \big( [0,1]^{\ot n} \big) = \sum_{\lambda \in \Lambda} \pm \ x_1^{(\lambda)} \ot \cdots \ot x_n^{(\lambda)} \ot y_1^{(\lambda)} \ot \cdots \ot y_n^{(\lambda)},
	\]
	where $\Lambda$ parameterizes all choices of $x_i^{(\lambda)} \in \{[0], [0,1]\}$ and $y_i^{(\lambda)} \in \{[0,1], [1]\}$ such that
	\begin{align*}
		x_i^{(\lambda)} = [0]   & \iff y_i^{(\lambda)} = [0,1], \\
		x_i^{(\lambda)} = [0,1] & \iff y_i^{(\lambda)} = [1].
	\end{align*}
	By \cref{l:cs explicit}, the summands above not sent to $0$ by $\CS \ot \CS$ are those basis elements for which $x_i^{(\lambda)} = [0]$ implies $x_j^{(\lambda)} = [0]$ for all $i < j$.
	For any one such summand, its sign is positive and its image by $\CS \ot \CS$ is $[0, \dots, k] \ot [k, \dots, n]$ where $k+1 = \min \{i \mid x_i^{(\lambda)} = [0]\}$ or $k = n$ if this set is empty.
	The summands $[0, \dots, k] \ot [k, \dots, n]$ are precisely those appearing when applying the Alexander--Whitney coproduct to $[0, \dots, n] = \CS \big( [0,1]^{\ot n} \big)$.
\end{proof}

\subsubsection{No-go results}

One may hope for higher structure to be preserved by the $\EZ$ and $\CS$ maps.
We now provide some examples constraining the scope of these expectations.

\begin{example*}
	The Eilenberg--Zilber map is not compatible with higher coproducts.
	Let us consider, as in \cref{e:prop cup-i}, the following cup-$1$ coproduct
	\[
	\Delta_1 = (\ast \ot \id) \circ (\id \ot (12) \Delta) \circ \Delta.
	\]
	As a map $\chains(\cube^2; \Ftwo) \to \chains(\cube^2; \Ftwo)^{\ot 2}$, it satisfies
	\begin{multline*}
		\Delta_1 \big( [01] [01] \big) = \\
		[01][01] \ot [1][01] + [01][1] \ot [01][01] + [0][01] \ot [01][01] - [01][01] \ot [01][0].
	\end{multline*}
	Therefore,
	\begin{multline*}
		\EZ^{\ot 2} \circ \, \Delta_1 \big( [01][01] \big) = \\
		\big( 011\times001 + 001\times011 \big) \ot 11 \times 01 \ +\
		01\times11 \ot \big( 011\times001 + 001\times011 \big) \ +\ \\
		00\times01 \ot \big( 011\times001 + 001\times011 \big) \ +\
		\big( 011\times001 + 001\times011 \big) \ot 01\times00.
	\end{multline*}
	On the other hand, as a map $\chains(\simplex^n; \Ftwo) \to \chains(\simplex^n; \Ftwo)^{\ot 2}$ we have
	\[
	\Delta_1 [0,1,2] = [0,1,2] \ot [0,1] + [0,2] \ot [0,1,2] + [0,1,2] \ot [1,2].
	\]
	Therefore,
	\begin{multline*}
		\Delta_1 \circ \EZ \big( [01] [01] \big) = \Delta_1 \big( 011\times001 + 001\times011 \big) = \\
		011\times001 \ot 01\times00 \ + \
		01\times01 \ot 011\times001 \ + \
		011\times001 \ot 11\times01 \ + \ \\
		001\times011 \ot 00\times01 \ + \
		01\times01 \ot 001\times011 \ + \
		001\times011 \ot 01\times11.
	\end{multline*}
	We conclude that $\EZ^{\ot 2} \circ \, \Delta_1 \big( [01][01] \big) \neq \Delta_1 \circ \EZ \big( [01] [01] \big)$ since, for example, the basis element $01\times11 \ot 011\times001$ appears in left sum but not in the right one.
\end{example*}

\begin{example*}
	The Cartan--Serre map does not preserve $\M$-structures.
	Consider $\CS \colon \chains(\cube^2; \Ftwo) \to \chains(\simplex^2; \Ftwo)$.
	Since $\CS \big( [0][01] \big) = 0$ we have
	\[
	\CS \big( [1][1] \big) \ast \CS \big( [0][01] \big) = 0
	\]
	but
	\[
	\CS \Big( ([1][1]) \ast ([0][01]) \Big) =
	\CS \Big( [01][01] \Big) = [012].
	\]
	The reason for this incompatibility is that $\ast$ in the simplicial context is commutative whereas in the cubical context it is not.
	From the same observation we deduce the next construction.
\end{example*}

\begin{example*}
	The Cartan--Serre map does not preserve $\UM$-structures.
	Let
	\[
	\widetilde\Delta_1 = (\ast \ot \id) \circ (12) (\id \ot (12) \Delta) \circ \Delta.
	\]
	It can be computed that
	\[
	\CS\Big( \widetilde\Delta_1 \big( [01][01] \big) \Big) =
	T \Delta_1 \big( [012] \big)
	\]
	whereas
	\[
	\widetilde\Delta_1 \circ \CS \big( [01][01] \big) =
	\Delta_1 \big( [012] \big).
	\]
	We will restrict the $\UM$-coalgebra structure on simplicial and cubical chains to a sub-$E_\infty$-operad where this symmetry incompatibility is avoided.
\end{example*}

\subsubsection{Suboperad of surjection-like graphs}

The operad $\USL$ is generated as a suboperad of $\UM$ by all so-called \textit{surjection-like graphs}, i.e., immerse connected graphs of the form:
\begin{equation*}
\begin{tikzpicture}[scale=1]
\draw (0,0)--(0,-.6) node[below, scale=.75]{$1$};
\draw (0,0)--(.5,.5);
\draw (-.3, .3)-- (-.2,.5) node[scale=.75] at (-.2,.7) {\qquad $1\, \ \ 2\ \ ...\ \ k_1$};
\draw (-.5,.5)--(0,0);
\node[scale=.75] at (.11,.4){$..$.};

\node[scale=.75] at (1,0){$\cdots$};
\node[scale=.75] at (1,-.9){$\cdots$};

\draw (2,0)--(2,-.68) node[scale=.75, below]{$r$};
\draw (2,0)--(2.5,.5);
\draw (1.7, .3)--(1.8,.5) node[scale=.75] at (1.78,.7) {\qquad $1\, \ \ 2\ \ ...\ \ k_r$};
\draw (1.5,.5)--(2,0);
\node[scale=.75] at (2.11,.4){$..$.};

\draw (1,2.5)--(1,3) node[scale=.75, above]{$1$};
\draw (1,2.5)--(0,2) node[scale=.75, below]{$1$};
\draw (.25,2.125)--(.5,2) node[scale=.75, below]{$2$};
\draw (.5,2.25)--(1,2) node[scale=.75, below]{$3$};
\draw (1,2.5)--(2,2) node[scale=.75, below]{\ \quad $n + r$};
\node[scale=.75] at (1.5,1.75){$\cdots$};

\node[scale=.75] at (1,1.3) {$\vdots$};

\node at (2.85,0){};
\end{tikzpicture}
\end{equation*}
where there are no hidden vertices and the strands are joined so that the associated maps $\{1, \dots, k_j\} \to \{1, \dots, n+k\}$ are order-preserving.
We notice that the subcomplex of surjection-like $(1,r)$-graphs is contractible using the same chain contraction employed in \cite{medina2020prop1}.
This implies that the suboperad $\USL$ of $\UM$ is also $E_\infty$, and that the restriction to $\USL$ of the $\UM$-structures on simplicial and cubical chains provides them with another $E_\infty$-structure.

\subsubsection{Local theorem}

We devote this subsection to the proof of the following key result.

\begin{theorem} \label{t:main local}
	The chain map $\CS \colon \chains(\cube^n) \to \chains(\simplex^n)$ is a quasi-isomorphism of $\USL$-coalgebras.
\end{theorem}

By naturality, it suffices to show that for any $\Gamma \in \UM(r)$ represented by a surjection-like $(1,r)$-graph we have
\begin{equation} \label{e:Cartan--Serre E-coalgebra map}
	\CS^{\ot r} \! \circ \, \Gamma \big( [0,1]^{\ot n} \big) =
	\Gamma \circ \CS \big( [0,1]^{\ot n} \big).
\end{equation}

Let us start by providing a more explicit description of the $\CS$ map.

\begin{lemma} \label{l:cs explicit}
	Let $x = x_1 \ot \cdots \ot x_n \in \chains(\cube^n)_m$ be a basis element with $x_{q_i} = [0,1]$ for all $\{q_1 < \dots < q_m\}$.
	Let $p = \min \set[\big]{i \mid x_i = [0]}$ or $p = n+1$ if empty, then
	\[
	\CS(x) =
	\begin{cases}
		\big[ q_1-1, \ \dots , \ q_m-1, \ p-1 \big] & \text{ if } p > q_m, \\
		\hfil 0 & \text{ otherwise}.
	\end{cases}
	\]
\end{lemma}

\begin{proof}
	It can be verified using the cell structure of $\gsimplex^n$ described in \cref{e:cell structure of gsimplex}, the fact that the image of $\cs \circ \, \delta^0_i \colon \gcube^{n-1} \to \gsimplex^n$ for $i \in \set{1,\dots,n-1}$ is in a lower dimensional skeleton of $\gsimplex^n$, and the identities
	\[
	\cs \circ \, \delta_i^1 = \delta_{i-1} \circ \cs,
	\qquad
	\cs \circ \, \delta_n^0 = \delta_n \circ \cs,
	\]
	for $i \in \set{1, \dots, n}$.
\end{proof}

We will consider the basis of $\chains(\cube^n)$ as a poset with
\[
(x_1 \ot \cdots \ot x_n) \leq (y_1 \ot \cdots \ot y_n)
\]
if and only if $x_i \leq y_i$ for each $i \in \{1, \dots, n\}$ with respect to $[0] < [0,1] < [1]$.

\begin{lemma}
	Let $\Delta^{r-1}$ be the $(r-1)^\th$-fold iterated Serre coproduct.
	If
	\[
	\Delta^{r-1} \big([0,1]^{\ot n}\big) =
	\sum \pm \ x{(1)} \ot \cdots \ot x{(r)}
	\]
	with each $x(i) \in \chains(\cube^n)$ a basis element, then $x{(1)} \leq \cdots \leq x{(r)}$.
\end{lemma}

\begin{proof}
	For $r = 2$ we have for every $i \in \{1, \dots, n\}$ that
	\begin{align*}
		x(1)_i = [0]   & \iff x(2)_i = [0,1], \\
		x(1)_i = [0,1] & \iff x(2)_i = [1],
	\end{align*}
	and that neither $x(1)_i = [1]$ or $x(2)_i = [0]$ can occur, hence $x(1) \leq x(2)$.
	The claim for $r > 2$ follows from a straightforward induction argument.
\end{proof}

\begin{lemma}
	Let $x, y, z \in \chains(\cube^n)$ be basis elements.
	If $x, y \leq z$ then either $(x \ast y) = 0$ or $(x \ast y) \leq z$.
\end{lemma}

\begin{proof}
	Recall that
	\begin{align*}
		(x_1 \ot \cdots \ot x_n) \ast (y_1 \ot \cdots \ot y_n) =
		(-1)^{|x|} \sum_{i=1}^n x_{<i}\, \epsilon(y_{<i}) \ot x_i \ast y_i \ot \epsilon(x_{>i}) \, y_{>i}.
	\end{align*}
	By assumption, for every $i \in \{1, \dots, n\}$ we have $x_{<i} \leq z_{<i}$ and $y_{>i} \leq z_{>i}$.
	If $x_i \ast y_i \neq 0$ then $x_i \ast y_i = [0,1]$ and either $x_i = [1]$ or $y_i = [1]$ which implies $z_i = [1]$ as well, so $x_i \ast y_i \leq z_i$.
\end{proof}

\begin{lemma}
	Let $x, y \in \chains(\cube^n)$ be basis elements.
	If $x \leq y$ then
	\begin{equation} \label{e:cs collapse as algebra map}
		\CS(x \ast y) = \CS(x) \ast \CS(y).
	\end{equation}
\end{lemma}

\begin{proof}
	We present this proof in the form of three claims.
	We use \cref{l:cs explicit}, the assumption $x \leq y$, and the fact that the join of basis elements in $\chains(\simplex^n)$ sharing a vertex is $0$ without explicit mention. \newline

	\noindent \textit{Claim 1}.
	If $\CS(x) = 0$ or $\CS(y) = 0$ then for every $i \in \{1, \dots, n\}$
	\begin{equation} \label{e:zero for join}
		\CS \big( x_{<i}\, \epsilon(y_{<i}) \ot x_i \ast y_i \ot \epsilon(x_{>i}) \, y_{>i} \big) = 0.
	\end{equation}

	Assume $\CS(x) = 0$, that is, there exists a pair $p < q$ such that $x_p = [0]$ and $x_q = [0,1]$, then \eqref{e:zero for join} holds since:
	\begin{enumerate}
		\item If $i > q$, then $x_p$ and $x_q$ are part of $x_{<i}$.
		\item If $i = q$, then $x_q \ast y_q = 0$ for any $y_q$.
		\item If $i < q$, then $\epsilon(x_{>i}) = 0$.
	\end{enumerate}
	Similarly, if there is a pair $p < q$ such that $y_p = [0]$ and $y_q = [0,1]$,  then \eqref{e:zero for join} holds since:
	\begin{enumerate}
		\item If $i < p$, then $y_p$ and $y_q$ are part of $y_{>i}$.
		\item If $i = p$, then $x_i = [0]$ and $x_i \ast y_i = 0$.
		\item If $i > p$, then either $x_i \ast y_i = 0$ or $x_i \ast y_i = [0,1]$ and $x_p = [0]$.
	\end{enumerate}
	This proves the first claim and identity \eqref{e:cs collapse as algebra map} under its hypothesis.

	\noindent \textit{Claim 2}.
	If $\CS(x) \neq 0$ and $\CS(y) \neq 0$ then
	\[
	\CS(x \ast y) =
	\CS \big( x_{<p_x} \epsilon(y_{<p_x}) \ot \, x_{p_x} \! \ast y_{p_x} \ot \epsilon(x_{>p_x}) \, y_{>p_x} \big)
	\]
	if $p_x = \min \big\{ i \mid x_i = [0] \big\}$ is well-defined and $x \ast y = 0$ if not.

	Assume $p_x$ is not well-defined, i.e., $x_i \neq [0]$ for all $i \in \{1, \dots, n\}$.
	Given that $x \leq y$ we have that $[0] < x_i$ implies $x_i \ast y_i = 0$, and the claim follows in this case.

	Assume $p_x$ is well-defined.
	We will show that for all $i \in \{1,\dots,n\}$ with the possible exception of $i = p_x$ we have
	\begin{equation} \label{e:case main lemma third claim}
		\CS \big( x_{<i} \, \epsilon(y_{<i}) \ot \, x_{i} \! \ast y_{i} \ot \epsilon(x_{>i}) \, y_{>i} \big) = 0
	\end{equation}
	This follows from:
	\begin{enumerate}
		\item If $i < p_x$ and $x_i = [1]$ then $y_i = [1]$ and $x_i \ast y_i = 0$.
		\item If $i < p_x$ and $x_i = [0,1]$ then $x_i \ast y_i = 0$ for any $y_i$.
		\item If $i > p_x$ then \cref{l:cs explicit} implies the claim since $x_{p_x} = [0]$ and $x_i \ast y_i \neq 0$ iff $x_i \ast y_i = [0,1]$.
	\end{enumerate}

	\noindent \textit{Claim 3}.
	If $\CS(x) \neq 0$ and $\CS(y) \neq 0$ then \eqref{e:cs collapse as algebra map} holds.

	Let us assume that $\big\{ i \mid x_i = [0] \big\}$ is empty, which implies the analogous statement for $y$ since $x \leq y$.
	Since neither of $x$ nor $y$ have a factor $[0]$ in them, \cref{l:cs explicit} implies that the vertex $[n]$ is in both $\CS(x)$ and $\CS(y)$, which implies $\CS(x) \ast \CS(y) = 0$ as claimed.

	Assume now that $p_x = \big\{ i \mid x_i = [0] \big\}$ is well defined, and let $\{q_1 < \dots < q_m\}$ with $x_{q_i} = [0,1]$ for $i \in \{1,\dots,m\}$.
	Since $\CS(x) \neq 0$ \cref{l:cs explicit} implies that $p_x > q_m$, so $\epsilon(x_{>p_x}) = 1$ and Claim 2 implies
	\[
	\CS(x \ast y) =
	\CS \big( x_{<p_x} \epsilon(y_{<p_x}) \ot \, x_{p_x} \! \ast y_{p_x} \ot y_{>p_x} \big).
	\]
	We have the following cases:
	\begin{enumerate}
		\item If $\epsilon(y_{<p_x}) = 0$ then there is $q_i$ such that $y_{q_i} = [0,1]$ so $[q_i-1]$ is in both $\CS(x)$ and $\CS(y)$.
		\item If $\epsilon(y_{p_x}) \neq 0$ and $y_{p_x} \in \{[0], [0,1]\}$ then $x_{p_x} \ast y_{p_x} = 0$ and $[p_x-1]$ is in both $\CS(x)$ and $\CS(y)$.
		\item If $\epsilon(y_{p_x}) \neq 0$ and $y_{p_x} = [1]$ let $\{\ell_1 < \dots < \ell_k\}$ be such that $y_{\ell_j} = [0,1]$ and let $p_y > \ell_k$ be either $n+1$ or $\min\{j \mid y_j = \{0\}\}$ then
		\begin{align*}
			\CS(x \ast y) & =
			\CS \big( x_{< p_x} \ot x_{p_x} \ast y_{p_x} \ot y_{> p_y} \big) \\ & =
			[q_1-1, \dots, q_m-1, p_x-1, \ell_1-1, \dots, \ell_k-1, p_y-1] \\ & =
			\CS(x) \ast \CS(y)
		\end{align*}
	\end{enumerate}
	which concludes the proof of Claim 3 and this lemma.
\end{proof}

This sequence of lemmas provides a proof of \cref{t:main local} using the decomposition of any surjection-like graph into pieces \coproduct \ and \product and the fact that $\USL$ is generated by these.
In the next subsection we will use \cref{t:main local} to deduce from a more general categorical statement that, for any topological space $\fZ$, the chain map $\CS_{\Schains(\fZ)} \colon \sSchains(\fZ) \to \cSchains(\fZ)$ is a quasi-isomorphism of $E_\infty$-coalgebras, specifically, of $\USL$-coalgebras.

\subsection{Categorical reformulation}

The assignment $2^n \mapsto \scube{n}$ defines a functor $\cube \to \sSet$ with $\delta_i^\varepsilon \colon \scube{n} \to \scube{(n+1)}$ inserting $[\varepsilon, \dots, \varepsilon]$ as $i^\th$ factor and $\sigma_i \colon \scube{(n+1)} \to \scube{n}$ removing the $i^\th$ factor.
Its Yoneda extension is denoted by
\[
\triangulate \colon \cSet \to \sSet.
\]
This functor admits a right adjoint
\[
\cubify \colon \sSet \to \cSet
\]
defined, as usual, by
\[
\cubify(Y)(2^n) = \sSet \big( \scube{n}, \, Y \big).
\]
Although we do not use this fact, we mention that, as proven in \cite[\S 8.4.30]{cisinski2006presheaves}, the pair $(\triangulate,\, \cubify)$ defines a Quillen equivalence when $\sSet$ and $\cSet$ are considered as model categories.

\subsubsection{Eilenberg--Zilber maps}

We have the following generalization of the map $\EZ \colon \chains(\cube^n) \to \chains(\scube{n})$.

\begin{definition}
	Let $X$ be a cubical set.
	The \textit{Eilenberg--Zilber subdivision}
	\[
	\EZ_X \colon \cchains(X) \to \schains(\triangulate(X))
	\]
	is the chain map induced by the natural cellular map
	\[
	\bars{X} \defeq
	\colim_{\cube^n \downarrow X} \gcube^n \xra{\ez}
	\colim_{\cube^n \downarrow X} \bars[\big]{\scube{n}} \defeq
	\bars[\big]{\triangulate(X)}.
	\]
\end{definition}

\begin{proposition}
	For any cubical set $X$ the Eilenberg--Zilber map $\EZ \colon \cchains(X) \to \schains(\triangulate X)$ is a quasi-isomorphism of coalgebras.
\end{proposition}

\begin{proof}
	This follows directly from \cref{p:ez is coalgebra map]}.
\end{proof}

% TO BE EXPLORED LATER
%\begin{proposition}
%	The map $\EZ_{\Schains(\fZ)}$ factors through $\EZ_{\cSing(\fZ)}$.
%	Explicitly,
%	\[
%	\begin{tikzcd}[row sep=0, column sep=small]
%		\cSchains(\fZ) \arrow[r] &
%		\schains \big( \triangulate \cSing(\fZ) \big) \arrow[r] &
%		\sSchains(\fZ) \\
%		(\gcube^n \xra{f} \fZ) \arrow[r, mapsto] &
%		\big( \bars{\scube{n}} \xra{f} \fZ \big) \arrow[r] &
%		\sum\limits_{\mathclap{\sigma \in \Sym_n}} \sign(\sigma) \big( \gsimplex^n \xra{\gi_\sigma} \bars{\scube{n}} \xra{f} \fZ \big).
%	\end{tikzcd}
%	\]
%\end{proposition}
%
%\begin{proposition}
%	The map ... coalgebra q-i.
%\end{proposition}

\subsubsection{Cartan--Serre maps}

\begin{definition}
	The simplicial map $\ccs \colon \scube{n} \to \simplex^n$ is defined by
	\[
	[\varepsilon_0^1, \dots, \varepsilon_m^1]
	\times \dots \times
	[\varepsilon_0^n, \dots, \varepsilon_m^n]
	\mapsto
	[v_0, \dots, v_m]
	\]
	where $v_i = \varepsilon_i^1 + \varepsilon_i^1 \varepsilon_i^2 + \dots + \varepsilon_i^1 \dotsm \varepsilon_i^n$.
\end{definition}

\begin{proposition}
	The maps $\cs$ and $\bars{\ccs} \circ \ez$ agree in $\CW(\gcube^n, \gsimplex^n)$.
\end{proposition}

\begin{definition}
	Let $Y$ be a simplicial set.
	The \textit{Cartan--Serre comparison}
	\[
	\CS_Y \colon \schains(Y) \to \cchains(\cubify Y)
	\]
	is the linear map induced by sending a simplex $y \in Y_n$ to the composition
	\[
	\scube{n} \xra{\ccs} \simplex^n \xra{\xi_y} Y
	\]
	where $\xi_y \colon \simplex^n \to Y$ is the characteristic map determined by $\xi_y \big( [n] \big) = y$.
\end{definition}

\begin{theorem} \label{t:main comparison}
	For any simplicial set $Y$ the Cartan--Serre map $\CS_Y \colon \schains(Y) \to \cchains(\cubify Y)$ is a quasi-isomorphism of $E_\infty$-coalgebras, specifically, of $\USL$-coalgebras.
\end{theorem}

\begin{proof}
	This is a direct consequence of \cref{t:main local} following from standard category theory argument which we now present.
	Consider the isomorphism
	\[
	\chains(\cubify Y) \cong \displaystyle\bigoplus_{n\in\N} \chains(\cube^n) \ot \k\set[\Big]{\sSet \big( \scube{n}, \simplex^n \big)} \Big/ \sim \,
	\]
	and the canonical inclusions:
	\[
	\begin{tikzcd} [row sep=0, column sep=small]
		\chains(\cube^n) \arrow[r] &
		\displaystyle\bigoplus_{m \in \N} \Hom \big( \chains(\cube^m),\, \chains(\cube^n) \big) \\
		(2^m \xra{\delta} 2^n) \arrow[r, mapsto] &
		\big(\chains(\cube^m) \xra{\chains(\delta)} \chains(\cube^n)\big)
	\end{tikzcd}
	\]
	and
	\[
	\begin{tikzcd} [row sep=0, column sep=small]
		\displaystyle \bigoplus_{n \in \N}
		\k\set[\Big]{\sSet\big(\scube{n}, \simplex^n\big)} \arrow[r] &
		\displaystyle \bigoplus_{n \in \N}
		\Hom \Big( \chains \big( \scube{n} \big),\, \chains(\simplex^n) \Big) \\
		\big( \scube{n} \xra{f} \simplex^n \big) \arrow[r, mapsto] &
		\Big( \chains \big( \scube{n} \big) \xra{\chains(f)} \chains(\simplex^n) \Big).
	\end{tikzcd}
	\]
	We can use these and the naturality of $\EZ$ to construct the following chain map which is an isomorphism onto its image.
	\[
	\begin{tikzcd}[column sep=small, row sep=0]
		\chains(\cubify Y) \arrow[r] &
		\displaystyle\bigoplus_{n\in\N} \Hom \big( \chains(\cube^n),\, \chains(Y)\, \big) \\
		(\delta \ot f) \arrow[r, mapsto]& \big( \chains(f) \circ \EZ \circ \chains(\delta) \, \big).
	\end{tikzcd}
	\]
	Using the naturality of the map $\Gamma^\square \colon \chains(\cubify Y) \to \big( \chains(\cubify Y) \big)^{\ot r}$ associated to any $\Gamma \in \USL$ we have that
	$\Gamma^\square(\delta \ot f)$ corresponds to $(\chains(f) \circ \EZ)^{\ot r} \circ \Gamma^\square \circ \chains(\delta)$.
	On the other hand, the map $\CS_Y$ corresponds to
	\[
	\begin{tikzcd}[column sep=small, row sep=0]
		\chains(Y)_n \arrow[r] & \chains(\cubify Y)_n \\
		y \arrow[r,mapsto]& \big( \chains(\xi_y) \circ \CS \big)
	\end{tikzcd}
	\]
	where $\xi_y \colon \simplex^n \to Y$ is determined by $\xi_y \big( [n] \big) = y$, and we used that $\CS = \chains(\ccs) \circ \EZ$ to ensure the above assignment is well defined.
	The image of $\Gamma^\triangle(y)$ corresponds to $\chains(\xi_y)^{\ot r} \circ \Gamma^\triangle \circ \CS$.
	So the claim follows from the identity
	\begin{align*}
		\Gamma^\cube \big( 2^n \ot (\xi_y \circ \ccs) \big) &=
		(\chains(\xi_y \circ \ccs) \circ \EZ)^{\ot r} \circ \Gamma^\square  \\ &=
		\chains(\xi_y)^{\ot r} \circ \CS^{\ot r} \circ \, \Gamma^\square  \\ &=
		\chains(\xi_y)^{\ot r} \circ \, \Gamma^\triangle \circ \CS,
	\end{align*}
	where we used that $\CS^{\ot r} \circ \, \Gamma^\square = \Gamma^\triangle \circ \CS$ as proven in \cref{t:main local}.
\end{proof}

%\begin{proposition}
%	The maps $\CS_{\Schains(\fZ)}$ and $\CS_{\sSing(\fZ)}$ agree in $\Ch(...)$.
%\end{proposition}

\begin{corollary}
	For any topological space $\fZ$ the chain map $\CS_{\Schains(\fZ)} \colon \sSchains(\fZ) \to \cSchains(\fZ)$ is a quasi-isomorphism of $E_\infty$-coalgebras, specifically, of $\USL$-coalgebras.
\end{corollary}

\begin{proof}
	The map $\CS_{\Schains(\fZ)}$ factors as $\sSchains(\fZ) \to \cchains \big( \cubify \sSing(\fZ) \big) \to \cSchains(\fZ)$ where the first map is $\CS_{\sSing(\fZ)}$, proven in \cref{t:main comparison} to be a quasi-isomorphism of $\USL$-coalgebras, and the second is induced by a cubical map $\cubify \sSing(\fZ) \to \cSing(\fZ)$ described next, which is therefore a morphisms of $\USL$-coalgebras.
	The two-out-of-three property of quasi-isomorphisms concludes the proof.
	Using the adjunction $\sSet \big( \scube{n}, \sSing(\fZ) \big) \cong \Top \big( \bars{\scube{n}}, \fZ \big)$ we define
	\[
	\begin{tikzcd} [row sep=0, column sep=small]
		\cubify \sSing(\fZ) \arrow[r] &
		\cSing(\fZ) \\
		\big( \bars{\scube{n} } \xra{f} \fZ \big) \arrow[r, mapsto] &
		\big( \gcube^n \xra{\ez} \bars{\scube{n}} \xra{f} \fZ \big)
	\end{tikzcd}
	\]
	which is a cubical map given the naturality of $\ez$.
\end{proof}


% TO BE EXPLORED LATER
%\subsubsection{The other map}
%
%\begin{definition}
%	The simplicial map $\ci \colon \simplex^n \to \scube{n}$ is defined by
%	\[
%	[0, \dots, n] \mapsto \varepsilon^1 \times \dots \times \varepsilon^n
%	\]
%	where $\varepsilon^j = [\varepsilon^j_0, \dots, \varepsilon^j_n]$ with
%	\[
%	\varepsilon^j_i =
%	\begin{cases}
%		0, & i \leq j, \\
%		1, & i > j. \\
%	\end{cases}
%	\]
%\end{definition}
%
%\begin{proposition}
%	Up to cellular isomorphisms the maps $\mathfrak{i}$ and $\bars{\ci}$ agree.
%\end{proposition}
%
%It has a left inverse
%\[
%\rI_{\,\cubify Y} \colon \cchains(\cubify Y) \to \schains(Y)
%\]
%induced by sending a cube $g \in \cubify Y$ to the image of the identity $[n]$ by the composition
%\[
%\simplex^n \xra{\incl} \scube{n} \xra{g} Y.
%\]