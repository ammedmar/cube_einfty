% !TEX root = ../cube_einfty.tex

\section{Introduction} \label{s:introduction}

Instead of simplices, in his groundbreaking work on fibered spaces Serre considered cubes as the basic shapes used to define cohomology, stating that:
\begin{displaycquote}[p.431]{serre1951homologie}
	Il est en effet evident que ces derniers se pretent mieux que les simplexes a l'etude des produits directs, et, a fortiori, des espaces fibres qui en sont la generalisation.
\end{displaycquote}
Cubical sets, a model for the homotopy category, were considered by Kan \cite{kan1955abstract, kan1956abstract} before introducing simplicial sets, they are central to nonabelian algebraic topology \cite{brown2011nonabelian}, and have become important in Voevodsky's program for univalent foundations and homotopy type theory \cite{kapulkin2020straightening, mortberg2017cubical}.
Other areas that highlight the relevance of cubical methods are applied topology, where cubical complexes are ubiquitous in the study of images \cite{tomasz2004computational}, condensed matter physics, where models on cubical lattices are central -- for example the Ising model -- \cite{baxter1985exactlysolved}, and geometric group theory \cite{gromov1987hyperbolic}, where fundamental results have been obtained considering actions on certain cube complexes characterized combinatorially \cite{agol2013haken}.

Cubical cochains are equipped with the \textit{Serre algebra structure}, a lift to the cochain level of the graded ring structure in cohomology.
Using an acyclic carrier argument it can be shown that this product is commutative up to coherent homotopies in a non-canonical way.
The study of such objects, referred to as $E_\infty$-algebras, has a long history, where (co)homology operations \cite{steenrod1962cohomology, may1970general}, the recognition of infinite loop spaces \cite{boardman1973homotopy, may1972geometry} and complete algebraic models of the $p$-adic homotopy category \cite{mandell2001padic} are key milestones.
The goal of this work is to introduce a description of an explicit $E_\infty$-algebra structure naturally extending the Serre algebra structure.

We use the combinatorial model of the $E_\infty$-operad $\UM$ obtained from the finitely presented prop $\M$ introduced in \cite{medina2020prop1}.
The resulting $\UM$-algebra structure on cubical cochains is induced from a natural $\M$-bialgebra structure on the cochains of standard cubes, which is determined by only three linear maps.
To our knowledge, this is the first effective construction of an $E_\infty$-algebra structure on cubical cochains.
Non-constructively, this result could be obtained using a lifting argument based on the cofibrancy of the reduced version of the operad $\UM$ in the model category of operads \cite{hinich1997homological, berger2003modelcategory}, but this existence statement misses the rich combinatorial structure present in our effective construction.

As described in \cite{medina2020prop1}, the operad $\UM$ also acts on simplicial cochains extending the Alexander--Whitney algebra structure.
We use a construction of Cartan and Serre to relate these cubical and simplicial $E_\infty$-structures.
More specifically, in \cite[p. 442]{serre1951homologie}, Serre describes for any topological space $Z$ a natural quasi-isomorphism
\[
\CScomp_Z^\vee \colon \cScochains(Z) \to \sScochains(Z)
\]
between its cubical and simplicial singular cochains.
Furthermore, he states this to be a quasi-isomorphism of algebras with respect to the Serre and Alexander--Whitney structures.
In the present work we deduce from a statement at the level of general simplicial sets that $\CScomp_Z^\vee$ is in fact a quasi-isomorphisms of $E_\infty$-algebras.
More specifically, let $\cubify$ be the right adjoint to the triangulation functor from cubical to simplicial sets.
We construct a natural quasi-isomorphism of $E_\infty$-algebras
\[
\CScomp_Y^\vee \colon \ccochains(\cubify Y) \to \scochains(Y)
\]
for any simplicial set $Y$, which factors $\CScomp_Z^\vee$ when $Y = \sSing(Z)$.

We now mention three application of the contributions in this paper.
For every prime $p$, the mod $p$ cohomology of a space is equipped with natural stable endomorphisms known as Steenrod operations \cite{steenrod1962cohomology}.
Following an operadic viewpoint developed by May \cite{may1970general}, in \cite{medina2021may_st} we effectively described a May--Steenrod structure on $\UM$, i.e., a compatible choice of elements in $\UM$ that represent Steenrod operations on the mod~$p$ homology of $\UM$-algebras.
Since, as proven in this article, cubical cochains are equipped with a $\UM$-structure, we use this May--Steenrod structure to extend the cubical cup-$i$ products of \cite{kadeishvili1999coproducts} and \cite{pilarczyk2016cubical} to a family of cochain level multioperations representing Steenrod operations at every prime.
Furthermore, the effective nature of these constructions permitted the implementation of these multioperations and associated Steenrod operations in the computer algebra system \texttt{ComCh} \cite{medina2021comch}.

For a closed smooth manifold $M$, in \cite{medina2021flowing} we compared a cochain complex generated by manifolds with corners over $M$, and the complex of cubical cochains defined by a choice of cubulation of $M$.
We used a canonical vector field associated to the cubulation to compare multiplicatively these two models of ordinary cohomology, whose product structures are respectively given by transverse intersection and the Serre product.
With the explicit description introduced here of an $E_\infty$-structure on cubical cochains, we expect to build on this multiplicative comparison and, using a coherent family of vector fields, describe the corresponding $E_\infty$-structure on geometric cochains extending the transverse intersection product.
For more details regarding this geometric model of cohomology please consult \cite{medina2022foundations}.

Our construction of an $E_\infty$-algebra structure on cubical cochains is obtained by dualizing an $E_\infty$-coalgebra structure on cubical chains.
In the fifties, Adams introduced in \cite{adams1956cobar} a comparison map
\[
\theta_Z \colon \cobar S^\simplex(Z,z) \to S^\cube(\loops_z Z)
\]
from his cobar construction on the simplicial singular chains of a pointed space $(Z,z)$ to the cubical singular chains on its based loop space $\loops_z Z$.
This comparison map is a quasi-isomorphism of algebras, which was shown by Baues \cite{baues1998hopf} to be one of bialgebras by considering Serre's cubical coproduct, see also \cite{galvez-carrilo202hopf1} for a generalization of Baues construction.
In \cite{medina2021cobar} we use the contributions of this paper to generalize Baues' result by showing that Adams' comparison map is a quasi-isomorphism of $E_\infty$-bialgebras, i.e., of monoids in the category of $\UM$-coalgebras, and to relate the cobar and Kan loop group constructions as functors to the category of $E_\infty$-coalgebras.

\subsection*{Outline}

We recall some basic notions from homological algebra and category theory in \cref{s:preliminaries}.
The required concepts from the theory of operads and props is reviewed in \cref{s:operads and props}, including the definition of the prop $\M$.
\cref{s:action} contains our main contribution; an explicit natural $\M$-bialgebra structure on the chains of standard cubes and, from it, a natural $E_\infty$-coalgebra structure on the chains of cubical sets.
The comparison between simplicial and cubical cochains is presented in \cref{s:the cartan-serre comparison map}, where we show that the Cartan--Serre comparison map is a quasi-isomorphism of $E_\infty$-algebras.