\section{Introduction} \label{s:introduction}

Instead of simplices, in his groundbreaking work on fibered spaces Serre considered cubes as the basic shapes used to define cohomology, stating that:
\begin{displaycquote}[p.431]{serre1951homologie}
	Il est en effet evident que ces derniers se pretent mieux que les simplexes a l'etude des produits directs, et, a fortiori, des espaces fibres qui en sont la generalisation.
\end{displaycquote}
Cubical sets, a model for the homotopy category, were considered by Kan \cite{kan1955abstract, kan1956abstract} before introducing simplicial sets, they are central to nonabelian algebraic topology \cite{brown2011nonabelian}, and have become important in Voevodsky's program for univalent foundations and homotopy type theory \cite{kapulkin2020straightening, mortberg2017cubical}.
Other areas that highlight the relevance of cubical methods are applied topology, where cubical complexes are ubiquitous in the study of images \cite{tomasz2004computational}, condensed matter physics, where models on cubical lattices are central \cite{baxter1985exactlysolved}, and geometric group theory \cite{gromov1987hyperbolic}, where fundamental results have been obtained considering actions on certain cube complexes characterized combinatorially \cite{agol2013haken}.

Cubical cochains are equipped with the \textit{Serre algebra structure}, a lift to the cochain level of the graded ring structure in cohomology.
Using an acyclic carrier argument it can be shown that this product is commutative up to coherent homotopies in a non-canonical way.
The study of such objects, referred to as $E_\infty$-algebras, has a long history, where (co)homology operations \cite{steenrod1962cohomology, may1970general}, the recognition of infinite loop spaces \cite{boardman1973homotopy, may1972geometry} and complete algebraic models of the $p$-adic homotopy category \cite{mandell2001padic} are key milestones.
The goal of this work is to introduce a description of an explicit $E_\infty$-algebra structure naturally extending the Serre algebra structure, and relate it to one on simplicial cochains extending the Alexander--Whitney algebra structure.

We use the combinatorial model of the $E_\infty$-operad $\UM$ obtained from the finitely presented prop $\M$ introduced in \cite{medina2020prop1}.
The resulting $\UM$-algebra structure on cubical cochains is induced from a natural $\M$-bialgebra structure on the chains of representable cubical sets, which is determined by only three linear maps.
To our knowledge, this is the first effective construction of an $E_\infty$-algebra structure on cubical cochains.
Non-constructively, this result could be obtained using a lifting argument based on the cofibrancy of the reduced version of the operad $\UM$ in the model category of operads \cite{hinich1997homological, berger2003modelcategory}, but this existence statement is not very useful in concrete situations.
To illustrate the advantages of an effective construction let us consider a prime $p$.
The mod $p$ cohomology of spaces is equipped with natural stable endomorphisms, known as Steenrod operations \cite{steenrod1962cohomology}.
Following an operadic viewpoint developed by May \cite{may1970general}, in \cite{medina2021may_st} we exhibited integral elements in $\UM$ representing Steenrod operations on the mod~$p$ homology of $\UM$-algebras.
Since, as proven in this article, the cochains of a cubical set are equipped with a $\UM$-algebra structure, we obtain natural cochain level multioperations for cubical sets representing Steenrod operation at every $p$.
This cubical cup-$(p,i)$ products are explicit enough to have been implemented in the open source computer algebra system \href{https://comch.readthedocs.io/en/latest/}{\texttt{ComCH}} \cite{medina2021comch}.

We now turn to the comparison between cubical and simplicial cochains.
In \cite[p. 442]{serre1951homologie}, Serre described for any topological space $\fZ$ a natural quasi-isomorphism
\begin{equation} \label{e:cs on singular cochains}
	\cScochains(\fZ) \to \sScochains(\fZ)
\end{equation}
between its cubical and simplicial singular cochains, stating this to be a quasi-isomorphism of algebras with respect to the usual structures.
We will consider a well known Quillen equivalence
\[
\begin{tikzcd}[column sep=0]
	\sSet \arrow[rr, "\cubify"', bend right] & \perp & \arrow[ll,"\triangulate"', bend right] \cSet
\end{tikzcd}
\]
between simplicial and cubical sets, and construct a natural chain map
\begin{equation} \label{e:intro main map}
	\ccochains(\cubify Y) \to \scochains(Y)
\end{equation}
for every simplicial set $Y$.
In \cite{medina2020prop1}, a natural $\UM$-algebra structure extending the Alexander--Whitney coalgebra structure was constructed on simplicial sets.
With respect to it and the one defined here for cubical sets we have the following results after passing to a sub-$E_\infty$-operad of $\UM$.

\begin{theorem*}
	The map presented in \cref{e:intro main map} is a quasi-isomorphism of $E_\infty$-algebras.
\end{theorem*}

From this result, stated as \cref{t:main comparison}, we deduce the following two.
The first one concerns the triangulation functor $\triangulate$ and it is stated more precisely as \cref{c:zig-zag}.

\begin{corollary*}
	There is a natural zig-zag of $E_\infty$-algebra quasi-isomorphisms between the cochains of a cubical set and those of its triangulation.
\end{corollary*}

The next one concerns the map presented in \cref{e:cs on singular cochains}, relating the cubical and simplicial singular cochains of a space, and it is stated more precisely as \cref{c:cs e infty}.

\begin{corollary*}
	The Cartan--Serre map is a quasi-isomorphism of $E_\infty$-algebras.
\end{corollary*}

\begin{remark*}
	In this introduction we have used the setting defined by cochains and products since it is more familiar, whereas in the rest of the text we use the more fundamental one defined by chains and coproducts.
\end{remark*}

\section*{Outline}

We recall the required notions from homological algebra and category theory in \cref{s:preliminaries}.
The necessary concepts from the theory of operads and props is reviewed in \cref{s:props}, including the definition of the prop $\M$.
\cref{s:action} contains our main contribution; an explicit natural $\M$-bialgebra structure on the chains of representable cubical sets and, from it, a natural $E_\infty$-coalgebra structure on the chains of cubical sets.
The comparison between simplicial and cubical chains is presented in \cref{s:comparison}, where we show that the Cartan--Serre map is a quasi-isomorphism respecting $E_\infty$-structures.
We close presenting some future work in \cref{s:future}.