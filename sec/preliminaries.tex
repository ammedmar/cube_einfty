
\section{Conventions and preliminaries} \label{s:preliminaries}

\subsection{Chain complexes}

Throughout this article $\k$ denotes a commutative and unital ring and we work over its associated symmetric monoidal category of chain complexes $(\Ch, \otimes, \k)$.
We denote the chain complex of $\k$-linear maps between two such by $\Hom(C, C^\prime)$ referring to the functor $\Hom(-, \k)$ as \textit{linear duality}.

\subsection{Kan extensions}

Given categories $\mathsf{B}$ and $\C$ we denote their associated \textit{functor category} by $\Fun(\mathsf{B}, \C)$.
Recall that a category is said to be \textit{small} if its objects and morphisms form sets.
We denote the category of small categories by $\Cat$.
A category is said to be \textit{cocomplete} if any functor to it from a small category has a colimit.
If $\mathsf{A}$ is small and $\mathsf{C}$ cocomplete, then the \textit{(left) Kan extension of $g$ along $f$} exists for any pair of functors $f$ and $g$ in the diagram below, and it is the initial object in $\Fun(\mathsf{B}, \mathsf{C})$ making
\begin{equation*}
\begin{tikzcd}[column sep=normal, row sep=normal]
\mathsf{A} \arrow[d, "f"'] \arrow[r, "g"] & \mathsf{C} \\
\mathsf{B} \arrow[dashed, ur, bend right] & \quad
\end{tikzcd}
\end{equation*}
commute.

\subsection{Simplicial sets}

For any non-negative integer $n$ denote by $[n]$ the category generated by the poset $\{0 \leq 1 \leq \dots \leq n\}$.
The \textit{simplex category} $\simplex$ is the category with set of objects $\big\{ [n] \big\}$ morphisms given by functors.
These are generated by the usual (simplicial) \textit{coface} and \textit{codegeneracy maps}
\[
\delta_i \colon [n-1] \to [n], \qquad \sigma_i \colon [n+1] \to [n]
\]
for $j \in \{0, \dots, n\}$.

The category of \textit{simplicial sets} is the functor category $\sSet = \Fun[\simplex^\op, \Set]$.
The \textit{standard $n$-simplex} is the simplicial set $\simplex^n = \simplex(-, [n])$, and the \textit{Yoneda embedding} $\Y \colon \simplex \to \sSet$ is the functor induced by $[n] \mapsto \cube^n$.
Additionally, for any simplicial set $X$ we write, as usual, $X_n$ instead of $X([n])$, and remark that
\[
X_n \cong \colim_{\simplex^n \to X} \simplex^n.
\]

If $X$ is such that $X_0$ is a singleton we say it is \textit{reduced}, and we denote the full subcategory of reduced simplicial sets by $\sSet^0$.

\subsection{Simplicial chains} \label{ss:simplicial sets}

For non-negative integers $m$ and $n$, let $\simplex_{\deg} \big( [m], [n] \big)$ be the subset of \textit{degenerate morphisms} in $\simplex \big( [m], [n] \big)$, i.e., those of the form $\sigma_i \circ \tau$ with $\tau$ any morphism in $\simplex \big( [m], [n+1] \big)$.
The functor of \textit{simplicial chains} $\schains \colon \sSet \to \Ch$ is the Kan extension along the Yoneda embedding of the functor $\simplex \to \Ch$ defined next.
To an object $[n]$ it assigns the chain complex having in degree $m$ the module
\[
\frac{\k\{\simplex \big( [m], [n] \big) \}}{\k\{\simplex_{\deg} \big( [m], [n] \big) \}},
\]
and differential
\[
\partial(\tau) = \sum (-1)^i \tau \circ \delta_i.
\]
To a morphism $\tau \colon [n] \to [n^\prime]$ it assigns the chain map
\[
\begin{tikzcd}[row sep=-3pt, column sep=normal,
/tikz/column 1/.append style={anchor=base east},
/tikz/column 2/.append style={anchor=base west}]
\schains(\simplex^n)_m \arrow[r] & \schains(\simplex^{n^\prime})_m \\
\big( [m] \to [n] \big) \arrow[r, mapsto] & \big( [m] \to [n] \xra{\tau} [n^\prime] \big).
\end{tikzcd}
\]
When no confusion arises from doing so we write $\chains$ instead of $\schains$ and refer to it simply as the functor of \textit{chains}.
We denote, as usual, the basis elements in $\chains(\simplex^n)_m$ by increasing tuples $[v_0, \dots, v_m]$ with $v_i \in \{0, \dots, n\}$.