
\section{Conventions and preliminaries} \label{s:preliminaries}

\subsection{Chain complexes}

Throughout this article $R$ denotes a commutative and unital ring and we work over its associated symmetric monoidal category of chain complexes $(\Ch, \otimes, R)$.
We denote the chain complex of $R$-linear maps between two such by $\Hom(C, C^\prime)$ referring to the functor $\Hom(-, R)$ as \textit{linear duality}.

\subsection{Kan extensions}

Given categories $\mathsf{B}$ and $\C$ we denote their associated \textit{functor category} by $\Fun(\mathsf{B}, \C)$.
Recall that a category is said to be \textit{small} if its objects and morphisms form sets.
We denote the category of small categories by $\Cat$.
A category is said to be \textit{cocomplete} if any functor to it from a small category has a colimit.
If $\mathsf{A}$ is small and $\mathsf{C}$ cocomplete, then the \textit{(left) Kan extension of $g$ along $f$} exists for any pair of functors $f$ and $g$ in the diagram below, and it is the initial object in $\Fun(\mathsf{B}, \mathsf{C})$ making
\begin{equation*}
\begin{tikzcd}[column sep=normal, row sep=normal]
\mathsf{A} \arrow[d, "f"'] \arrow[r, "g"] & \mathsf{C} \\ 
\mathsf{B} \arrow[dashed, ur, bend right] & \quad 
\end{tikzcd}
\end{equation*}
commute.
