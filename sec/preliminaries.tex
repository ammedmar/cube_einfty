
\section{Conventions and preliminaries} \label{s:preliminaries}

\subsection{Chain complexes}

Throughout this article $\k$ denotes a commutative and unital ring and we work over its associated closed symmetric monoidal category of differential (homological) graded $\k$-modules $(\Ch, \otimes, \k)$.
We refer to the objects and morphisms of this category as \textit{chain complexes} and \textit{chain maps} respectively. We denote by $\Hom(C, C^\prime)$ the chain complex of $\k$-linear maps between chain complexes $C$ and $C^\prime$, and refer to the functor $\Hom(-, \k)$ as \textit{linear duality}.

\subsection{Presheaves}

Given categories $\sB$ and $\sC$ we denote their associated \textit{functor category} by $\Fun(\sB, \sC)$.
Recall that a category is said to be \textit{small} if its objects and morphisms form sets.
We denote the Cartesian category of small categories by $\Cat$.
A category is said to be \textit{cocomplete} if any functor to it from a small category has a colimit.
If $\sA$ is small and $\sC$ cocomplete, then the (left) \textit{Kan extension of $g$ along $f$} exists for any pair of functors $f$ and $g$ in the diagram below, and it is the initial object in $\Fun(\sB, \sC)$ making
\begin{equation*}
\begin{tikzcd}[column sep=normal, row sep=normal]
\sA \arrow[d, "f"'] \arrow[r, "g"] & \sC \\
\sB \arrow[dashed, ur, bend right] & \quad
\end{tikzcd}
\end{equation*}
commute.
A Kan extension along the \textit{Yoneda embedding}, i.e., the functor
\[
\yoneda \colon \sA \to \Fun(\sA^\op, \Set)
\]
induced by the assignment
\[
a \mapsto \big( a^\prime \mapsto \sA(a^\prime, a) \big),
\]
is referred to as a \textit{Yoneda extension}.
We refer to the objects in $\Fun(\sA^\op, \Set)$ as \textit{presheaves on $\sA$} and to those in the image of the Yoneda embedding as \textit{representables}.
We remark that for any presheaf $P$ on $\sA$ and object $a$ of $\sA$
\[
P(a) \, \cong \colim_{\yoneda(a) \to P} \yoneda(a).
\]