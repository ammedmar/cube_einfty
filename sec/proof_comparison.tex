
\subsection{Proof of \cref{l:cartan-serre is e infinity}} \label{ss:comparison proof}

To prove \cref{l:cartan-serre is e infinity} we need to show the commutativity of the diagram
\[
\begin{tikzcd}
U(\Sl)(r) \otimes \gchains(\gcube^n) \arrow[r] \arrow[d, "\id\, \otimes\, \psi"'] &
\gchains(\gcube^n)^{\otimes r} \arrow[d, "\psi^{\otimes r}"] \\
U(\Sl)(r) \otimes \gchains(\gsimplex^n) \arrow[r] &
\gchains(\gsimplex^n)^{\otimes r}.
\end{tikzcd}
\]
By naturality, it suffices to show that
\begin{equation} \label{e:Cartan-Serre E-coalgebra map}
\psi^{\otimes r} \circ \Gamma\big( [0,1]^{\otimes n} \big) = \Gamma \circ \psi \big( [0,1]^{\otimes n} \big),
\end{equation}
where $\Gamma$ is represented by a surjection-like graph, a generator of $U(\Sl)$.

We will need the following characterization of the kernel of $\psi$.
\begin{lemma} \label{l:kernel of psi}
	For any basis element $x = x_1 \otimes \cdots \otimes x_n$ we have $\psi(x) = 0$ if and only if there exists $i < j$ such that $x_i = [0]$ and $x_j = [0,1]$. 
\end{lemma}

\begin{proof}
	This follows from noticing that for any point in $\gcube^n$ whose $i\th$ coordinate is $0$, its image under the Cartan-Serre map $\gcube^n \to \gsimplex^n$ defined in \cref{ss:the cartan-serre map} has $j\th$ coordinate equal to $0$ for every $j \geq i$.
\end{proof}

\begin{lemma}
	If $\Gamma =$ \counit \ or \coproduct \ then identity \eqref{e:Cartan-Serre E-coalgebra map} holds.
\end{lemma}

\begin{proof}
	Since vertices are sent to vertices the map $\psi$ is a map of augmented chain complexes, establishing the case $\Gamma =$ \counit \ .
	To study the compatibility of $\psi$ with \coproduct \, notice that for $n=0$ identity \eqref{e:Cartan-Serre E-coalgebra map} holds trivially.
	For $n > 0$, consider
	\[
	\Delta([0,1]^{\otimes n}) = \sum_{\lambda \in \Lambda} \pm \ x_1^{(\lambda)} \otimes \cdots \otimes x_n^{(\lambda)} \bm{\otimes} y_1^{(\lambda)} \otimes \cdots \otimes y_n^{(\lambda)},
	\]
	where $\Lambda$ parameterizes all choices of $x_i^{(\lambda)} \in \{[0], [0,1]\}$ and $y_i^{(\lambda)} \in \{[0,1], [1]\}$ such that
	\begin{align*}
	x_i^{(\lambda)} = [0]   & \iff y_i^{(\lambda)} = [0,1], \\
	x_i^{(\lambda)} = [0,1] & \iff y_i^{(\lambda)} = [1].
	\end{align*}
	By \cref{l:kernel of psi}, the summands above not sent to $0$ by $\psi \otimes \psi$ are those basis elements for which $x_i^{(\lambda)} = [0]$ implies $x_j^{(\lambda)} = [0]$ for all $i < j$.
	For any one such summand, its sign is positive and its image by $\psi \otimes \psi$ is $[0, \dots, k] \otimes [k, \dots, n]$ where $k+1 = \min \{i \mid x_i^{(\lambda)} = [0]\}$ or $k = n$ if this set is empty.
	The summands $[0, \dots, k] \otimes [k, \dots, n]$ are precisely those appearing when applying the Alexander-Whitney coproduct to $[0, \dots, n] = \psi \big( [0,1]^{\otimes n} \big)$.
\end{proof}

We will consider the basis of $\chains(\cube^n)$ as a poset in the following way.
For $n = 1$ we set $[0] < [0,1] < [1]$, and for $n > 1$ we have $(x_1 \otimes \cdots \otimes x_n) \leq (y_1 \otimes \cdots \otimes y_n)$ if $x_i \leq y_i$ for each $i \in \{1, \dots, n\}$. 

\begin{lemma}
	Let $\Delta^{r-1}$ be the $(r-1)\th$ iterated Serre diagonal.
	If
	\[	
	\Delta^{r-1} \big([0,1]^{\otimes n}\big) =
	\sum \pm \ x{(1)} \bm{\otimes} \cdots \bm{\otimes} x{(r)}
	\]
	with each $x(k) \in \chains(\cube^n)$ a basis element, then $x{(1)} \leq \cdots \leq x{(r)}$.
\end{lemma}

\begin{proof}
	For $r = 2$ we have for every $i \in \{1, \dots, n\}$ that
	\begin{align*}
	x(1)_i = [0]   & \iff x(2)_i = [0,1], \\
	x(1)_i = [0,1] & \iff x(2)_i = [1],
	\end{align*}
	and that neither $x(1)_i = [1]$ or $x(2)_i = [0]$ can occur, hence $x(1) \leq x(2)$.
	The claim for $r > 2$ follows from a straightforward induction argument.
\end{proof}

\begin{lemma}
	Let $x, y, z \in \chains(\cube^n)$ be basis elements.
	If $x, y \leq z$ then $(x \ast y) \leq z$.
\end{lemma}

\begin{proof}
	Recall that
	\begin{align*}
	(x_1 \otimes \cdots \otimes x_n) \ast (y_1 \otimes \cdots \otimes y_n) =
	(-1)^{|x|} \sum_{i=1}^n x_{<i}\, \epsilon(y_{<i}) \otimes x_i \ast y_i \otimes \epsilon(x_{>i}) \, y_{>i}.
	\end{align*}
	By assumption, for every $i$ we have $x_{<i} \leq z_{<i}$ and $y_{>i} \leq z_{>i}$, and if $x_i \ast y_i \neq 0$ then either $x_i = [1]$ or $y_i = [1]$ which implies $z_i = [1]$ as well.
\end{proof}

\begin{lemma}
	Let $x, y \in \chains(\cube^n)$ be basis elements.
	If $x \leq y$ then $\psi(x \ast y) = \psi(x) \ast \psi(y)$.
\end{lemma}

\begin{proof}
	If $\psi(x) = 0$ or $\psi(y) = 0$ we will show that
	\begin{equation} \label{e:zero for join}
	\psi \big( x_{<i}\, \epsilon(y_{<i}) \otimes x_i \ast y_i \otimes \epsilon(x_{>i}) \, y_{>i} \big) = 0.
	\end{equation}
	Assume $\psi(x) = 0$, that is, there exists a pair $p < q$ such that $x_p = [0]$ and $x_q = [0,1]$, then \eqref{e:zero for join} holds since:
	\begin{enumerate}
		\item If $i > q$, then $x_p$ and $x_q$ are part of $x_{<i}$.
		\item If $i = q$, then $x_q \ast y_q = 0$ for any $y_q$.
		\item If $i < q$, then $\varepsilon(x_{>i}) = 0$.
	\end{enumerate}
	Similarly, if there is a pair $p < q$ such that $y_p = [0]$ and $y_q = [0,1]$,  then \eqref{e:zero for join} holds since:
	\begin{enumerate}
		\item If $i < p$, then $y_p$ and $y_q$ are part of $y_{>i}$.
		\item If $i = p$ or, more generally, $y_i = [0]$, then $x_i = [0]$ and $x_i \ast y_i = 0$.
		\item If $i = q$ or, more generally, $y_i = [0,1]$, then $x_i \ast y_i = 0$ for any $x_i$.
		\item If $i > q$, then $\varepsilon(y_{<i}) = 0$.
		\item If $p < i < q$ and $y_i = [1]$ then either $x_i \ast x_j = 0$ or $x_i \ast x_j = [0,1]$, implying $(x \ast y)_p = [0]$ and $(x \ast y)_i = [0,1]$.
	\end{enumerate}
	Let us now assume that $\psi(x) \neq 0$ and $\psi(y) \neq 0$.
	In particular, $x = v_x \otimes [0] \otimes w_x$ and $y = v_y \otimes [0] \otimes w_y$ with $v_x, v_y$ having tensor factors in $\{[0,1], [1]\}$, and $w_x, w_y$ in $\{[0], [1]\}$, we also admit them to be the unit of the tensor product.
	Additionally, let $p_x$ and $p_y$ be the tensor position of the first $[0]$ in $x$ and $y$ respectively.
	To make sure all possible elements $x$ and $y$ have a factor $[0]$ in them we consider $\chains(\cube^n)$ as a subcomplex of $\chains(\cube^{n+1})$ via the inclusion that tensors on the right with $[0]$.
	By naturality, we do not loose generality making this assumption.
	Since $x \leq y$ we have that $p_x \leq p_y$.
	We claim that
	\[
	\psi(x \ast y) = \psi \big( x_{<p_x} \, \varepsilon(y_{<p_x}) \otimes x_{p_x} \ast y_{p_x} \otimes \varepsilon(x_{>p_x}) \, y_{>p_x} \big).
	\]
	To see this we notice that if $i < p_x$ then $x_i = [1]$ or $x_i = [0,1]$.
	If $x_i = [1]$, since $x_i \leq y_i$, it is impossible for $y_i = [0]$, the only case when $x_i \ast y_i \neq 0$.
	If $x_i = [0,1]$ then $x_i \ast y_i = 0$ for any $y_i$.
	If $i > p_x$, then either $x_i \ast y_i = 0$ or $x_i \ast y_i = [0,1]$.
	In the first case there is nothing to prove and in the second we notice that $(x \ast y)_{p_x} = [0]$ and $(x \ast y)_{} = [0,1]$ so $\psi(x \ast y) = 0$.
	Let $q_y$ the tensor position of the first occurrence of the tensor factor $[0,1]$ in $y$, setting it to $+\infty$, if not present.
	If $p_x > q_y$ then $x \ast y = 0$ since $\varepsilon(y_{<p_x}) = 0$, and if $p_x = q_y$ then $x \ast y = 0$ since $[0] \ast [0,1] = 0$.
	We now prove that in this case $\psi(x) \ast \psi(y) = 0$.
	Since $x_{q_y} = [0,1]$ since $x \leq y$ and $x_{p_x}$ is the first tensor factor equal to $[0]$.
	This implies that both $\psi(x)$ and $\psi(y)$ contain the vertex $q_{y} - 1$.
	Let us now assume that $p_x < q_y$.
	Then, using that $y_i = [1]$ for every $i < p_x$,
	\begin{align*}
	\psi(x \ast y) & =
	\psi \big( v_x \otimes x_{p_x} \ast y_{p_x} \otimes w_y \big) \\ & =
	\psi(x) \ast \psi(y)
	\end{align*}
	as claimed.
\end{proof}

This sequence of lemmas provides a proof of \cref{l:cartan-serre is e infinity} using the decomposition of any surjection-like graph not equal to \counit \ into pieces \coproduct \ and \product.

\begin{remark}
	One might ask for an example illustrating the need to restrict to $\Sl$.
	Notice for this end that $\psi([0] \otimes [0,1]) = 0$, but $([1] \otimes [1]) \ast ([0] \otimes [0,1]) = -([0,1] \otimes [0,1])$ which is mapped to $-[0,1,2]$ by $\psi$.
	This proves in particular that $\psi$ is not a morphisms of $\M$ algebras.
\end{remark}