
\section{Proof of \cref{l:main comparison lemma}} \label{s:comparison proof}

Let $\psi \colon \chains(\cube^n) \to \chains(\simplex^n)$ denote the composition
\begin{equation*}
\begin{tikzcd}[column sep=normal]
\chains(\cube^n) \arrow[r, "\chains(\T)"] &[-3pt]
\chains \big( \scube{n} \big) \arrow[r, "\chains(s)"] &[-3pt]
\chains \simplex^n,
\end{tikzcd}
\end{equation*}
which agrees with the chain map induced by the Cartan-Serre map $\gcube^n \to \gsimplex^n$.

We need to show the commutativity of the diagram
\begin{equation*}
\begin{tikzcd}
U(\MS)(r) \otimes \chains(\cube^n) \arrow[r] \arrow[d, "\id\, \otimes\, \psi"'] &
\cchains(\cube^n)^{\otimes r} \arrow[d, "\psi^{\otimes r}"] \\
U(\MS)(r) \otimes \cchains(\simplex^n) \arrow[r] &
\cchains(\simplex^n)^{\otimes r}.
\end{tikzcd}
\end{equation*}
By naturality, it suffices to show that
\begin{equation} \label{e:Cartan-Serre E-coalgebra map}
\psi^{\otimes r} \circ \Gamma\big( [0,1]^{\otimes n} \big) = \Gamma \circ \psi \big( [0,1]^{\otimes n} \big),
\end{equation}
where $\Gamma$ is represented by a surjection-like graph, a generator of $U(\MS)$.
Let us start by verifying that $\psi$ is a morphism of counital coalgebras.

\begin{lemma}
	If $\Gamma =$ \counit \ or \coproduct \ then identity \eqref{e:Cartan-Serre E-coalgebra map} holds.
\end{lemma}

\begin{proof}
	Since vertices are sent to vertices the map $\psi$ is a map of augmented chain complexes, establishing the case $\Gamma =$ \counit \ .
	To study the compatibility of $\psi$ with \coproduct \, notice that for $n=0$ identity \eqref{e:Cartan-Serre E-coalgebra map} holds trivially.
	For $n > 0$,
	\begin{equation*}
	\Delta([0,1]^{\otimes n}) = \sum_{\lambda \in \lambda} \pm \ x_1^{(\lambda)} \otimes \cdots \otimes x_n^{(\lambda)} \bm{\otimes} y_1^{(\lambda)} \otimes \cdots \otimes y_n^{(\lambda)},
	\end{equation*}
	where $\Lambda$ parameterizes all choices of $x_i^{(\lambda)} \in \{[0], [0,1]\}$ and $y_i^{(\lambda)} \in \{[0,1], [1]\}$ such that
	\begin{align*}
	x_i^{(\lambda)} = [0]   & \iff y_i^{(\lambda)} = [0,1], \\
	x_i^{(\lambda)} = [0,1] & \iff y_i^{(\lambda)} = [1].
	\end{align*}
	The only pairs that are not sent to $0$ by $\psi \otimes \psi$ are those for which $x_i^{(\lambda)} = [0]$ implies $x_j^{(\lambda)} = [0]$ for all $i < j$, for which the sign is positive, and their image is $[0,\dots,k] \otimes [k,\dots,n]$ where $k+1 = \min \{i \mid x_i^{(\lambda)} = [0]\}$ or $k = n$ if empty.
\end{proof}

We will consider the basis of $\cchains(\cube^n)$ as a poset defined as follows:
For $n = 1$ we set $[0] < [0,1] < [1]$, and for $n > 1$ we have $(x_1 \otimes \cdots \otimes x_n) \leq (y_1 \otimes \cdots \otimes y_n)$ if $x_i \leq y_i$ for each $i \in \{1, \dots, n\}$. 

\begin{lemma}
	Let
	\begin{equation*}
	\begin{tikzpicture}[scale=.35]
	\node at (9.5, 1){$\Gamma \, = $};
	\draw (11,.5)--(12,1.5)--(12,2.5);
	\draw (13,.5)--(12,1.5);
	\node at (11,0){$\scriptstyle 1$};
	\node at (12,0){$\scriptstyle \dots$};
	\node at (13,0){$\scriptstyle r$};
	\end{tikzpicture}
	\end{equation*}
	be the surjection-like graph representing the $(r-1)\th$ iterated Serre diagonal.
	If
	\begin{equation*}	
	\Gamma\big([0,1]^{\otimes n}\big) =
	\sum \pm \ x^{(1)} \bm{\otimes} \cdots \bm{\otimes} x^{(r)}
	\end{equation*}
	with each $x^{(k)} \in \cchains(\cube^n)$ a basis element, then $x^{(1)} \leq \cdots \leq x^{(r)}$.
\end{lemma}

\begin{proof}
	The claim follows from definition after inspecting
	\begin{equation*}
	\Delta([0]) = [0] \otimes [0], \quad \Delta([1]) = [1] \otimes [1], \quad \Delta([0, 1]) = [0] \otimes [0, 1] + [0, 1] \otimes [1].
	\end{equation*}
\end{proof}

\begin{lemma}
	Let $x, y, z \in \cchains(\cube^n)$ be basis elements.
	If $x, y \leq z$ then $(x \ast y) \leq z$.
\end{lemma}

\begin{proof}
	Recall that
	\begin{align*}
	(x_1 \otimes \cdots \otimes x_n) \ast (y_1 \otimes \cdots \otimes y_n) =
	(-1)^{|x|} \sum_{i=1}^n x_{<i}\, \epsilon(y_{<i}) \otimes x_i \ast y_i \otimes \epsilon(x_{>i}) \, y_{>i}.
	\end{align*}
	By assumption, for every $i$ we have $x_{<i} \leq z_{<i}$ and $y_{>i} \leq z_{>i}$, and if $x_i \ast y_i \neq 0$ then either $x_i = [1]$ or $y_i = [1]$ which implies $z_i = [1]$ as well.
\end{proof}

\begin{lemma}
	Let $x, y \in \cchains(\cube^n)$ be basis elements.
	If $x \leq y$ then $\psi(x \ast y) = \psi(x) \ast \psi(y)$.
\end{lemma}

\begin{proof}
	If $\psi(x) = 0$ or $\psi(y) = 0$ we will show that
	\begin{equation} \label{e:zero for join}
	\psi \big( x_{<i}\, \epsilon(y_{<i}) \otimes x_i \ast y_i \otimes \epsilon(x_{>i}) \, y_{>i} \big) = 0.
	\end{equation}
	Assume $\psi(x) = 0$, that is, there exists a pair $p < q$ such that $x_p = [0]$ and $x_q = [0,1]$, then \eqref{e:zero for join} holds since:
	\begin{itemize}
		\item If $i > q$, then $x_p$ and $x_q$ are part of $x_{<i}$.
		\item If $i = q$, then $x_q \ast y_q = 0$ for any $y_q$.
		\item If $i < q$, then $\varepsilon(x_{>i}) = 0$.
	\end{itemize}
	Similarly, if there is a pair $p < q$ such that $y_p = [0]$ and $y_q = [0,1]$,  then \eqref{e:zero for join} holds since:
	\begin{itemize}
		\item If $i < p$, then $y_p$ and $y_q$ are part of $y_{>i}$.
		\item If $i = p$ or, more generally, $y_i = [0]$, then $x_i = [0]$ and $x_i \ast y_i = 0$.
		\item If $i = q$ or, more generally, $y_i = [0,1]$, then $x_i \ast y_i = 0$ for any $x_i$.
		\item If $i > q$, then $\varepsilon(y_{<i}) = 0$.
		\item If $p < i < q$ and $y_i = [1]$ then either $x_i \ast x_j = 0$ or $x_i \ast x_j = [0,1]$, implying $(x \ast y)_p = [0]$ and $(x \ast y)_i = [0,1]$.
	\end{itemize}
	Let us now assume that $\psi(x) \neq 0$ and $\psi(y) \neq 0$.
	In particular, $x = v_x \otimes [0] \otimes w_x$ and $y = v_y \otimes [0] \otimes w_y$ with $v_x, v_y$ having tensor factors in $\{[0,1], [1]\}$, and $w_x, w_y$ in $\{[0,1], [1]\}$, we also admit them to be the unit of the tensor product.
	Additionally, let $p_x$ and $p_y$ be the tensor position of the first $[0]$ in $x$ and $y$ respectively.
	To make sure all possible elements $x$ and $y$ have a factor $[0]$ in them we consider $\chains(\cube^n)$ as a subcomplex of $\chains(\cube^{n+1})$ via the inclusion that tensors on the right with $[0]$.
	By naturality, we do not loose generality making this assumption.
	Since $x \leq y$ we have that $p_x \leq p_y$.
	We claim that
	\begin{equation*}
	\psi(x \ast y) = \psi \big( x_{<p_x} \, \varepsilon(y_{<p_x}) \otimes x_{p_x} \ast y_{p_x} \otimes \varepsilon(x_{>p_x}) \, y_{>p_x} \big).
	\end{equation*}
	To see this we notice that if $i < p_x$ then $x_i = [1]$ or $x_i = [0,1]$.
	If $x_i = [1]$, since $x_i \leq y_i$, it is impossible for $y_i = [0]$, the only case when $x_i \ast y_i \neq 0$.
	If $x_i = [0,1]$ then $x_i \ast y_i = 0$ for any $y_i$.
	If $i > p_x$, then either $x_i \ast y_i = 0$ or $x_i \ast y_i = [0,1]$.
	In the first case there is nothing to prove and in the second we notice that $(x \ast y)_{p_x} = [0]$ and $(x \ast y)_{} = [0,1]$ so $\psi(x \ast y) = 0$.
	Let $q_y$ the tensor position of the first occurrence of the tensor factor $[0,1]$ in $y$, setting it to $+\infty$, if not present.
	If $p_x > q_y$ then $x \ast y = 0$ since $\varepsilon(y_{<p_x}) = 0$, and if $p_x = q_y$ then $x \ast y = 0$ since $[0] \ast [0,1] = 0$.
	We now prove that in this case $\psi(x) \ast \psi(y) = 0$.
	Since $x_{q_y} = [0,1]$ since $x \leq y$ and $x_{p_x}$ is the first tensor factor equal to $[0]$.
	This implies that both$\psi(x)$ and $\psi(y)$ contain the vertex $q_{y} - 1$.
	Let us now assume that $p_x < q_y$.
	Then, using that $y_i = [1]$ for every $i < p_x$,
	\begin{align*}
	\psi(x \ast y) & =
	\psi \big( v_x \otimes x_{p_x} \ast y_{p_x} \otimes w_y \big) \\ & =
	\psi(x) \ast \psi(y).
	\end{align*}
\end{proof}







%We use triples $(F_0, F_{01}, F_1)$ of subsets of $\{0, \dots, n\}$ to represent faces $F$ of $\gcube^n$ with $F_\epsilon = \{i \mid \forall x \in F, \, x_i = \epsilon\}$ for $\epsilon \in \{0,1\}$ and $F_{01} = \{0, \dots, n\} \setminus F_0 \cup F_1$.

%Given a subset $U = \{u_1 < \cdots < u_q\}$ of $\{0, \dots, n\}$ we write $d_U$ for $d_{u_1} \! \cdots \, d_{u_q}$.
%In this notation, any face of $\gsimplex^n$ can be written as $d_U [0, \dots, n]$ for some $U$.

%With this notation we have the following description of the chain map $\psi$.

%\begin{lemma}
%	On basis elements $\psi \colon \cchains(\cube^n) \to \chains(\triangle^n)$ is given by
%	\begin{equation*}
%	(F_0, F_{01}, F_1) \mapsto
%	\begin{cases}
%	d_U [0, \dots, n] & \text{ if } F_{01} \cap \{i \mid i > \min(F_0)\} = \emptyset, \\
%	0 & \text{ otherwise},
%	\end{cases}
%	\end{equation*}
%	where $U = \{i-1 \mid i \in F_1 \text{ or } i > \min(F_0)\}$ with the convention $\min(\emptyset) = +\infty$.
%\end{lemma}
%
%\begin{proof}
%	We assume $n > 0$ since otherwise there is nothing to prove.
%	Consider a face $(F_0, F_{01}, F_1)$ and let $M \subseteq \{0, \dots, n\}$ be empty if $F_0$ is empty or be characterized by $i > \min (F_0)$ otherwise.
%	Notice in \eqref{e:cartan-serre CW map} that if $x_i = 0$ then $y_j = 0$ for every $j > i$.
%	Therefore, the image of $(F_0, F_{01}, F_1)$ in $\simplex^n$ is a face of $d_U[0, \dots, n]$ where $U = \{i-1 \mid i \in M\}$ or, more explicitly, $[0, \dots, n]$ if $F_0$ is empty and $[0, \dots, \min(F_0)]$ otherwise.
%	In particular, $S(F_0, F_{01}, F_1)$ is non-zero only if $M \cap F_{01} = \emptyset$, and we can assume without loss of generality that $F_0 = \emptyset$ or $F_0 = \{n\}$.
%	
%	Notice from \eqref{e:cartan-serre CW map} that $x_i = 1$ if and only if $y_{i-1} = 0$, so $S(\emptyset, F_{01}, F_1) = d_{U} [0, \dots, n]$ where $U = \{i-1 \mid i \in F_1\}$.
%\end{proof}

%\begin{definition}
%	A basis element $x_1 \otimes \cdots \otimes x_n \in \cchains(\cube^n)$ is said to be  \textit{vertex-ordered} if $x_i = [0]$ and $x_j = [1]$ imply $i < j$.
%\end{definition}


%\begin{lemma}
%	Let
%	\begin{equation*}
%	\begin{tikzpicture}[scale=.35]
%	\node at (9.5, 1){$\Gamma \, = $};
%	\draw (11,.5)--(12,1.5)--(12,2.5);
%	\draw (13,.5)--(12,1.5);
%	\node at (11,0){$\scriptstyle 1$};
%	\node at (12,0){$\scriptstyle \dots$};
%	\node at (13,0){$\scriptstyle r$};
%	\end{tikzpicture}
%	\end{equation*}
%	be the surjection-like graph representing the $(r-1)\th$ iterated Serre diagonal, and $x_1 \otimes \cdots \otimes x_n \in \cchains(\cube^n)$ a basis element.
%	If
%	\begin{equation*}	
%	\Gamma \big( x_1 \otimes \cdots \otimes x_n \big) =
%	\sum \pm \left( x_1^{(1)} \otimes \cdots \otimes x_n^{(1)} \right)
%	\otimes \cdots \otimes
%	\left( x_1^{(r)} \otimes \cdots \otimes x_n^{(r)} \right)
%	\end{equation*}
%	with each $x_i^{(k)}$ a basis element of $\cchains(\cube^1)$, then
%	\begin{enumerate}
%		\item For every $i \in \{1, \dots, n\}$ the element $x_i^{(1)} \otimes \cdots \otimes x_i^{(r)}$ is vertex-ordered.
%	
%	 	\item For every $k \in \{1, \dots, r\}$ if $x_1 \otimes \cdots \otimes x_n$ is vertex-ordered then $x_1^{(k)} \otimes \cdots \otimes x_n^{(k)}$ is vertex-ordered.
%	\end{enumerate}
%\end{lemma}