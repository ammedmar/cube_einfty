
\section{Operads and props} \label{s:operads and props}

Let $\S$ be the category whose objects are the natural numbers and whose set of morphisms between $m$ and $n$ is empty if $m \neq n$ and is otherwise the symmetric group $\S_n$.
A \textit{left $\S$-module} (resp. \textit{right} $\S$-\textit{module} or $\S$-\textit{bimodule}) is a functor from $\S$ (resp. $\S^\op$ or $\S \times \S^\op$) to $\Ch$.
In this paper we prioritize left module structures over their right counterparts. As usual, taking inverses makes both perspectives equivalent.
We respectively denote by $\smod$ and $\sbimod$ the categories of left $\S$-modules and of $\S$-bimodules with morphisms given by natural transformations.

The collections of group homomorphism $\S_n \to \S_n \times \S_1$ for $n \geq 0$ induces a forgetful functor $U \colon \sbimod \to \smod$.
The similarly defined forgetful functor to right $\S$-modules will will not be used.

Given a chain complexes $C$ define $\End^C$, $\End_C$ and $\End_C^C$ with their natural left $\S$-module, right $\S$-module and $\S$-bimodule structures, by
\begin{align*}
\End^C(r) &= \Hom(C, C^{\otimes r}),
& \End_C(r) &= \Hom(C^{\otimes r}, C),
& \End^C_C(r, s) &= \Hom(C^{\otimes r}, C^{\otimes s}),
\end{align*}
respectively.
The forgetful functor $U$ sends $\End^C_C$ to $\End^C$.

We can define \textit{operads} and \textit{props} by enriching $\S$-modules and \mbox{$\S$-bimodules} with certain composition structures.
For a complete presentation of these concepts we refer to Definition 11 and 54 of \cite{Markl08}.
Intuitively, operads and props can be understood by abstracting the composition structure naturally present in the left $\S$-module $\End^C$ (or right $\S$-module $\End_C$), naturally an operad, and the $\S$-bimodule $\End^C_C$, naturally a prop.
We remark that the structure on a prop $\P$ restricts to an operad structure on $U(\P)$.

The free prop generated by an $\S$-bimodule can be constructed using isomorphism classes of directed graphs with no directed loops equipped with the following labeling structure.
We think of each directed edge as built from two compatibly directed half-edges. For each vertex $v$ of a directed graph $\Gamma$, we have the sets $in(v)$ and $out(v)$ of half-edges that are respectively incoming to and outgoing from $v$. Half-edges that do not belong to $in(v)$ or $out(v)$ for any $v$ are divided into the disjoint sets $in(\Gamma)$ and $out(\Gamma)$ of incoming and outgoing external half-edges.
For any positive integer $n$ let $\overline{n} = \{1,\dots,n\}$ and set $\overline{0} = \emptyset$.
The labeling is given by bijections  
\begin{equation*}
\overline{m} \to in(\Gamma), \qquad
\overline{n} \to out(\Gamma),
\end{equation*}
and, for every vertex $v$,
\begin{equation*}
\overline{p} \to in(v), \qquad
\overline{q} \to out(v).
\end{equation*}
We refer to the isomorphism classes of such labeled directed graphs as $(n,m)$\textit{-graphs} and use graphs immersed in the plane to represent them, please see Figure \ref{f:immersion}.
The set of $(m,n)$-graphs has an action of $\S_n \times \S_m^{\op}$ given by relabeling of of incoming and outgoing external half-edges.
The collection $\{\G(m,n)\}_{m,n \geq 0}$ is equipped with a composition structure induced by grafting and disjoint union.
The free prop generated by an $\S$-bimodule is constructed using $M$ to decorate the vertices of $(m,n)$-graphs.
Please consult \cite{Markl08} or \cite{Fresse2010props} for a complete presentation.
The free operad generated by an $\S$-module is constructed similarly using only $(1,m)$-graphs.

The free prop functor can be combined with the free $\S$-bimodule functor
\begin{equation*}
\big\{ S(m,n) \big\}_{m,n \geq 0} \, \mapsto \, \big\{ \S_n \times S(m,n) \times \S_m \big\}_{m,n \geq 0}
\end{equation*}
to define the concept of prop presentation.
For more details please consult 2.4. in \cite{Medina20prop1}, where the following appears as Definition 3.1.
\begin{definition}
	Let $\M$ be the quotient of the free prop generated by
	\begin{equation*}
	\counit\,, \hspace*{.6cm} \coproduct\,, \hspace*{.6cm} \product,
	\end{equation*}
	by the prop ideal generated by the relations
	\begin{equation*}
	\productcounit, \hspace*{.6cm} \leftcounitality\,, \hspace*{.6cm} \rightcounitality\,,
	\end{equation*}
	where the first two generators are in degree $0$ and the third in degree $1$, with boundaries
	\begin{equation*}
	\partial\ \counit = 0,
	\hspace*{.6cm}
	\partial\ \coproduct = 0,
	\hspace*{.6cm}
	\partial\ \product = \ \boundary\,.
	\end{equation*}
\end{definition}

An $\S$-module $M$ is said to be $E_\infty$ if for each $r$ the chain complex $M(r)$ is an algebraic model for the universal bundle $E\S_r$, i.e., it is free as an $\S_r$-module and its homology is that of a point.
An operad is said to be $E_\infty$ if its underlying $\S$-modules is $E_\infty$, and following Boardman-Vogt \cite{boardman2006homotopy}, a prop $\P$ is said to be $E_\infty$ if $U(\P)$ is an $E_\infty$ operad.
The following appears as Theorem 3.3. in \cite{Medina20prop1}.

\begin{figure}
	\input{immersion}
	\caption{Immersed graphs represent labeled directed graphs with the direction implicitly given from top to bottom and the labeling from left to right.}
	\label{f:immersion}
\end{figure}

\begin{proposition}
	The prop $\M$ is an $E_\infty$-prop.
\end{proposition}