
\section{Cubical topology} \label{s:cubical}

\subsection{Cubical sets}

The \textit{cube category} $\cube$ is the subcategory of $\Cat$ with objects $2^n = (0 \to 1)^n$.
We refer to \cite{grandis2003cubicalsite} for a more leisure exposition and for variations on this definition.
The morphisms of the cube category are generated by the \textit{coface} and \textit{codegeneracy functors} defined by
\begin{align*}
\delta_i^\varepsilon & = \mathrm{id}_{2^{i-1}} \times \delta^\varepsilon \times \mathrm{id}_{2^{n-1-i}} \colon 2^{n-1} \to 2^n, \\
\sigma_i & = \mathrm{id}_{2^{i-1}} \times \, \sigma \times \mathrm{id}_{2^{n-i}} \quad \colon 2^{n} \to 2^{n-1},
\end{align*}
where $\varepsilon \in \{0,1\}$ and the functors
\begin{equation*}
\begin{tikzcd} [column sep=16pt]
2^0 \arrow[r, bend left, "\delta^0"] \arrow[r, bend right, "\delta^1"'] & 2^1 \arrow[r, "\sigma"] & 2^0
\end{tikzcd}
\end{equation*}
are defined by
\begin{equation*}
\delta^0(0) = 0, \qquad \delta^1(0) = 1, \qquad \sigma(0) = \sigma(1) = 0.
\end{equation*}

We denote by $\cube_{\deg}(2^m, 2^n)$ the subset of morphism in $\cube(2^m, 2^n)$ of the form $\sigma_i \circ \tau$ with $\tau \in \cube(2^m, 2^{n+1})$.

The category of \textit{cubical sets} is the functor category $\cSet = \Fun(\cube^\op, \Set)$.
The \textit{standard $n$-cube} is the cubical set $\cube^n = \cube(-, 2^n)$, and the \textit{Yoneda embedding} $\cube \to \cSet$ is the functor induced by $2^n \mapsto \cube^n$.

For any cubical set $X$ we have
\begin{equation*}
X_n \cong \colim_{\cube^n \to X} \cube^n,
\end{equation*}
and we denote $X(\delta_i^\varepsilon)$ and $X(\sigma_i)$ by $d_i^\varepsilon$ and $s_i$.

\subsection{Cubical chains}

The functor of \textit{chains} $\chains \colon \cSet \to \Ch$ is the Kan extension along the Yoneda embedding of the functor $\cube \to \Ch$ assigning to an object $2^n$ the chain complex having in degree $m$ the $R$ module
\begin{equation*}
\frac{R\{\cube(2^m, 2^n)\}}{R\{\cube_{\deg}(2^m, 2^n)\}}
\end{equation*}
and boundary map defined by
\begin{equation*}
\partial (\id_{2^n}) = \sum_{i=1}^{n} \ (-1)^i \
\big(\delta_i^1 - \delta_i^0 \big),
\end{equation*}
and to a morphism $\tau \colon 2^n \to 2^{n^\prime}$ the chain map
\begin{equation*}
\begin{tikzcd}[row sep=-3pt, column sep=normal,
/tikz/column 1/.append style={anchor=base east},
/tikz/column 2/.append style={anchor=base west}]
\chains_m(\cube^n) \arrow[r] &  \chains_m(\cube^{n^\prime}) \\
(2^m \to 2^n) \arrow[r, mapsto] & (2^m \to 2^n \stackrel{\tau}{\to} 2^{n^\prime}).
\end{tikzcd}
\end{equation*}
Explicitly,
\begin{equation*}
\chains(X) = \bigoplus_{n \geq 0} \chains(\cube^n) \otimes R[X_n] \ \Big/ \sim
\end{equation*}
where $(2^m \to 2^n) \circ \delta_i^\varepsilon \otimes x \sim (2^m \to 2^n) \otimes d_i^\varepsilon(x)$.

\subsection{Cubical singular complex}

Consider the topological $n$-cube
\begin{equation*}
\gcube^{n} = \{(x_1, \dots, x_n) \mid x_i \in [0,1]\}.
\end{equation*}
The assignment $2^n \to \gcube^n$ defines a functor $\cube \to \Top$ whose Kan extension is known as \textit{geometric realization}.
It has a right adjoint $\cSing \colon \Top \to \cSet$ given by
\begin{equation*}
Z \to \Big(2^n \to \Top(\gcube^n, X)\Big)
\end{equation*}
and referred to as the \textit{cubical singular complex} of the topological space $Z$.
The chain complex $\chains(\cSing Z)$ is referred to as the \textit{cubical singular chains} of $Z$.
