
\section{An $E_\infty$ structure on cubical chains} \label{s:action}

In this section we construct a natural $\M$-bialgebra structure on the chains of standard cubes $\chains(\cube^n)$.
These are determined by three natural linear maps satisfying the relations defining $\mathcal M$.
A Kan extension argument then provides the chains of any cubical set with a natural $U(\M)$ coalgebra structure.

\subsection{Cellular chains}

The chain complex $\chains(\cube^n)$ is by definition equal to $\chains(\cube^1)^{\otimes n}$, which is isomorphic to the chain complex $C(\gcube)^{\otimes n}$, where $C(\gcube)$ is the complex of cellular chains on the standard interval.
We will use the notation $x_1 \otimes \cdots \otimes x_n$ with $x_i \in \{[0], [0,1], [1]\}$ for the elements in its basis, and we remark that this chain complex is also isomorphic to $\gchains(\gcube^n)$, the cellular chains of the geometric $n$-cube with its standard CW structure.

\subsection{Counit, coproduct and product}

For $n \in \mathbb{N}$, define: \vspace*{5pt} \\
(1) The \textit{counit} $\epsilon \in \Hom(\chains(\square^n), \Z)$ known as the \textit{augmentation} by
\begin{equation*}
\epsilon \left( x_1 \otimes \cdots \otimes x_d \right) = \epsilon(x_1) \, \cdots \, \epsilon(x_n),
\end{equation*}
where
\begin{equation*}
\epsilon([0]) = \epsilon([1]) = 1, \qquad \epsilon([0, 1]) = 0.
\end{equation*} \vspace*{-6pt} \\
(2) The \textit{coproduct} $\Delta \in \Hom \left( \chains(\square^n), \chains(\square^n)^{\otimes 2} \right)$ known as the \textit{Serre coproduct} by
\begin{equation*}	
\Delta (x_1 \otimes \cdots \otimes x_n) = 	
\sum \pm \left( x_1^{(1)} \otimes \cdots \otimes x_n^{(1)} \right) \otimes 	
\left( x_1^{(2)} \otimes \cdots \otimes x_n^{(2)} \right),	
\end{equation*}	
where the sign is determined using the Koszul convention, and we are using Sweedler's notation
\begin{equation*}	
\Delta(x_i) = \sum x_i^{(1)} \otimes x_i^{(2)}
\end{equation*}
for the chain map $\Delta \colon \chains(\square^1) \to \chains(\square^1)^{\otimes 2}$ defined by
\begin{equation*}
\Delta([0]) = [0] \otimes [0], \quad \Delta([1]) = [1] \otimes [1], \quad \Delta([0, 1]) = [0] \otimes [0, 1] + [0, 1] \otimes [1].
\end{equation*}
Using that $\chains(\square^n) = \chains(\square^1)^{\otimes n}$, $\Delta$ is the composition
\begin{equation*}
\begin{tikzcd}
\chains(\square^1)^{\otimes n} \arrow[r, "\Delta^{\otimes n}"] &[3pt] \left( \chains(\square^1)^{\otimes 2}  \right)^{\otimes n} \arrow[r, "sh"] &[-5pt] \left( \chains(\square^1)^{\otimes n} \right)^{\otimes 2}
\end{tikzcd}
\end{equation*}
where $sh$ is the shuffle map that places tensor factors in odd position first. \vspace*{5pt} \\
(3) The \textit{product} $\ast \in \Hom(\chains(\square^n)^{\otimes 2}, \chains(\square^n))$ by
\begin{align*}
(x_1 \otimes \cdots \otimes x_n) \ast (y_1 \otimes \cdots \otimes y_n) =
(-1)^{|x|} \sum_{i=1}^n x_{<i}\, \epsilon(y_{<i}) \otimes x_i \ast y_i \otimes \epsilon(x_{>i}) \, y_{>i},
\end{align*}
where
\begin{align*}
x_{<i} & = x_1 \otimes \cdots \otimes x_{i-1}, &
y_{<i} & = y_1 \otimes \cdots \otimes y_{i-1}, \\
x_{>i} & = x_{i+1} \otimes \cdots \otimes x_n, & 
y_{>i} & = y_{i+1} \otimes \cdots \otimes y_n,
\end{align*}
with the convention
\begin{equation*}
x_{<1} = y_{<1} = x_{>n} = y_{>n} = 1 \in \Z,
\end{equation*}
and the only non-zero values of $x_i \ast y_i$ are
\begin{equation*}
\ast([0] \otimes [1]) = [0, 1], \qquad  \ast([1] \otimes [0]) = -[0, 1].
\end{equation*}

\subsection{Main construction} \label{ss:main construction}

The following is the main technical result of this paper.
Its proof is given in \cref{ss:proof action}.

\begin{lemma} \label{l:cubical chain bialgebra}
	The assignment
	\begin{equation*}
	\counit \mapsto \epsilon, \quad \coproduct \mapsto \Delta, \quad \product \mapsto \ast,
	\end{equation*}
	induces natural $\mathcal M$-bialgebra structure on $\chains(\square^n)$ for every $n \in \mathbb{N}$, or, equivalently, a functor $\cube \to \biAlg_{\M}$.
\end{lemma}

The category of bialgebras over a prop is in general not cocomplete, but those of algebras and coalgebras over operads are.
So we have the following result, the main contribution of this paper.

\begin{theorem} \label{t:lift to e infinity coalgebras}
	Composing the functor defined in \cref{l:cubical chain bialgebra} with the forgetful functor $\biAlg_{\M} \to \coAlg_{U(\M)}$ defines a functor $\cube \to \coAlg_{U(\M)}$ whose Kan extension endows the chains of a cubical set with a natural $E_\infty$ coalgebra structure extending the Serre coproduct.	
\end{theorem}

By linear duality, the same argument can be used to define a natural $E_\infty$ algebra structure on cubical cochains extending the Serre product.