\documentclass{amsart}
\usepackage{amsmath, amssymb}
\usepackage{tikz-cd}
% For Serre's quotes:
\usepackage{csquotes}
% Since boldface for geometric cubes and simplices:
\usepackage{bm}
\usepackage{mathbbol}
\DeclareSymbolFontAlphabet{\amsmathbb}{AMSb}
% hyperref
\usepackage[bookmarks=true, linktocpage=true,
bookmarksnumbered=true, breaklinks=true,
pdfstartview=FitH, hyperfigures=false,
plainpages=false, naturalnames=true,
colorlinks=true, pagebackref=true,
pdfpagelabels]{hyperref}
\hypersetup{
	colorlinks,
	citecolor=blue,
	filecolor=blue,
	linkcolor=blue,
	urlcolor=blue}
\usepackage[capitalize, noabbrev]{cleveref}

% layout
\setlength{\textwidth}{\paperwidth}
\addtolength{\textwidth}{-1.85in}
\setlength{\textheight}{\paperheight}
\addtolength{\textheight}{-2in}
\calclayout

% Updating to MSC2020
\makeatletter
\@namedef{subjclassname@2020}{%
	\textup{2020} Mathematics Subject Classification}
\makeatother

% elements
\newcommand{\id}{\mathrm{id}}
\renewcommand{\P}{\mathcal{P}}
\renewcommand{\O}{\mathcal{O}}
\renewcommand{\S}{\amsmathbb{S}}
\newcommand{\A}{\mathcal{A}s}
\newcommand{\M}{\mathcal{M}}
\newcommand{\MS}{\mathcal{MS}}
\newcommand{\scube}[1]{(\triangle^1)^{\times #1}}
\newcommand{\angles}[1]{\langle #1 \rangle}

% sets
\newcommand{\Z}{\amsmathbb{Z}}
\newcommand{\R}{\amsmathbb{R}}
\newcommand{\End}{\mathrm{End}}
\newcommand{\Hom}{\mathrm{Hom}}
\newcommand{\Bij}{\mathfrak{Bij}}
\newcommand{\G}{\mathfrak{G}}
\newcommand{\gcube}{\bm{\mathbb{I}}}
\newcommand{\gsimplex}{\mathbb{\Delta}}
\newcommand{\zero}{[\,\bar{0}\,]}
\newcommand{\one}{[\,\bar{1}\,]}
\newcommand{\interval}{[\,\bar 0, \bar 1\,]}

% categories
\newcommand{\Cat}{\mathsf{Cat}}
\newcommand{\C}{\mathsf{C}}
\newcommand{\Fun}{\mathsf{Fun}}
\newcommand{\Ch}{\mathsf{Ch}}
\newcommand{\Alg}{\mathsf{Alg}}
\newcommand{\coAlg}{\mathsf{coAlg}}
\newcommand{\biAlg}{\mathsf{biAlg}}
\newcommand{\cube}{\square}
\newcommand{\simplex}{\triangle}
\newcommand{\Set}{\mathsf{Set}}
\newcommand{\sSet}{\mathsf{sSet}}
\newcommand{\cSet}{\mathsf{cSet}}
\newcommand{\Nec}{\mathsf{Nec}}
\newcommand{\nSet}{\mathsf{nSet}}
\newcommand{\Mon}{\mathsf{Mon}}
\newcommand{\smod}{\mathsf{Mod}_{\S}}
\newcommand{\sbimod}{\mathsf{biMod}_{\S}}
\newcommand{\operads}{\mathsf{Oper}}
\newcommand{\props}{\mathsf{Prop}}

% functions and functors
\DeclareMathOperator*{\tensor}{\otimes}
\newcommand{\Y}{\mathcal{Y}}
\DeclareMathOperator{\T}{\mathcal{T}}
\DeclareMathOperator{\U}{\mathcal{U}}
\DeclareMathOperator{\chains}{N}
\DeclareMathOperator{\cochains}{N}
\DeclareMathOperator{\cchains}{N}
\DeclareMathOperator{\ccochains}{N}
\newcommand{\op}{\mathrm{op}}
\DeclareMathOperator*{\colim}{colim}
\newcommand{\cobar}{\Omega}
\newcommand{\gcobar}{\mathbb{\Omega}}
\newcommand{\norm}[1]{\vert #1 \vert}
\DeclareMathOperator{\Sing}{Sing}

% environments
\newtheorem{theorem}{Theorem}
\newtheorem{proposition}[theorem]{Proposition}
\newtheorem{lemma}[theorem]{Lemma}
\newtheorem{corollary}[theorem]{Corollary}
\theoremstyle{definition}
\newtheorem{definition}[theorem]{Definition}
\newtheorem{example}[theorem]{Example}
\newtheorem{remark}[theorem]{Remark}
\newtheorem*{notation}{Notation}

% other
\newcommand{\anibal}[1]{\textcolor{blue}{\underline{Anibal}: #1}}
\renewcommand{\th}{^\mathrm{th}}

% drawings
\include{figures}

\begin{document}
\title{A combinatorial $E_\infty$ algebra structure on cubical cochains}
\author{Anibal M. Medina-Mardones}
\address{Max Plank Institute for Mathematics, Bonn, Germany}
\email{ammedmar@mpim-bonn.mpg.de}
\address{Department of Mathematics, University of Notre Dame, Notre Dame, IN, USA}
\email{amedinam@nd.edu}

\keywords{...}
\subjclass[2020]{...}

\begin{abstract}
	
\end{abstract} 

\maketitle
\tableofcontents
\section{Introduction} \label{s:introduction}

Instead of simplices, in his groundbreaking work on fibered spaces Serre considered cubes as the basic shapes used to define cohomology, stating that:
\begin{displaycquote}[p.431]{serre1951homologie}
	Il est en effet evident que ces derniers se pretent mieux que les simplexes a l'etude des produits directs, et, a fortiori, des espaces fibres qui en sont la generalisation.
\end{displaycquote}
Cubical sets, a model for the homotopy category, were considered by Kan \cite{kan1955abstract, kan1956abstract} before introducing simplicial sets, they are central to nonabelian algebraic topology \cite{brown2011nonabelian}, and have become important in Voevodsky's program for univalent foundations and homotopy type theory \cite{kapulkin2020straightening, mortberg2017cubical}.
Other areas that highlight the relevance of cubical methods are applied topology, where cubical complexes are ubiquitous in the study of images \cite{tomasz2004computational}, condensed matter physics, where models on cubical lattices are central \cite{baxter1985exactlysolved}, and geometric group theory \cite{gromov1987hyperbolic}, where fundamental results have been obtained considering actions on certain cube complexes characterized combinatorially \cite{agol2013haken}.

Cubical cochains are equipped with the \textit{Serre algebra structure}, a lift to the cochain level of the graded ring structure in cohomology.
Using an acyclic carrier argument it can be shown that this product is commutative up to coherent homotopies in a non-canonical way.
The study of such objects, referred to as $E_\infty$-algebras, has a long history, where (co)homology operations \cite{steenrod1962cohomology, may1970general}, the recognition of infinite loop spaces \cite{boardman1973homotopy, may1972geometry} and complete algebraic models of the $p$-adic homotopy category \cite{mandell2001padic} are key milestones.
The goal of this work is to introduce a description of an explicit $E_\infty$-algebra structure naturally extending the Serre algebra structure, and relate it to one on simplicial cochains extending the Alexander--Whitney algebra structure.

We use the combinatorial model of the $E_\infty$-operad $\UM$ obtained from the finitely presented prop $\M$ introduced in \cite{medina2020prop1}.
The resulting $\UM$-algebra structure on cubical cochains is induced from a natural $\M$-bialgebra structure on the chains of representable cubical sets, which is determined by only three linear maps.
To our knowledge, this is the first effective construction of an $E_\infty$-algebra structure on cubical cochains.
Non-constructively, this result could be obtained using a lifting argument based on the cofibrancy of the reduced version of the operad $\UM$ in the model category of operads \cite{hinich1997homological, berger2003modelcategory}, but this existence statement is not very useful in concrete situations.
To illustrate the advantages of an effective construction let us consider a prime $p$.
The mod $p$ cohomology of spaces is equipped with natural stable endomorphisms, known as Steenrod operations \cite{steenrod1962cohomology}.
Following an operadic viewpoint developed by May \cite{may1970general}, in \cite{medina2021may_st} we exhibited integral elements in $\UM$ representing Steenrod operations on the mod~$p$ homology of $\UM$-algebras.
Since, as proven in this article, the cochains of a cubical set are equipped with a $\UM$-algebra structure, we obtain natural cochain level multioperations for cubical sets representing Steenrod operation at every $p$.
This cubical cup-$(p,i)$ products are explicit enough to have been implemented in the open source computer algebra system \href{https://comch.readthedocs.io/en/latest/}{\texttt{ComCH}} \cite{medina2021comch}.

We now turn to the comparison between cubical and simplicial cochains.
In \cite[p. 442]{serre1951homologie}, Serre described for any topological space $\fZ$ a natural quasi-isomorphism
\begin{equation} \label{e:cs on singular cochains}
	\cScochains(\fZ) \to \sScochains(\fZ)
\end{equation}
between its cubical and simplicial singular cochains, stating this to be a quasi-isomorphism of algebras with respect to the usual structures.
We will consider a well known Quillen equivalence
\[
\begin{tikzcd}[column sep=0]
	\sSet \arrow[rr, "\cubify"', bend right] & \perp & \arrow[ll,"\triangulate"', bend right] \cSet
\end{tikzcd}
\]
between simplicial and cubical sets, and construct a natural chain map
\begin{equation} \label{e:intro main map}
	\ccochains(\cubify Y) \to \scochains(Y)
\end{equation}
for every simplicial set $Y$.
In \cite{medina2020prop1}, a natural $\UM$-algebra structure extending the Alexander--Whitney coalgebra structure was constructed on simplicial sets.
With respect to it and the one defined here for cubical sets we have the following results after passing to a sub-$E_\infty$-operad of $\UM$.

\begin{theorem*}
	The map presented in \cref{e:intro main map} is a quasi-isomorphism of $E_\infty$-algebras.
\end{theorem*}

From this result, stated as \cref{t:main comparison}, we deduce the following two.
The first one concerns the triangulation functor $\triangulate$ and it is stated more precisely as \cref{c:zig-zag}.

\begin{corollary*}
	There is a natural zig-zag of $E_\infty$-algebra quasi-isomorphisms between the cochains of a cubical set and those of its triangulation.
\end{corollary*}

The next one concerns the map presented in \cref{e:cs on singular cochains}, relating the cubical and simplicial singular cochains of a space, and it is stated more precisely as \cref{c:cs e infty}.

\begin{corollary*}
	The Cartan--Serre map is a quasi-isomorphism of $E_\infty$-algebras.
\end{corollary*}

\begin{remark*}
	In this introduction we have used the setting defined by cochains and products since it is more familiar, whereas in the rest of the text we use the more fundamental one defined by chains and coproducts.
\end{remark*}

\section*{Outline}

We recall the required notions from homological algebra and category theory in \cref{s:preliminaries}.
The necessary concepts from the theory of operads and props is reviewed in \cref{s:props}, including the definition of the prop $\M$.
\cref{s:action} contains our main contribution; an explicit natural $\M$-bialgebra structure on the chains of representable cubical sets and, from it, a natural $E_\infty$-coalgebra structure on the chains of cubical sets.
The comparison between simplicial and cubical chains is presented in \cref{s:comparison}, where we show that the Cartan--Serre map is a quasi-isomorphism respecting $E_\infty$-structures.
We close presenting some future work in \cref{s:future}.

\section{Cubical topology} \label{s:cubical}

\subsection{Cubical sets}

The \textit{cube category} $\cube$ is the subcategory of $\Cat$ with objects $2^n = (0 \to 1)^n$.
We refer to \cite{grandis2003cubical} for a more leisure exposition and for variations on this definition.
The morphisms of the cube category are generated by the \textit{coface} and \textit{codegeneracy functors} defined by
\begin{align*}
\delta_i^\varepsilon & = \mathrm{id}_{2^{i-1}} \times \delta^\varepsilon \times \mathrm{id}_{2^{n-1-i}} \colon 2^{n-1} \to 2^n, \\
\sigma_i & = \mathrm{id}_{2^{i-1}} \times \, \sigma \times \mathrm{id}_{2^{n-i}} \quad \colon 2^{n} \to 2^{n-1},
\end{align*}
where $\varepsilon \in \{0,1\}$ and the functors
\begin{equation*}
\begin{tikzcd} [column sep=16pt]
2^0 \arrow[r, bend left, "\delta^0"] \arrow[r, bend right, "\delta^1"'] & 2^1 \arrow[r, "\sigma"] & 2^0
\end{tikzcd}
\end{equation*}
are defined by
\begin{equation*}
\delta^0(0) = 0, \qquad \delta^1(0) = 1, \qquad \sigma(0) = \sigma(1) = 0.
\end{equation*}

We denote by $\cube_{\deg}(2^m, 2^n)$ the subset of morphism in $\cube(2^m, 2^n)$ of the form $\sigma_i \circ \tau$ with $\tau \in \cube(2^m, 2^{n+1})$.

The category of \textit{cubical sets} is the functor category $\cSet = \Fun(\cube^\op, \Set)$.
The \textit{standard $n$-cube} is the cubical set $\cube^n = \cube(-, 2^n)$, and the \textit{Yoneda embedding} $\cube \to \cSet$ is the functor induced by $2^n \mapsto \cube^n$.

For any cubical set $X$ we have
\begin{equation*}
X_n \cong \colim_{\cube^n \to X} \cube^n,
\end{equation*}
and we denote $X(\delta_i^\varepsilon)$ and $X(\sigma_i)$ by $d_i^\varepsilon$ and $s_i$.

\subsection{Cubical chains}

The functor of \textit{chains} $\chains \colon \cSet \to \Ch$ is the Kan extension along the Yoneda embedding of the functor $\cube \to \Ch$ assigning to an object $2^n$ the chain complex having in degree $m$ the $R$ module
\begin{equation*}
\frac{R\{\cube(2^m, 2^n)\}}{R\{\cube_{\deg}(2^m, 2^n)\}}
\end{equation*}
and boundary map defined by
\begin{equation*}
\partial (\id_{2^n}) = \sum_{i=1}^{n} \ (-1)^i \
\big(\delta_i^1 - \delta_i^0 \big),
\end{equation*}
and to a morphism $\tau \colon 2^n \to 2^{n^\prime}$ the chain map
\begin{equation*}
\begin{tikzcd}[row sep=-3pt, column sep=normal,
/tikz/column 1/.append style={anchor=base east},
/tikz/column 2/.append style={anchor=base west}]
\chains_m(\cube^n) \arrow[r] &  \chains_m(\cube^{n^\prime}) \\
(2^m \to 2^n) \arrow[r, mapsto] & (2^m \to 2^n \stackrel{\tau}{\to} 2^{n^\prime}).
\end{tikzcd}
\end{equation*}
Explicitly,
\begin{equation*}
\chains(X) = \bigoplus_{n \geq 0} \chains(\cube^n) \otimes R[X_n] \ \Big/ \sim
\end{equation*}
where $(2^m \to 2^n) \circ \delta_i^\varepsilon \otimes x \sim (2^m \to 2^n) \otimes d_i^\varepsilon(x)$.

\subsection{Cubical singular complex}

Consider the topological $n$-cube
\begin{equation*}
\gcube^{n} = \{(x_1, \dots, x_n) \mid x_i \in [0,1]\}.
\end{equation*}
The assignment $2^n \to \gcube^n$ defines a functor $\cube \to \Top$ whose Kan extension is known as \textit{geometric realization}.
It has a right adjoint $\cSing \colon \Top \to \cSet$ given by
\begin{equation*}
Z \to \Big(2^n \to \Top(\gcube^n, X)\Big)
\end{equation*}
and referred to as the \textit{cubical singular complex} of the topological space $Z$.
The chain complex $\chains(\cSing Z)$ is referred to as the \textit{cubical singular chains} of $Z$.


\section{The \texorpdfstring{$E_\infty$}{E-infty}-prop \texorpdfstring{$\M$}{M}} \label{s:operads and props}

We now review the definition of the finitely presented $E_\infty$-prop $\M$ introduced in \cite{medina2020prop1} which, given its small number of generators and relations, is well suited to define $E_\infty$-structures.
In the next section we use this model to define a natural $E_\infty$-structure on cubical chains and cochains.
We start by reviewing the basic material in the theory of operads and props, referring the reader to, for example, \cite{markl2008props} for a more complete treatment.

\subsection{Symmetric modules and bimodules}

Let $\S$ be the category whose objects are the natural numbers and whose set of morphisms between $m$ and $n$ is empty if $m \neq n$ and is otherwise the symmetric group $\S_n$.
A \textit{left $\S$-module} (resp. \textit{right} $\S$-\textit{module} or $\S$-\textit{bimodule}) is a functor from $\S$ (resp. $\S^\op$ or $\S \times \S^\op$) to $\Ch$.
In this paper we prioritize left module structures over their right counterparts.
As usual, taking inverses makes both perspectives equivalent.
We respectively denote by $\smod$ and $\sbimod$ the categories of left $\S$-modules and of $\S$-bimodules with morphisms given by natural transformations.

The group homomorphisms $\S_n \to \S_n \times \S_1$ induce a forgetful functor $\forget \colon \sbimod \to \smod$.
Explicitly, $\forget(\P)(r) = \P(1, r)$ for $r \in \N$.
The similarly defined forgetful functor to right $\S$-modules will not be used.

\subsection{Composition structures}

We can define \textit{operads} and \textit{props} by enriching $\S$-modules and $\S$-bimodules with certain composition structures.
For a complete presentation of these concepts we refer to Definition~11 and 54 of \cite{markl2008props}.
Intuitively, using examples defined in the next subsection, operads and props can be understood by abstracting the composition structure naturally present in the left $\S$-module $\End^C$ (or right $\S$-module $\End_C$), naturally an operad, and the $\S$-bimodule $\End^C_C$, naturally a prop.
We remark that the prop structure on $\P$ restricts to an operad structure on $\forget(\P)$.

\subsection{Representations}

Given a chain complex $C$ define $\End^C$, $\End_C$ and $\End_C^C$ by
\begin{align*}
\End^C(r) &= \Hom(C, C^{\otimes r}),
& \End_C(r) &= \Hom(C^{\otimes r}, C),
& \End^C_C(r, s) &= \Hom(C^{\otimes r}, C^{\otimes s}),
\end{align*}
with their natural operad and prop structures respectively.
We remark that the forgetful functor $U$ sends $\End^C_C$ to $\End^C$.

Let $C$ be a chain complex, $\O$ an operad, and $\P$ a prop.
An $\O$-\textit{coalgebra} (resp. $\O$-\textit{algebra} or $\P$-\textit{bialgebra}) structure on $C$ is a structure preserving morphism $\O \to \End^C$ (resp. $\O \to \End_C$ or $\P \to \End_C^C$).

\subsection{\texorpdfstring{$E_\infty$}{E-infty}-operads and -props}

Recall that a \textit{free $\S_r$-resolution} of a chain complex $C$ is a quasi-isomorphism $R \to C$ from a chain complex of free $\k[\S_r]$-modules.

An $\S$-module $M$ is said to be $E_{\infty}$ if there exists a morphism of $\S$-modules $M \to \underline{\k}$ inducing for each $r \in \N$ a free $\S_r$-resolution $M(r) \to \k$.
For example, we can obtain one such $\S$-module by using the functor of singular chains and the set, parameterized by $r \in \N$, of maps to the terminal space from models of the universal bundle $\mathrm{E} \S_r$.

An operad is said to be $E_{\infty}$ if its underlying $\S$-module is $E_\infty$, and, following Boardman-Vogt \cite{boardman1973homotopy}, a prop $\P$ is said to be an $E_\infty$-prop if $U(\P)$ is an $E_\infty$-operad.

\subsection{Free prop construction} \label{ss:free prop}

\begin{figure}
	\input{aux/immersion}
	\caption{Immersed graphs represent labeled directed graphs with the direction implicitly given from top to bottom and the labeling from left to right.}
	\label{f:immersion}
\end{figure}

The \textit{free prop} $\free(M)$ generated by an $\S$-bimodule $M$ is constructed using directed graphs with no directed loops that are enriched with a labeling described next.
We think of each directed edge as built from two compatibly directed half-edges.
For each vertex $v$ of a directed graph $G$, we have the sets $in(v)$ and $out(v)$ of half-edges that are respectively incoming to and outgoing from $v$.
Half-edges that do not belong to $in(v)$ or $out(v)$ for any $v$ are divided into the disjoint sets $in(G)$ and $out(G)$ of incoming and outgoing external half-edges.
For any positive integer $n$ let $\overline{n} = \{1, \dots, n\}$ and set $\overline{0} = \emptyset$.
For any finite set $S$, denote the cardinality of $S$ by $|S|$.
The labeling is given by bijections
\[
\overline{|in(G)|}\to in(G), \qquad
\overline{|out(G)|}\to out(G),
\]
and
\[
\overline{|in(v)|}\to in(v), \qquad
\overline{|out(v)|}\to out(v),
\]
for every vertex $v$.
We refer to the isomorphism classes of such labeled directed graphs with no directed loops as $(n,m)$\textit{-graphs} denoting the set of these by $\G(m,n)$.
We use graphs immersed in the plane to represent elements in $\G(m,n)$, please see \cref{f:immersion}.
We consider the right action of $\S_n$ and the left action of $\S_m$ on a $(n,m)$-graph given respectively by permuting the labels of $in(G)$ and $out(G)$.
This action defines the $\S$-bimodule structure on the free prop
\begin{equation} \label{e:free prop}
\free(M)(m,n) \ = \bigoplus_{\Gamma \in \G(m,n)} \bigotimes_{v \in Vert(\Gamma)} out(v) \otimes_{\S_q} M(p, q) \otimes_{\S_p} in(v),
\end{equation}
where we simplified the notation writing $p$ and $q$ for $\overline{|in(v)|}$ and $\overline{|out(v)|}$ respectively.
The composition structure is defined by (relabeled) grafting and disjoint union.

\subsection{The prop $\M$}

We now recall the model of $E_\infty$ that is central to our constructions.

\begin{definition}
	Let $\M$ be the prop generated by
	\begin{equation} \label{e:generators of M}
	\counit\,, \hspace*{.6cm} \coproduct\,, \hspace*{.6cm} \product,
	\end{equation}
	in degrees $0$, $0$ and $1$ respectively, and boundaries
	\begin{equation} \label{e:boundary of M}
	\partial\ \counit = 0,
	\hspace*{.6cm}
	\partial\, \coproduct = 0,
	\hspace*{.6cm}
	\partial \product = \ \boundary\,,
	\end{equation}
	modulo the prop ideal generated by
	\begin{equation} \label{e:relations of M}
	\leftcounitality\,, \hspace*{.6cm} \rightcounitality\,, \hspace*{.6cm} \productcounit.
	\end{equation}
\end{definition}

Explicitly, any element in $\M(m,n)$ can be written as a linear combination of the $(m,n)$-graphs generated by those in \eqref{e:generators of M} via grafting, disjoint union and relabeling, modulo the prop ideal generated by the relations in \eqref{e:relations of M}, and its boundary is determined, using \eqref{e:free prop}, by \eqref{e:boundary of M}.

The same proof given in \cite[Theorem 3.3]{medina2020prop1} establishes the following.

\begin{proposition}
	The prop $\M$ is $E_{\infty}$.
\end{proposition}

We remark that, as proven in \cite{medina2018prop2}, this prop is obtained from applying the functor of cellular chains to a finitely presented prop over the category of CW-complexes.


\section{$\M$-bialgebra on cubes} \label{s:coaction}

Let $C$ be a chain complex, $\O$ an operad, and $\P$ a prop.
An $\O$-\textit{coalgebra} (resp. $\O$-\textit{algebra} or $\P$-\textit{bialgebra}) structure on $C$ is a structure preserving morphism $\O \to \End^C$ (resp. $\O \to \End_C$ or $\P \to \End_C^C$).

In this section we construct a natural $\M$-bialgebra structure on the chains of standard cubes $\cchains(\cube^n)$.
These are determined by three linear maps satisfying the relations in the presentation of $\mathcal M$.
For $n \in \mathbb{N}$, define: \vspace*{5pt} \\
(1) The counit $\epsilon \in \Hom(\cchains(\square^n), \Z)$ known as the \textit{augmentation} by
\begin{equation*}
\epsilon \left( x_1 \otimes \cdots \otimes x_d \right) = \epsilon(x_1) \cdots \, \epsilon(x_n),
\end{equation*}
where
\begin{equation*}
\epsilon([0]) = \epsilon([1]) = 1, \qquad \epsilon([0, 1]) = 0.
\end{equation*} \vspace*{-6pt} \\
(2) The coproduct $\Delta \in \Hom \left( \cchains(\square^n), \cchains(\square^n)^{\otimes 2} \right)$ known as the \textit{Serre diagonal} by
\begin{equation*}	
\Delta (x_1 \otimes \cdots \otimes x_n) = 	
\sum \pm \left( x_1^{(1)} \otimes \cdots \otimes x_n^{(1)} \right) \otimes 	
\left( x_1^{(2)} \otimes \cdots \otimes x_n^{(2)} \right),	
\end{equation*}	
where the sign is determined using the Koszul convention, and we are using Sweedler's notation
\begin{equation*}	
\Delta(x_i) = \sum x_i^{(1)} \otimes x_i^{(2)}
\end{equation*}
for the chain map $\Delta \colon \cchains(\square^1) \to \cchains(\square^1)^{\otimes 2}$ defined by
\begin{equation*}
\Delta([0]) = [0] \otimes [0], \quad \Delta([1]) = [1] \otimes [1], \quad \Delta([0, 1]) = [0] \otimes [0, 1] + [0, 1] \otimes [1].
\end{equation*}
Using that $\cchains(\square^n) = \cchains(\square^1)^{\otimes n}$, $\Delta$ is the composition
\begin{equation*}
\begin{tikzcd}
\cchains(\square^1)^{\otimes n} \arrow[r, "\Delta^{\otimes n}"] &[3pt] \left( \cchains(\square^1)^{\otimes 2}  \right)^{\otimes n} \arrow[r, "sh"] &[-5pt] \left( \cchains(\square^1)^{\otimes n} \right)^{\otimes 2}
\end{tikzcd}
\end{equation*}
where $sh$ is the shuffle map that places tensor factors in odd position first. \vspace*{5pt} \\
(3) The product $\ast \in \Hom(\cchains(\square^n)^{\otimes 2}, \cchains(\square^n))$ by
\begin{align*}
(x_1 \otimes \cdots \otimes x_n) \ast (y_1 \otimes \cdots \otimes y_n) =
(-1)^{|x|} \sum_{i=1}^n x_{<i}\, \epsilon(y_{<i}) \otimes x_i \ast y_i \otimes \epsilon(x_{>i}) \, y_{>i},
\end{align*}
where
\begin{align*}
x_{<i} & = x_1 \otimes \cdots \otimes x_{i-1}, &
y_{<i} & = y_1 \otimes \cdots \otimes y_{i-1}, \\
x_{>i} & = x_{i+1} \otimes \cdots \otimes x_n, & 
y_{>i} & = y_{i+1} \otimes \cdots \otimes y_n,
\end{align*}
with the convention
\begin{equation*}
x_{<1} = y_{<1} = x_{>n} = y_{>n} = 1 \in \Z,
\end{equation*}
and the only non-zero values of $x_i \ast y_i$ are
\begin{equation*}
\ast([0] \otimes [1]) = [0, 1], \qquad  \ast([1] \otimes [0]) = -[0, 1].
\end{equation*}

We devote Section~\ref{s:proof} to the proof of the following statement, our main result.

\begin{theorem} \label{t:cubical chain bialgebra}
	The assignment
	\begin{equation*}
	\counit \mapsto \epsilon, \quad \coproduct \mapsto \Delta, \quad \product \mapsto \ast,
	\end{equation*}
	induces a functor $\cube \to \biAlg_{\M}$ or, equivalently, a natural $\mathcal M$-bialgebra structure on $\cchains(\square^n)$ for every $n \in \mathbb{N}$.
\end{theorem}

The category of bialgebras over a prop is in general not cocomplete, but those of algebras and coalgebras over operads are.
Consider the composition $\cube \to \biAlg_{\M} \to \coAlg_{U(\M)}$ of the previous functor and the one induced by the forgetful functor $U$ from props to operads.
A Kan extension of this composition along the Yoneda embedding gives us the following result.

\begin{corollary}
	The assignment
	\begin{equation*}
	\counit \mapsto \epsilon, \quad \coproduct \mapsto \Delta, \quad \product \mapsto \ast,
	\end{equation*}
	induces a lift of the functor of chains
	\begin{equation*}
	\begin{tikzcd}
	& \coAlg_{U(\M)} \arrow[d] \\
	\cSet \arrow[r, "N"] \arrow[dashed, ur, bend left]& \Ch,
	\end{tikzcd}
	\end{equation*}
	as well as a lift to $\Alg_{U(\M)}$ of the functor of cochains $\Hom(-, R) \circ \cchains$.
\end{corollary}

Explicitly, if $\Gamma \in U(\M)(r)$ and $\Gamma \colon \chains(\cube^n) \to \chains(\cube^n)^{\otimes r}$ is determined by $\Gamma(\id_{2^n}) = \sum_{\lambda \in \Lambda} \alpha_\lambda \, \tau_\lambda^1 \otimes \cdots \otimes \tau_\lambda^r$ with $\tau_\lambda^i \colon 2^{m_\lambda^i} \to 2^n$ and $\alpha_\lambda \in R$. Then, $\Gamma \colon N(X) \to N(X)^{\otimes r}$ is determined by
\begin{equation} \label{e:explicit cubical coaction}
\Gamma(\id_{2^n} \otimes x) =
\sum_{\lambda \in \Lambda} \alpha_\lambda \, \big(2^{m_\lambda^1} \otimes X(\tau_\lambda^1)(x)\big) \otimes \cdots \otimes \big(2^{m_\lambda^r} \otimes X(\tau_\lambda^r)(x)\big)
\end{equation}
where $x \in X_n$.

\section{The Cartan--Serre comparison map} \label{s:the cartan-serre comparison map}

Let us consider, with their usual CW structures, the topological simplex $\gsimplex^n$
%\[
%\gsimplex^{\!n} = \{(y_0, \dots, y_n) \mid y_i \in [0,1], \ \textstyle{\sum} \, y_i = 1\}
%\]
and the topological cube $\gcube^n$.
In \cite[p. 442]{serre1951homologie}, Serre described for any space a quasi-isomorphism of coalgebras between its simplicial and cubical singular chains induced by precomposing with a canonical cellular map $\CScollapse \colon \gcube^n \to \gsimplex^n$ also considered in \cite[p.199]{eilenberg1953acyclic} where it is attributed to Cartan.

%\begin{definition}
%	The \textit{Cartan--Serre collapse map}
%	\[
%	\CScollapse \colon \gcube^n \to \gsimplex^{\!n}
%	\]
%	is the natural cellular map defined by
%	\begin{equation} \label{e:cartan-serre collapse map}
%	\begin{split}
%	&y_0 = 1 - x_1, \\
%	&y_1 = x_1(1 - x_2), \\
%	&\ \vdots \\
%	&y_{n-1} = x_1 x_2 \dots x_{n-1}(1-x_n), \\
%	&y_{n} = x_1 x_2 \dots x_n.
%	\end{split}
%	\end{equation}
%\end{definition}

%The aforementioned comparison map of singular chains is given, for any space $Z$, by precomposing a simplicial singular simplex $\gsimplex^n \to Z$ with $\CScollapse$.

The goal of this section is to deduce from a more general categorical statement that this comparison map is a quasi-isomorphism of $E_\infty$-coalgebras and, as a consequence, that its linear dual is one of $E_\infty$-algebras.

\subsection{Simplicial sets}

We denote the \textit{simplex category} by $\simplex$, the category of \textit{simplicial sets} $\Fun(\simplex^\op, \Set)$ by $\sSet$ and the standard $n$-simplex by $\simplex^n$.
As usual, we denote an element in $\simplex^n_m$ by a non-decreasing tuples $[v_0, \dots, v_m]$ with $v_i \in \{0, \dots, n\}$.
The \textit{product} of simplicial sets is defined object-wise.
For example,
\[
\big(\simplex^n \times \simplex^{n^\prime}\big)_m = \simplex^n_m \times \simplex^{n^\prime}_m
\]
consists of pairs of non-decreasing tuples $[v_0, \dots, v_m] \times [w_0, \dots, w_m]$ of appropriate integers.

The \textit{simplicial singular complex} functor is denoted by $\sSing \colon \Top \to \sSet$ and the functor of (normalized) \textit{chains} by $\schains \colon \sSet \to \Ch$.
We omit the superscript $\simplex$ from either of these if no confusion may result from doing so.

The \textit{Alexander--Whitney coalgebra} functor is the Yoneda extension of the functor defined by the following natural maps.
For any $n \in \N$, define $\epsilon \colon \chains(\simplex^n) \to \k$ by
\[
\epsilon \big( [v_0, \dots, v_q] \big) = \begin{cases} 1 & \text{ if } q = 0, \\ 0 & \text{ if } q>0, \end{cases}
\]
and $\Delta \colon \chains(\simplex^n) \to \chains(\simplex^n)^{\otimes2}$ by
\[
\Delta \big( [v_0, \dots, v_q] \big) = \sum_{i=0}^q [v_0, \dots, v_i] \otimes [v_i, \dots, v_q].
\]




\subsection{Triangulation and its right adjoint}

The \textit{simplicial $n$-cube} is the $n^\th$-fold Cartesian product $\scube{n}$.
The assignment $2^n \mapsto \scube{n}$ defines a functor $\cube \to \sSet$ with $\delta_i^\varepsilon \colon \scube{n} \to \scube{(n+1)}$ inserting $[\varepsilon, \dots, \varepsilon]$ as $i^\th$ factor and $\sigma_i \colon \scube{(n+1)} \to \scube{n}$ removing the $i^\th$ factor.
Its Yoneda extension
\[
\triangulate \colon \cSet \to \sSet
\]
is referred to as the \textit{triangulation} functor, which admits a right adjoint
\[
\cubify \colon \sSet \to \cSet
\]
defined by
\[
\cubify(X)(2^m) = \sSet \big( \scube{n}, \, X \big).
\]

As proven in \cite[\subsectionSymbol 8.4.30]{cisinski2006presheaves}, the pair $(\triangulate,\, \cubify)$ defines a Quillen equivalence when $\sSet$ and $\cSet$ are considered as model categories.

\subsection{Projection and inclusion} \label{ss:projection and inclusion}

We define for each $n \in \N$ a natural simplicial map
\[
\projection \colon \scube{n} \to \simplex^n
\]
referred to as the \textit{projection} by
\[
[\varepsilon_0^1, \dots, \varepsilon_m^1] \times \dots \times [ \varepsilon_0^n, \dots, \varepsilon_m^n] \mapsto
[v_0, \dots, v_m]
\]
where
\[
v_i = \varepsilon_i^1 + \varepsilon_i^1 \varepsilon_i^2 + \dots + \varepsilon_i^1 \dotsm \varepsilon_i^n.
\]
Together with the naturality of the projection with respect morphisms in the simplex category we have the following compatibility with respect to cubical coface maps:
For $i \in \{1,\dots,n\}$
\begin{equation} \label{e:projection cubical compatibility}
\projection \circ \, \delta_i^1 = \delta_{i-1} \circ \pi, \qquad
\projection \circ \, \delta_n^0 = \delta_{n} \circ \pi,
\end{equation}
and $\projection \circ \, \delta^0_i$ is degenerate for $1 < i < n$.

The projection has a section
\[
\inclusion \colon \simplex^n \to \scube{n}
\]
defined by sending $\id_{[n]}$ to $\varepsilon^1 \times \dots \times \varepsilon^n$ with
\[
\varepsilon^i = [\overbrace{0, \dots, 0}^{i}, 1, \dots, 1].
\]


\subsection{The subdivision map}

The \textit{subdivision} map of a cubical set $X$
\[
\subdivide_X \colon \cchains(X) \to \schains(\triangulate X)
\]
is the natural chain map defined by the well known Eilenberg--Zilber map
\[
\subdivide \colon \chains(\cube^n) \cong \chains(\simplex^1)^{\otimes n} \xra{EZ} \chains \scube{n}.
\]

The $EZ$ map is natural with respect to cubical morphisms and, consequently, so is the subdivision map $\subdivide$.

We can use the subdivision map to provide an alternative description of the complex $\cchains(\cubify Y)$ for any simplicial set $Y$.
Since the category of cubical maps $\cube^n \to \cubify Y$ is equivalent to the category $\sC_Y$ whose objects are chain maps
\[
\chains(\cube^n) \xra{\subdivide} \chains \scube{n} \to \schains(Y)
\]
where the second map is induced from a simplicial map
and morphisms are appropriate commutative diagrams, we have
\begin{equation} \label{e:alternative description of NUY}
\cchains(\cubify Y) \cong \colim_{\cC_Y} \chains(\cube^n).
\end{equation}

\subsection{The Cartan--Serre collapse map}

Let us consider the model of the topological $n$-simplex given by
\[
\gsimplex^n = \big\{ (y_1, \dots, y_n) \in \gcube^n \mid i \leq j \Rightarrow y_i \geq y_j \big\},
\]
whose cell structure associates $[v_0, \dots, v_m]$ with the subset
\[
\Big\{ \big( \underbrace{1, \dots, 1}_{v_0}, \underbrace{y^\prime_1, \dots y^\prime_1}_{v_1-v_0}, \dots, \underbrace{y^\prime_m, \dots y^\prime_m}_{v_m-v_{m-1}}, \underbrace{0, \dots, 0}_{n-v_m} \big) \mid y^\prime_1 \geq \dots \geq y^\prime_m \Big\}.
\]
The spaces $\gsimplex^n$ define a functor $\simplex \to \Top$ with codegeneracies given by
\[
\sigma_i(x_1, \dots, x_n) = (x_1, \dots, \widehat x_i, \dots, x_n)
\]
and coface maps by
\begin{align*}
\delta_0(x_1, \dots, x_n) &= (1, x_1, \dots, x_n), \\
\delta_i(x_1, \dots, x_n) &= (x_1, \dots, x_i, x_i, \dots, x_n), \\
\delta_n(x_1, \dots, x_n) &= (x_1, \dots, x_n, 0).
\end{align*}

The \textit{Cartan-Serre collapse map} is the cellular map defined by
\[
\CScollapse(x_1, \dots, x_n) = (x_1,\ x_1 x_2, \, \dots \, , \ x_1 \dotsm x_n).
\]
It is straightforward to verify that
\begin{align*}
\CScollapse \circ \, \delta_i^1 = \delta_{i-1} \circ \CScollapse, \qquad
\CScollapse \circ \, \delta_n^0 = \delta_n \circ \CScollapse,
\end{align*}
for $i \in \{1, \dots, n\}$ and that the image of $\CScollapse \circ \, \delta_i^0$ for $1 \leq i < n$ is in a lower dimensional skeleton of $\gsimplex^n$.
Therefore, since the image under the induced chain map $\CScollapse_\ast$ of the top dimensional generator of $\gchains(\gcube^n)$ is $[0, \dots, n]$ as is for $\chains(\cube^n)$ under $\projection_\ast \circ \subdivide$, we have the following.

\begin{lemma}
	The composition
	\[
	\gchains(\gcube^n) \cong \chains(\cube^n) \xra{\subdivide} \chains \big( \scube{n} \big) \xra{\projection_\ast} \chains(\simplex^n) \cong \gchains(\gsimplex^n)
	\]
	agrees the chain map $\CScollapse_\ast$ induced by Cartan--Serre collapse map.
\end{lemma}

For later use we record the following.
\begin{lemma} \label{l:kernel of psi}
	For any basis element $x = x_1 \otimes \cdots \otimes x_n$ of degree $m$ we have $\CScollapse_\ast(x) = 0$ if there exists $i < j$ such that $x_i = [0]$ and $x_j = [0,1]$.
	Otherwise, $\CScollapse_\ast(x) = [v_0, \dots, v_m]$ with
\end{lemma}

\begin{proof}
	This follows from noticing that for any point in $\gcube^n$ whose $i^\th$ coordinate is $0$, its image under the Cartan--Serre collapse map has $j^\th$ coordinate equal to $0$ for every $j \geq i$.
\end{proof}

We remark that Cartan used the model of the topological $n$-simplex as a subset of $\R^{n+1}$ in \cite[p. 442]{serre1951homologie} to define the collapse map.
We find it more convenient to use the model of $\gsimplex^n$ as a subset of $\gcube^n$.

%There is a preferred element in $\cubify \simplex^n$ given by the composition $\subdivide \circ \projection_\ast$ where $\projection_\ast$
%is the chain map induced by the projection $\projection$.
%We now give a geometric description of this map.

%We now study the induced chain map.
%Let $x = x_1 \otimes \cdots \otimes x_n \in \gchains(\gcube^n)$ be basis element of degree $m$ with $x_{p_i} = [0,1]$ for some $p_1 < \dots < p_m$.
%Let $q = \min\{i \mid x_i = [0]\}$ or $q = n+1$ if empty.
%If $q < p_i$ for some $i$ then $\CScollapse_\ast(x) = 0$ since the image of its associated cell is mapped into a lower dimensional skeleton.
%If not, let us consider the element $x_{<q} = x_1 \otimes \dots \otimes x_{q-1}$.
%
%we have
%
%and we have
%\begin{lemma} \label{l:kernel of psi}
%	For any basis element $x = x_1 \otimes \cdots \otimes x_n$ of degree $m$ we have $\CScollapse_\ast(x) = 0$ if there exists $i < j$ such that $x_i = [0]$ and $x_j = [0,1]$.
%	Otherwise, $\CScollapse_\ast(x) = [v_0, \dots, v_m]$ with
%\end{lemma}
%
%\begin{proof}
%	This follows from noticing that for any point in $\gcube^n$ whose $i^\th$ coordinate is $0$, its image under the Cartan--Serre collapse map has $j^\th$ coordinate equal to $0$ for every $j \geq i$.
%\end{proof}

\subsection{The Cartan--Serre comparison map} \label{ss:comparison map}

The projection $\projection \colon \scube{n} \to \simplex^{\!n}$ induces for any simplicial set $Y$ a natural morphism of graded sets
\[
Y \to \cubify Y
\]
defined on a standard simplex $\simplex^n$ by
\[
\big( [m] \xra{\sigma} [n] \big) \mapsto
\big( \scube{m} \xra{\projection} \simplex^m \xra{\sigma_\ast} \simplex^n \big).
\]
Additionally, passing to chains gives a graded linear map
\[
\CScomp_Y \colon \schains(Y) \to \cchains(\cubify Y)
\]
which we refer to as the \textit{Cartan--Serre comparison map}.

We can give a more explicit description of this map using \eqref{e:alternative description of NUY}:
It suffices to describe the case $Y = \simplex^n$ given by
\[
\begin{tikzcd}[column sep=small, row sep=0]
\chains(\simplex^n) \arrow[r, "\CScomp"] &
\chains(\cubify \simplex^n) \\
\id_{[n]} \arrow[r, mapsto] &
\CScollapse_\ast.
\end{tikzcd}
\]

We have the following key result.

\begin{lemma} \label{l:cartan serre quasi-iso}
	The Cartan--Serre comparison map $\CScomp_Y$ is a quasi-isomorphism for any simplicial set $Y$
\end{lemma}

\begin{proof}
	It suffices to prove this for $Y = \simplex^n$ since the general statement follows from naturality and an acyclic carrier argument \cite{eilenberg1953acyclic}.
	Since $\CScollapse_\ast = \chains \projection \circ \subdivide$ is a chain map, we have
	\begin{align*}
	\CScomp \big( \partial^\simplex \id_{[n]} \big) \defeq
	\partial^\simplex \circ \CScollapse_\ast =
	\CScollapse_\ast \circ \, \partial^\cube \defeq
	\partial^\cube \CScomp (\id_{[n]}).
	\end{align*}
	That $\CScomp$ induces an isomorphism in homology can be seen easily from the contractibility of both $\simplex^n$ and $\cubify \simplex^n$.
\end{proof}

\subsection{Simplicial $E_\infty$-structure} \label{ss:e infinity structures}

In \cite{medina2020prop1}, a similar construction to the one introduced in \cref{s:action} provides the chains of simplicial sets with a natural $\UM$-coalgebra structure.
It is also induced from a natural $\M$-bialgebra structure on the chains of representables objects, standard simplices in this case.
This $\M$-bialgebra structure on $\chains(\simplex^{\!n})$ is defined by the assignment
\[
\counit \mapsto \epsilon, \quad \coproduct \mapsto \Delta, \quad \product \mapsto \ast,
\]
where $\epsilon$ and $\Delta$ constitute the Alexander--Whitney coalgebra structure on simplicial chains, and
\[
\ast \colon \chains(\simplex^{\!n})^{\otimes 2} \to \chains(\simplex^{\!n})
\]
is an algebraic version of the \textit{join} defined by
\[
\left[v_0, \dots, v_p \right] \ast \left[v_{p+1}, \dots, v_q\right] = \begin{cases} (-1)^{p+|\pi|} \left[v_{\pi(0)}, \dots, v_{\pi(q)}\right] & \text{ if } v_i \neq v_j \text{ for } i \neq j, \\
0 & \text{ if not}, \end{cases}
\]
where $\pi$ is the permutation that orders the totally ordered set of vertices and $(-1)^{|\pi|}$ is its sign.

Although for any simplicial set $Y$ both $\schains(Y)$ and $\cchains(\cubify Y)$ have natural $\UM$-structures, the map $\CScomp_Y$ is not a morphism of $\UM$-coalgebras for an arbitrary $Y$.
Nevertheless, after restriction of their $\UM$-structures via an inclusion of $E_\infty$-operads $\USL \to \UM$, the Cartan--Serre comparison map becomes a morphism of $E_\infty$-coalgebras.

The operad $\USL$ is generated as a suboperad of $\UM$ by all so called \textit{surjection-like} graphs, i.e., immerse connected graphs of the form
\input{aux/surjection_like}
where there are no hidden vertices and the strands are joined so that the associated maps $\{1, \dots, k_j\} \to \{1, \dots, n+k\}$ are order-preserving.
We notice that the subcomplex of surjection-like $(1,r)$-graphs is contractible using the same chain contraction employed in \cite{medina2020prop1}.
This implies that the suboperad $\USL$ of $\UM$ is also $E_\infty$.

\begin{example}
	We illustrate the need to restrict to $\USL$ by providing a simple example showing that $\CScollapse_\ast$ does not preserve $\M$-structures.
	From it is easy to construct others showing $\CScollapse_\ast$ does not preserve the $\UM$-structures either.

	Since $\CScollapse_\ast \big( [0] \otimes [0,1] \big) = 0$ and
	\begin{align*}
	\big( [1] \otimes [1] \big) \ast \big( [0] \otimes [0,1] \big) &=
	- \big( [0,1] \otimes [0,1] \big), \\
	- \CScollapse_\ast \big( [0,1] \otimes [0,1] \big) & = - [0,1,2]
	\end{align*}
	we have
	\[
	\CScollapse_\ast \big( [1] \otimes [1] \big) \ast \CScollapse_\ast \big( [0] \otimes [0,1] \big) \neq \CScollapse_\ast \Big( ([1] \otimes [1]) \ast ([0] \otimes [0,1]) \Big).
	\]
	We will return to this example in \cref{ss:comparison proof} after introducing further structure on the image of elements in $\USL$ missing from that of general elements in $\UM$.
\end{example}

\subsection{The Cartan--Serre comparison map as an $E_\infty$-coalgebra morphism} \label{ss:the cartan-serre chain map}

We now come to the main result of this section.

\begin{theorem} \label{t:main comparison}
	The Cartan--Serre comparison map $\CScomp_Y \colon \schains(Y) \to \cchains(\cubify Y)$ is a quasi-isomorphism of $\USL$-coalgebras for any simplicial set $Y$
\end{theorem}

We deduce this from the following result which we prove in \cref{ss:comparison proof}.

\begin{lemma} \label{l:main comparison}
	The chain map $\CScollapse_\ast \colon \chains(\cube^n) \to \chains(\simplex^n)$ induced by the Cartan--Serre collapse map is a quasi-isomorphism of $\USL$-coalgebras for every $n \in \N$.
\end{lemma}

\begin{proof}[Proof of \cref{t:main comparison}]
	It suffices to prove this for $Y = \simplex^n$.
	Since $\CScollapse_\ast = \projection_\ast \circ \subdivide$ is a morphism of $\USL$-coalgebras (\cref{l:main comparison}), for any $\Gamma \in \USL(r)$ we have
	\begin{align*}
	\CScomp^{\otimes r} \! \big( \Gamma(\id_{[n]}) \big) \defeq
	\Gamma \circ \CScollapse_\ast =
	\CScollapse_\ast^{\otimes r} \! \circ \, \Gamma \defeq
	\Gamma \big( \CScomp (\id_{[n]}) \big),
	\end{align*}
	as claimed.
\end{proof}

For any topological space $Z$ the (\textit{topological}) \textit{Cartan--Serre comparison map}
\[
\CScomp_Z \colon \chains(\sSing Z) \to \chains(\cSing Z)
\]
is the chain map obtained by precomposing a singular simplex with the Cartan--Serre collapse map $\CScollapse \colon \gcube^n \to \gsimplex^{\!n}$.
We have the following consequence of \cref{t:main comparison}.

\begin{corollary} \label{t:topological comparison}
	For any topological space $Z$, the quasi-isomorphism
	\[
	\chains(\sSing Z) \to \chains(\cSing Z)
	\]
	is a morphism of $E_\infty$-coalgebras, where the domain and target $E_\infty$-structures extend respectively the Alexander--Whitney and Serre coalgebra structures.
\end{corollary}

\begin{proof}
	This map factors as a composition
	\[
	\chains(\sSing Z) \to \chains(\cubify \sSing Z) \to \chains(\cSing Z)
	\]
	where the first map is the Cartan--Serre comparison map of \cref{ss:comparison map} and the second is induced from the morphism of cubical sets defined by the assignment
	\[
	\big( \scube{n} \xra{F} \sSing Z \big) \mapsto
	\big( \gcube^n \cong \bars{\scube{n}} \xra{\bars{\projection}} \gsimplex^n \xra{F(\inclusion(\id_{[n]}))} Z \big)
	\]
	where $\inclusion$ is the section of $\projection$ defined in \cref{ss:projection and inclusion}.
	The first is a morphism of $\USL$-coalgebras by \cref{t:main comparison} and the second by naturality, so the statement is proven.
\end{proof}


\subsection{Proof of \cref{l:main comparison}} \label{ss:comparison proof}

Throughout this subsection we use the identifications $\chains(\cube^n) \cong \gchains(\gcube^n)$ and $\chains(\simplex^n) \cong \gchains(\gsimplex^n)$.
By naturality, it suffices to show that
\begin{equation} \label{e:Cartan--Serre E-coalgebra map}
\CScollapse_\ast^{\otimes r} \! \circ \, \Gamma \big( [0,1]^{\otimes n} \big) =
\Gamma \circ \CScollapse_\ast \big( [0,1]^{\otimes n} \big),
\end{equation}
where $\Gamma$ is represented by a surjection-like graph in $\USL(r)$.

We begin with the following enhancement of \cref{l:cartan serre quasi-iso}.

\begin{lemma}
	The chain map $\CScollapse_\ast$ is a quasi-isomorphism of coalgebras.
\end{lemma}

\begin{proof}
	Since vertices are sent to vertices by $\CScollapse$, we have $\CScollapse_\ast \circ \, \epsilon = \epsilon \circ \CScollapse_\ast$.
	To study the compatibility of $\CScollapse$ with coproducts consider $n > 0$ and
	\[
	\Delta \big( [0,1]^{\otimes n} \big) = \sum_{\lambda \in \Lambda} \pm \ x_1^{(\lambda)} \otimes \cdots \otimes x_n^{(\lambda)} \bm{\otimes} y_1^{(\lambda)} \otimes \cdots \otimes y_n^{(\lambda)},
	\]
	where $\Lambda$ parameterizes all choices of $x_i^{(\lambda)} \in \{[0], [0,1]\}$ and $y_i^{(\lambda)} \in \{[0,1], [1]\}$ such that
	\begin{align*}
	x_i^{(\lambda)} = [0]   & \iff y_i^{(\lambda)} = [0,1], \\
	x_i^{(\lambda)} = [0,1] & \iff y_i^{(\lambda)} = [1].
	\end{align*}
	By \cref{l:kernel of psi}, the summands above not sent to $0$ by $\CScollapse_\ast \otimes \CScollapse_\ast$ are those basis elements for which $x_i^{(\lambda)} = [0]$ implies $x_j^{(\lambda)} = [0]$ for all $i < j$.
	For any one such summand, its sign is positive and its image by $\CScollapse_\ast \otimes \CScollapse_\ast$ is $[0, \dots, k] \otimes [k, \dots, n]$ where $k+1 = \min \{i \mid x_i^{(\lambda)} = [0]\}$ or $k = n$ if this set is empty.
	The summands $[0, \dots, k] \otimes [k, \dots, n]$ are precisely those appearing when applying the Alexander--Whitney coproduct to $[0, \dots, n] = \CScollapse_\ast \big( [0,1]^{\otimes n} \big)$.
\end{proof}

We will consider the basis of $\chains(\cube^n)$ as a poset in the following way.

\begin{definition}
	For $n = 1$ we set $[0] < [0,1] < [1]$, and for $n > 1$ we have $(x_1 \otimes \cdots \otimes x_n) \leq (y_1 \otimes \cdots \otimes y_n)$ iff $x_i \leq y_i$ for each $i \in \{1, \dots, n\}$.
\end{definition}

\begin{lemma}
	Let $\Delta^{r-1}$ be the $(r-1)^\th$ iterated Serre coproduct.
	If
	\[
	\Delta^{r-1} \big([0,1]^{\otimes n}\big) =
	\sum \pm \ x{(1)} \bm{\otimes} \cdots \bm{\otimes} x{(r)}
	\]
	with each $x(i) \in \chains(\cube^n)$ a basis element, then $x{(1)} \leq \cdots \leq x{(r)}$.
\end{lemma}

\begin{proof}
	For $r = 2$ we have for every $i \in \{1, \dots, n\}$ that
	\begin{align*}
	x(1)_i = [0]   & \iff x(2)_i = [0,1], \\
	x(1)_i = [0,1] & \iff x(2)_i = [1],
	\end{align*}
	and that neither $x(1)_i = [1]$ or $x(2)_i = [0]$ can occur, hence $x(1) \leq x(2)$.
	The claim for $r > 2$ follows from a straightforward induction argument.
\end{proof}

\begin{lemma}
	Let $x, y, z \in \chains(\cube^n)$ be basis elements.
	If $x, y \leq z$ then $(x \ast y) \leq z$.
\end{lemma}

\begin{proof}
	Recall that
	\begin{align*}
	(x_1 \otimes \cdots \otimes x_n) \ast (y_1 \otimes \cdots \otimes y_n) =
	(-1)^{|x|} \sum_{i=1}^n x_{<i}\, \epsilon(y_{<i}) \otimes x_i \ast y_i \otimes \epsilon(x_{>i}) \, y_{>i}.
	\end{align*}
	By assumption, for every $i$ we have $x_{<i} \leq z_{<i}$ and $y_{>i} \leq z_{>i}$, and if $x_i \ast y_i \neq 0$ then either $x_i = [1]$ or $y_i = [1]$ which implies $z_i = [1]$ as well.
\end{proof}

\begin{lemma}
	Let $x, y \in \chains(\cube^n)$ be basis elements.
	If $x \leq y$ then
	\begin{equation} \label{e:cs collapse as algebra map}
	\CScollapse_\ast(x \ast y) = \CScollapse_\ast(x) \ast \CScollapse_\ast(y).
	\end{equation}
\end{lemma}

\begin{proof}
	We present this proof a sequence of three claims. \newline

	\noindent \textit{Claim 1}.
	If $\CScollapse_\ast(x) = 0$ or $\CScollapse_\ast(y) = 0$ then
	\begin{equation} \label{e:zero for join}
	\CScollapse_\ast \big( x_{<i}\, \epsilon(y_{<i}) \otimes x_i \ast y_i \otimes \epsilon(x_{>i}) \, y_{>i} \big) = 0.
	\end{equation}

	Assume $\CScollapse_\ast(x) = 0$, that is, there exists a pair $p < q$ such that $x_p = [0]$ and $x_q = [0,1]$, then \eqref{e:zero for join} holds since:
	\begin{enumerate}
		\item If $i > q$, then $x_p$ and $x_q$ are part of $x_{<i}$.
		\item If $i = q$, then $x_q \ast y_q = 0$ for any $y_q$.
		\item If $i < q$, then $\varepsilon(x_{>i}) = 0$.
	\end{enumerate}
	Similarly, if there is a pair $p < q$ such that $y_p = [0]$ and $y_q = [0,1]$,  then \eqref{e:zero for join} holds since:
	\begin{enumerate}
		\item If $i < p$, then $y_p$ and $y_q$ are part of $y_{>i}$.
		\item If $i = p$ or, more generally, $y_i = [0]$, then $x_i = [0]$ and $x_i \ast y_i = 0$.
		\item If $i = q$ or, more generally, $y_i = [0,1]$, then $x_i \ast y_i = 0$ for any $x_i$.
		\item If $i > q$, then $\varepsilon(y_{<i}) = 0$.
		\item If $p < i < q$ and $y_i = [1]$ then either $x_i \ast x_j = 0$ or $x_i \ast x_j = [0,1]$, implying $(x \ast y)_p = [0]$ and $(x \ast y)_i = [0,1]$.
	\end{enumerate}
	This proves the first claim and identity \eqref{e:cs collapse as algebra map} under its hypothesis. \newline

	\noindent \textit{Claim 2}.
	If $\CScollapse_\ast(x) \neq 0$ and $\CScollapse_\ast(y) \neq 0$ then
	\[
	\CScollapse_\ast(x \ast y) = \CScollapse_\ast \big( x_{<p_x} \, \varepsilon(y_{<p_x}) \otimes \, x_{p_x} \! \ast y_{p_x} \otimes \varepsilon(x_{>p_x}) \, y_{>p_x} \big).
	\]

	To make sure the elements $x$ and $y$ have $[0]$ as a tensor factor in them we consider $\chains(\cube^n)$ as a subcomplex of $\chains(\cube^{n+1})$ via the inclusion that tensors on the right with $[0]$.
	By naturality, we do not loose generality making this assumption.
	Let $p_x = \min \big\{ i \mid x_i = [0] \big\}$ and let $p_y$ be defined analogously.

	Given the assumptions of the claim and \cref{l:kernel of psi} the basis elements
	\[
	v_x = x_{< p_x}
	\quad \text{ and } \quad
	v_y = y_{< p_y}
	\]
	have tensor factors in $\big\{ [0,1], [1] \big\}$, whereas
	\[
	w_x = x_{> p_x}
	\quad \text{ and } \quad
	w_y = y_{> p_y}
	\]
	have tensor factors in $\big\{ [0], [1] \big\}$.
	We remark that any of these could be the unit $1_\k$.
	Since $x \leq y$ we have that $p_x \leq p_y$.
	If $i < p_x$ then $x_i = [1]$ or $x_i = [0,1]$.
	If $x_i = [1]$, since $x_i \leq y_i$, it is impossible for $y_i = [0]$, the only case when $x_i \ast y_i \neq 0$.
	If $x_i = [0,1]$ then $x_i \ast y_i = 0$ for any $y_i$.
	If $i > p_x$, then either $x_i \ast y_i = 0$ or $x_i \ast y_i = [0,1]$.
	In the first case there is nothing to prove and in the second we notice that $(x \ast y)_{p_x} = [0]$ and $(x \ast y)_{} = [0,1]$ so $\CScollapse_\ast(x \ast y) = 0$.
	This establishes the second claim. \newline

	\noindent \textit{Claim 3}.
	If $\CScollapse_\ast(x) \neq 0$ and $\CScollapse_\ast(y) \neq 0$ then \eqref{e:cs collapse as algebra map} holds.

	Let $q_y$ be the tensor position of the first occurrence of the tensor factor $[0,1]$ in $y$, setting it to $+\infty$, if not present.
	If $p_x > q_y$ then $x \ast y = 0$ since $\varepsilon(y_{<p_x}) = 0$, and if $p_x = q_y$ then $x \ast y = 0$ since $[0] \ast [0,1] = 0$.
	We now prove that in this case $\CScollapse_\ast(x) \ast \CScollapse_\ast(y) = 0$.
	Given that $x_{q_y} = [0,1]$ since $x \leq y$ and $x_{p_x}$ is the first tensor factor equal to $[0]$.
	This implies that both $\CScollapse_\ast(x)$ and $\CScollapse_\ast(y)$ contain the vertex $q_{y} - 1$.
	Let us now assume that $p_x < q_y$.
	Then, using that $y_i = [1]$ for every $i < p_x$ and the second claim we have
	\begin{align*}
	\CScollapse_\ast(x \ast y) & =
	\CScollapse_\ast \big( v_x \otimes x_{p_x} \ast y_{p_x} \otimes w_y \big) \\ & =
	\CScollapse_\ast(x) \ast \CScollapse_\ast(y)
	\end{align*}
	as desired.
\end{proof}

%\begin{proof}
%	\textit{Claim 1}. If $\CScollapse_\ast(x) = 0$ or $\CScollapse_\ast(y) = 0$ then
%	\begin{equation} \label{e:zero for join}
%	\CScollapse_\ast \big( x_{<i}\, \epsilon(y_{<i}) \otimes x_i \ast y_i \otimes \epsilon(x_{>i}) \, y_{>i} \big) = 0.
%	\end{equation}
%	Assume $\CScollapse_\ast(x) = 0$, that is, there exists a pair $p < q$ such that $x_p = [0]$ and $x_q = [0,1]$, then \eqref{e:zero for join} holds since:
%	\begin{enumerate}
%		\item If $i > q$, then $x_p$ and $x_q$ are part of $x_{<i}$.
%		\item If $i = q$, then $x_q \ast y_q = 0$ for any $y_q$.
%		\item If $i < q$, then $\varepsilon(x_{>i}) = 0$.
%	\end{enumerate}
%	Similarly, if there is a pair $p < q$ such that $y_p = [0]$ and $y_q = [0,1]$,  then \eqref{e:zero for join} holds since:
%	\begin{enumerate}
%		\item If $i < p$, then $y_p$ and $y_q$ are part of $y_{>i}$.
%		\item If $i = p$ or, more generally, $y_i = [0]$, then $x_i = [0]$ and $x_i \ast y_i = 0$.
%		\item If $i = q$ or, more generally, $y_i = [0,1]$, then $x_i \ast y_i = 0$ for any $x_i$.
%		\item If $i > q$, then $\varepsilon(y_{<i}) = 0$.
%		\item If $p < i < q$ and $y_i = [1]$ then either $x_i \ast x_j = 0$ or $x_i \ast x_j = [0,1]$, implying $(x \ast y)_p = [0]$ and $(x \ast y)_i = [0,1]$.
%	\end{enumerate}
%	\textit{Claim 2}. If $\CScollapse_\ast(x) \neq 0$ and $\CScollapse_\ast(y) \neq 0$ then
%	\[
%	\CScollapse_\ast(x \ast y) =
%	\CScollapse_\ast(x) \ast \CScollapse_\ast(y).
%	\]
%	Given these assumptions there exist basis elements $v_x, v_y$ having tensor factors in $\{[0,1], [1]\}$, and $w_x, w_y$ with tensor factors in $\{[0], [1]\}$ such that
%	\[
%	x = v_x \otimes [0] \otimes w_x, \qquad
%	y = v_y \otimes [0] \otimes w_y.
%	\]
%	Let $p_x$ and $p_y$ be the tensor position of the first $[0]$ in $x$ and $y$ respectively.
%	To make sure all possible elements $x$ and $y$ have a factor $[0]$ in them we consider $\chains(\cube^n)$ as a subcomplex of $\chains(\cube^{n+1})$ via the inclusion that tensors on the right with $[0]$.
%	By naturality, we do not loose generality making this assumption.
%	Since $x \leq y$ we have that $p_x \leq p_y$.
%	We will now show that
%	\[
%	\CScollapse_\ast(x \ast y) = \CScollapse_\ast \big( x_{<p_x} \, \varepsilon(y_{<p_x}) \otimes \, x_{p_x} \! \ast y_{p_x} \otimes \varepsilon(x_{>p_x}) \, y_{>p_x} \big).
%	\]
%	To see this we notice that if $i < p_x$ then $x_i = [1]$ or $x_i = [0,1]$.
%	If $x_i = [1]$, since $x_i \leq y_i$, it is impossible for $y_i = [0]$, the only case when $x_i \ast y_i \neq 0$.
%	If $x_i = [0,1]$ then $x_i \ast y_i = 0$ for any $y_i$.
%	If $i > p_x$, then either $x_i \ast y_i = 0$ or $x_i \ast y_i = [0,1]$.
%	In the first case there is nothing to prove and in the second we notice that $(x \ast y)_{p_x} = [0]$ and $(x \ast y)_{} = [0,1]$ so $\CScollapse_\ast(x \ast y) = 0$.
%
%	Let $q_y$ be the tensor position of the first occurrence of the tensor factor $[0,1]$ in $y$, setting it to $+\infty$, if not present.
%	If $p_x > q_y$ then $x \ast y = 0$ since $\varepsilon(y_{<p_x}) = 0$, and if $p_x = q_y$ then $x \ast y = 0$ since $[0] \ast [0,1] = 0$.
%	We now prove that in this case $\CScollapse_\ast(x) \ast \CScollapse_\ast(y) = 0$.
%	Since $x_{q_y} = [0,1]$ since $x \leq y$ and $x_{p_x}$ is the first tensor factor equal to $[0]$.
%	This implies that both $\CScollapse_\ast(x)$ and $\CScollapse_\ast(y)$ contain the vertex $q_{y} - 1$.
%	Let us now assume that $p_x < q_y$.
%	Then, using that $y_i = [1]$ for every $i < p_x$,
%	\begin{align*}
%	\CScollapse_\ast(x \ast y) & =
%	\CScollapse_\ast \big( v_x \otimes x_{p_x} \ast y_{p_x} \otimes w_y \big) \\ & =
%	\CScollapse_\ast(x) \ast \CScollapse_\ast(y)
%	\end{align*}
%	as claimed.
%\end{proof}

\begin{proof}[Proof of \cref{l:main comparison}]
	This sequence of lemmas provides a proof of \cref{l:main comparison} using the decomposition of any surjection-like graph into pieces \coproduct \ and \product.
\end{proof}

\section{Proof of \cref{t:cubical chain bialgebra}} \label{s:proof}

We need to show that the assignment
\begin{equation*}
\counit \mapsto \epsilon, \quad \coproduct \mapsto \Delta, \quad \product \mapsto \ast,
\end{equation*}
defined in \cref{s:action} is compatible with the relations
\begin{equation*}
\productcounit = 0,
\qquad
\leftcounitality = 0,
\qquad
\rightcounitality = 0,
\end{equation*}
and
\begin{equation*}
\partial\ \counit = 0,
\hspace*{.6cm}
\partial\ \coproduct = 0,
\hspace*{.6cm}
\partial\ \product = \ \boundary\,.
\end{equation*}
For the rest of this section let us consider two basis elements of $N_\bullet(\square^n) = N_\bullet(\square^1)^{\otimes n}$
\begin{align*}
x = x_1 \otimes \cdots \otimes x_n
\qquad \text{ and } \qquad
y = y_1 \otimes \cdots \otimes y_n.
\end{align*}
Since the degree of $\ast$ is $1$ and $\epsilon([0,1]) = 0$, we can verify the first relation easily:
\begin{align*}
\varepsilon(x \otimes y) & =
\sum (-1)^{|x|} \epsilon(y_{<i}) \epsilon(x_{<i}) \otimes \epsilon(x_i \ast y_i) \otimes \epsilon(x_{>i}) \epsilon(y_{>i}) = 0.
\end{align*}
For the second relation we want to show that $(\epsilon \otimes \id) \circ \Delta = \id$.
Since
\begin{gather*}
(\epsilon \otimes \id) \circ \Delta([0]) = \epsilon([0]) \otimes [0] = [0], \qquad
(\epsilon \otimes \id) \circ \Delta([1]) = \epsilon([1]) \otimes [1] = [1], \\
(\epsilon \otimes \id) \circ \Delta([0, 1]) = \epsilon([0]) \otimes [0, 1] + \epsilon([0, 1]) \otimes [1] = [0,1],
\end{gather*}
we have
\begin{align*}	
(\epsilon \otimes \id) \circ \Delta (x_1 \otimes \cdots \otimes x_n) &=
\sum \pm \left( \epsilon \big(x_1^{(1)}\big) \otimes \cdots \otimes \epsilon\big(x_n^{(1)}\big) \right) \otimes 	
\left( x_1^{(2)} \otimes \cdots \otimes x_n^{(2)} \right), \\ &=
x_1 \otimes \cdots \otimes x_n,
\end{align*}	
where the sign is obtained by noticing that the only non-zero term occurs when each factor $x_i^{(0)}$ is of degree $0$.
The third relation is verified analogously.
The fourth is precisely the fact that $\epsilon$ is a chain maps.
We will verify that $\Delta$ is a chain map recalling two facts: The map $\Delta$ is equal to the composition
\begin{equation*}
\begin{tikzcd}
\cchains(\square^1)^{\otimes n} \arrow[r, "\Delta^{\otimes n}"] & \left( \cchains(\square^1)^{\otimes 2}  \right)^{\otimes n} \arrow[r, "sh"] & \left( \cchains(\square^1)^{\otimes n} \right)^{\otimes 2},
\end{tikzcd}
\end{equation*}
and the tensor product of chain maps is a chain map.
Therefore, since
\begin{gather*}
\partial (\Delta)([0]) = \partial ([0] \otimes [0]) = 0, \qquad
\partial (\Delta)([1]) = \partial ([1] \otimes [1]) = 0, \\
\partial (\Delta)([0,1]) = \partial \big([0] \otimes [0,1] + [0,1] \otimes [1]\big) - [1] \otimes [1] + [0] \otimes [0] = 0,
\end{gather*}
the fifth relation follows.
To verify the sixth and final relation we need to show that
\begin{equation*}
\partial (x \ast y)\ +\ \partial x \ast y\ +\ (-1)^{|x|}x \ast \partial y\ =\ \epsilon(x) y \ -\ \epsilon(y) x.
\end{equation*}
We have
\begin{equation*}
x \ast y = \sum (-1)^{|x|} x_{<i} \, \epsilon(y_{<i}) \otimes x_i \ast y_i \otimes \epsilon(x_{>i})\, y_{>i}
\end{equation*}
and
\begin{align*}
\partial(x \ast y) & = 
\sum (-1)^{|x|} \, \partial x_{<i}\, \epsilon(y_{<i}) \otimes x_i \ast y_i \otimes \epsilon(x_{>i})\, y_{>i} \\ & +
\sum (-1)^{|x|+|x_{<i}|} \, x_{<i}\, \epsilon(y_{<i}) \otimes \partial (x_i \ast y_i) \otimes \epsilon(x_{>i}) \, y_{>i} \\ & -
\sum (-1)^{|x|+|x_{<i}|} \, x_{<i}\, \epsilon(y_{<i}) \otimes x_i \ast y_i \otimes \epsilon(x_{>i})\, \partial y_{>i}.
\end{align*}
Since
\begin{equation*}
|x| = |x_{<i}| + |x_i| + |x_{>i}|, \quad \epsilon(x_{>i}) \neq 0 \Leftrightarrow |x_{>i}| = 0, \quad \partial(x_i \ast y_i) \neq 0 \Rightarrow |x_i| = 0,
\end{equation*}
we have
\begin{equation} \label{e:boundary of product 1}
\begin{split}
\partial(x \ast y) & = 
\sum (-1)^{|x|} \, \partial x_{<i}\, \epsilon(y_{<i}) \otimes x_i \ast y_i \otimes \epsilon(x_{>i})\, y_{>i} \\ & +
\sum x_{<i} \, \epsilon(y_{<i}) \otimes \partial (x_i \ast y_i) \otimes \epsilon(x_{>i})\, y_{>i} \\ & -
\sum x_{<i} \, \epsilon(y_{<i}) \otimes x_i \ast y_i \otimes \epsilon(x_{>i})\, \partial y_{>i}.
\end{split}
\end{equation}
We also have
\begin{align*}
\partial x \ast y & = 
\sum (-1)^{|x|-1} \, \partial x_{<i}\, \epsilon(y_{<i}) \otimes x_i \ast y_i \otimes \epsilon(x_{>i}) \, y_{>i} \\ & +
\sum (-1)^{|x|-1+|x_{<i}|} \, x_{<i}\, \epsilon(y_{<i}) \otimes \partial x_i \ast y_i \otimes \epsilon(x_{>i}) \, y_{>i} \\ & +
\sum (-1)^{|x|-1+|x_{<i}|} \, x_{<i}\, \epsilon(y_{<i}) \otimes x_i \ast y_i \otimes \epsilon(\partial x_{>i}) \, y_{>i}.
\end{align*}
Since
\begin{equation*}
\epsilon(\partial x_{>i}) = 0, \quad \partial x_i \neq 0 \Leftrightarrow |x_i| = 1,
\end{equation*}
we have
\begin{equation} \label{e:boundary of product 2}
\begin{split}
\partial x \ast y & = 
\sum (-1)^{|x|-1} \, \partial x_{<i}\, \epsilon(y_{<i}) \otimes x_i \ast y_i \otimes \epsilon(x_{>i})\, y_{>i} \\ & +
\sum x_{<i}\, \epsilon(y_{<i}) \otimes \partial x_i \ast y_i \otimes \epsilon(x_{>i})\, y_{>i}.
\end{split}
\end{equation}
We also have
\begin{align*}
(-1)^{|x|} \, x \ast \partial y & = 
\sum x_{<i} \, \epsilon(\partial y_{<i}) \otimes x_i \ast y_i \otimes \epsilon(x_{>i})\, y_{>i} \\ & +
\sum (-1)^{|y_{<i}|} \, x_{<i}\, \epsilon(y_{<i}) \otimes x_i \ast \partial y_i \otimes \epsilon(x_{>i}) \, y_{>i} \\ & +
\sum (-1)^{|y_{<i}| + |y_i|} \, x_{<i}\, \epsilon(y_{<i}) \otimes x_i \ast y_i \otimes \epsilon(x_{>i}) \, \partial y_{>i},
\end{align*}
which is equivalent to
\begin{equation} \label{e:boundary of product 3}
\begin{split}
(-1)^{|x|} \, x \ast \partial y & = 
\sum x_{<i} \, \epsilon(y_{<i}) \otimes x_i \ast \partial y_i \otimes \epsilon(x_{>i})\, y_{>i} \\ & +
\sum x_{<i}\, \epsilon(y_{<i}) \otimes x_i \ast y_i \otimes \epsilon(x_{>i})\, \partial y_{>i}.
\end{split}
\end{equation}
Putting identities \eqref{e:boundary of product 1}, \eqref{e:boundary of product 2} and \eqref{e:boundary of product 3} together, we get
\begin{align*}
\partial (x \otimes y) \ +\ & \partial x \ast y\ +\, (-1)^{|x|}x \ast \partial y \\
& = \sum \epsilon(y_{<i})\, x_{<i} \otimes \big(\partial(x_i \ast y_i) + \partial x_i \ast y_i + x_i \ast \partial y_i\big) \otimes \epsilon(x_{>i})\, y_{>i}.
\end{align*}
Since
\begin{align*}
\partial(x_i \ast y_i)\ +\ \partial x_i \ast y_i\ +\ x_i \ast \partial y_i =
\epsilon(x_i)y_i\ -\ \epsilon(y_i)x_i,
\end{align*}
we have
\begin{align*}
\partial (x \ast y) \ +\ \partial x \ast y\ +\ & (-1)^{|x|}x \ast \partial y \\ = \ &
\sum \epsilon(y_{<i}) \, x_{<i} \otimes \epsilon(x_{\geq i}) y_{\geq i}\ -\
\epsilon(y_{\leq i}) \, x_{\leq i} \otimes \epsilon(x_{>i}) y_{>i} \\ = \ &
\epsilon(x)y - \epsilon(y)x,
\end{align*}
as desired, where the last equality follows from a telescopic sum argument.

\section{Proof of \cref{l:main comparison lemma}} \label{s:comparison proof}

Let $\psi \colon \chains(\cube^n) \to \chains(\simplex^n)$ denote the composition
\begin{equation*}
\begin{tikzcd}[column sep=normal]
\chains(\cube^n) \arrow[r, "\chains(\T)"] &[-3pt]
\chains \big( \scube{n} \big) \arrow[r, "\chains(s)"] &[-3pt]
\chains \simplex^n,
\end{tikzcd}
\end{equation*}
which agrees with the chain map induced by the Cartan-Serre map $\gcube^n \to \gsimplex^n$.

We need to show the commutativity of the diagram
\begin{equation*}
\begin{tikzcd}
U(\MS)(r) \otimes \chains(\cube^n) \arrow[r] \arrow[d, "\id\, \otimes\, \psi"'] &
\cchains(\cube^n)^{\otimes r} \arrow[d, "\psi^{\otimes r}"] \\
U(\MS)(r) \otimes \cchains(\simplex^n) \arrow[r] &
\cchains(\simplex^n)^{\otimes r}.
\end{tikzcd}
\end{equation*}
By naturality, it suffices to show that
\begin{equation} \label{e:Cartan-Serre E-coalgebra map}
\psi^{\otimes r} \circ \Gamma\big( [0,1]^{\otimes n} \big) = \Gamma \circ \psi \big( [0,1]^{\otimes n} \big),
\end{equation}
where $\Gamma$ is represented by a surjection-like graph, a generator of $U(\MS)$.
Let us start by verifying that $\psi$ is a morphism of counital coalgebras.

\begin{lemma}
	If $\Gamma =$ \counit \ or \coproduct \ then identity \eqref{e:Cartan-Serre E-coalgebra map} holds.
\end{lemma}

\begin{proof}
	Since vertices are sent to vertices the map $\psi$ is a map of augmented chain complexes, establishing the case $\Gamma =$ \counit \ .
	To study the compatibility of $\psi$ with \coproduct \, notice that for $n=0$ identity \eqref{e:Cartan-Serre E-coalgebra map} holds trivially.
	For $n > 0$,
	\begin{equation*}
	\Delta([0,1]^{\otimes n}) = \sum_{\lambda \in \lambda} \pm \ x_1^{(\lambda)} \otimes \cdots \otimes x_n^{(\lambda)} \bm{\otimes} y_1^{(\lambda)} \otimes \cdots \otimes y_n^{(\lambda)},
	\end{equation*}
	where $\Lambda$ parameterizes all choices of $x_i^{(\lambda)} \in \{[0], [0,1]\}$ and $y_i^{(\lambda)} \in \{[0,1], [1]\}$ such that
	\begin{align*}
	x_i^{(\lambda)} = [0]   & \iff y_i^{(\lambda)} = [0,1], \\
	x_i^{(\lambda)} = [0,1] & \iff y_i^{(\lambda)} = [1].
	\end{align*}
	The only pairs that are not sent to $0$ by $\psi \otimes \psi$ are those for which $x_i^{(\lambda)} = [0]$ implies $x_j^{(\lambda)} = [0]$ for all $i < j$, for which the sign is positive, and their image is $[0,\dots,k] \otimes [k,\dots,n]$ where $k+1 = \min \{i \mid x_i^{(\lambda)} = [0]\}$ or $k = n$ if empty.
\end{proof}

We will consider the basis of $\cchains(\cube^n)$ as a poset defined as follows:
For $n = 1$ we set $[0] < [0,1] < [1]$, and for $n > 1$ we have $(x_1 \otimes \cdots \otimes x_n) \leq (y_1 \otimes \cdots \otimes y_n)$ if $x_i \leq y_i$ for each $i \in \{1, \dots, n\}$. 

\begin{lemma}
	Let
	\begin{equation*}
	\begin{tikzpicture}[scale=.35]
	\node at (9.5, 1){$\Gamma \, = $};
	\draw (11,.5)--(12,1.5)--(12,2.5);
	\draw (13,.5)--(12,1.5);
	\node at (11,0){$\scriptstyle 1$};
	\node at (12,0){$\scriptstyle \dots$};
	\node at (13,0){$\scriptstyle r$};
	\end{tikzpicture}
	\end{equation*}
	be the surjection-like graph representing the $(r-1)\th$ iterated Serre diagonal.
	If
	\begin{equation*}	
	\Gamma\big([0,1]^{\otimes n}\big) =
	\sum \pm \ x^{(1)} \bm{\otimes} \cdots \bm{\otimes} x^{(r)}
	\end{equation*}
	with each $x^{(k)} \in \cchains(\cube^n)$ a basis element, then $x^{(1)} \leq \cdots \leq x^{(r)}$.
\end{lemma}

\begin{proof}
	The claim follows from definition after inspecting
	\begin{equation*}
	\Delta([0]) = [0] \otimes [0], \quad \Delta([1]) = [1] \otimes [1], \quad \Delta([0, 1]) = [0] \otimes [0, 1] + [0, 1] \otimes [1].
	\end{equation*}
\end{proof}

\begin{lemma}
	Let $x, y, z \in \cchains(\cube^n)$ be basis elements.
	If $x, y \leq z$ then $(x \ast y) \leq z$.
\end{lemma}

\begin{proof}
	Recall that
	\begin{align*}
	(x_1 \otimes \cdots \otimes x_n) \ast (y_1 \otimes \cdots \otimes y_n) =
	(-1)^{|x|} \sum_{i=1}^n x_{<i}\, \epsilon(y_{<i}) \otimes x_i \ast y_i \otimes \epsilon(x_{>i}) \, y_{>i}.
	\end{align*}
	By assumption, for every $i$ we have $x_{<i} \leq z_{<i}$ and $y_{>i} \leq z_{>i}$, and if $x_i \ast y_i \neq 0$ then either $x_i = [1]$ or $y_i = [1]$ which implies $z_i = [1]$ as well.
\end{proof}

\begin{lemma}
	Let $x, y \in \cchains(\cube^n)$ be basis elements.
	If $x \leq y$ then $\psi(x \ast y) = \psi(x) \ast \psi(y)$.
\end{lemma}

\begin{proof}
	If $\psi(x) = 0$ or $\psi(y) = 0$ we will show that
	\begin{equation} \label{e:zero for join}
	\psi \big( x_{<i}\, \epsilon(y_{<i}) \otimes x_i \ast y_i \otimes \epsilon(x_{>i}) \, y_{>i} \big) = 0.
	\end{equation}
	Assume $\psi(x) = 0$, that is, there exists a pair $p < q$ such that $x_p = [0]$ and $x_q = [0,1]$, then \eqref{e:zero for join} holds since:
	\begin{itemize}
		\item If $i > q$, then $x_p$ and $x_q$ are part of $x_{<i}$.
		\item If $i = q$, then $x_q \ast y_q = 0$ for any $y_q$.
		\item If $i < q$, then $\varepsilon(x_{>i}) = 0$.
	\end{itemize}
	Similarly, if there is a pair $p < q$ such that $y_p = [0]$ and $y_q = [0,1]$,  then \eqref{e:zero for join} holds since:
	\begin{itemize}
		\item If $i < p$, then $y_p$ and $y_q$ are part of $y_{>i}$.
		\item If $i = p$ or, more generally, $y_i = [0]$, then $x_i = [0]$ and $x_i \ast y_i = 0$.
		\item If $i = q$ or, more generally, $y_i = [0,1]$, then $x_i \ast y_i = 0$ for any $x_i$.
		\item If $i > q$, then $\varepsilon(y_{<i}) = 0$.
		\item If $p < i < q$ and $y_i = [1]$ then either $x_i \ast x_j = 0$ or $x_i \ast x_j = [0,1]$, implying $(x \ast y)_p = [0]$ and $(x \ast y)_i = [0,1]$.
	\end{itemize}
	Let us now assume that $\psi(x) \neq 0$ and $\psi(y) \neq 0$.
	In particular, $x = v_x \otimes [0] \otimes w_x$ and $y = v_y \otimes [0] \otimes w_y$ with $v_x, v_y$ having tensor factors in $\{[0,1], [1]\}$, and $w_x, w_y$ in $\{[0,1], [1]\}$, we also admit them to be the unit of the tensor product.
	Additionally, let $p_x$ and $p_y$ be the tensor position of the first $[0]$ in $x$ and $y$ respectively.
	To make sure all possible elements $x$ and $y$ have a factor $[0]$ in them we consider $\chains(\cube^n)$ as a subcomplex of $\chains(\cube^{n+1})$ via the inclusion that tensors on the right with $[0]$.
	By naturality, we do not loose generality making this assumption.
	Since $x \leq y$ we have that $p_x \leq p_y$.
	We claim that
	\begin{equation*}
	\psi(x \ast y) = \psi \big( x_{<p_x} \, \varepsilon(y_{<p_x}) \otimes x_{p_x} \ast y_{p_x} \otimes \varepsilon(x_{>p_x}) \, y_{>p_x} \big).
	\end{equation*}
	To see this we notice that if $i < p_x$ then $x_i = [1]$ or $x_i = [0,1]$.
	If $x_i = [1]$, since $x_i \leq y_i$, it is impossible for $y_i = [0]$, the only case when $x_i \ast y_i \neq 0$.
	If $x_i = [0,1]$ then $x_i \ast y_i = 0$ for any $y_i$.
	If $i > p_x$, then either $x_i \ast y_i = 0$ or $x_i \ast y_i = [0,1]$.
	In the first case there is nothing to prove and in the second we notice that $(x \ast y)_{p_x} = [0]$ and $(x \ast y)_{} = [0,1]$ so $\psi(x \ast y) = 0$.
	Let $q_y$ the tensor position of the first occurrence of the tensor factor $[0,1]$ in $y$, setting it to $+\infty$, if not present.
	If $p_x > q_y$ then $x \ast y = 0$ since $\varepsilon(y_{<p_x}) = 0$, and if $p_x = q_y$ then $x \ast y = 0$ since $[0] \ast [0,1] = 0$.
	We now prove that in this case $\psi(x) \ast \psi(y) = 0$.
	Since $x_{q_y} = [0,1]$ since $x \leq y$ and $x_{p_x}$ is the first tensor factor equal to $[0]$.
	This implies that both$\psi(x)$ and $\psi(y)$ contain the vertex $q_{y} - 1$.
	Let us now assume that $p_x < q_y$.
	Then, using that $y_i = [1]$ for every $i < p_x$,
	\begin{align*}
	\psi(x \ast y) & =
	\psi \big( v_x \otimes x_{p_x} \ast y_{p_x} \otimes w_y \big) \\ & =
	\psi(x) \ast \psi(y).
	\end{align*}
\end{proof}







%We use triples $(F_0, F_{01}, F_1)$ of subsets of $\{0, \dots, n\}$ to represent faces $F$ of $\gcube^n$ with $F_\epsilon = \{i \mid \forall x \in F, \, x_i = \epsilon\}$ for $\epsilon \in \{0,1\}$ and $F_{01} = \{0, \dots, n\} \setminus F_0 \cup F_1$.

%Given a subset $U = \{u_1 < \cdots < u_q\}$ of $\{0, \dots, n\}$ we write $d_U$ for $d_{u_1} \! \cdots \, d_{u_q}$.
%In this notation, any face of $\gsimplex^n$ can be written as $d_U [0, \dots, n]$ for some $U$.

%With this notation we have the following description of the chain map $\psi$.

%\begin{lemma}
%	On basis elements $\psi \colon \cchains(\cube^n) \to \chains(\triangle^n)$ is given by
%	\begin{equation*}
%	(F_0, F_{01}, F_1) \mapsto
%	\begin{cases}
%	d_U [0, \dots, n] & \text{ if } F_{01} \cap \{i \mid i > \min(F_0)\} = \emptyset, \\
%	0 & \text{ otherwise},
%	\end{cases}
%	\end{equation*}
%	where $U = \{i-1 \mid i \in F_1 \text{ or } i > \min(F_0)\}$ with the convention $\min(\emptyset) = +\infty$.
%\end{lemma}
%
%\begin{proof}
%	We assume $n > 0$ since otherwise there is nothing to prove.
%	Consider a face $(F_0, F_{01}, F_1)$ and let $M \subseteq \{0, \dots, n\}$ be empty if $F_0$ is empty or be characterized by $i > \min (F_0)$ otherwise.
%	Notice in \eqref{e:cartan-serre CW map} that if $x_i = 0$ then $y_j = 0$ for every $j > i$.
%	Therefore, the image of $(F_0, F_{01}, F_1)$ in $\simplex^n$ is a face of $d_U[0, \dots, n]$ where $U = \{i-1 \mid i \in M\}$ or, more explicitly, $[0, \dots, n]$ if $F_0$ is empty and $[0, \dots, \min(F_0)]$ otherwise.
%	In particular, $S(F_0, F_{01}, F_1)$ is non-zero only if $M \cap F_{01} = \emptyset$, and we can assume without loss of generality that $F_0 = \emptyset$ or $F_0 = \{n\}$.
%	
%	Notice from \eqref{e:cartan-serre CW map} that $x_i = 1$ if and only if $y_{i-1} = 0$, so $S(\emptyset, F_{01}, F_1) = d_{U} [0, \dots, n]$ where $U = \{i-1 \mid i \in F_1\}$.
%\end{proof}

%\begin{definition}
%	A basis element $x_1 \otimes \cdots \otimes x_n \in \cchains(\cube^n)$ is said to be  \textit{vertex-ordered} if $x_i = [0]$ and $x_j = [1]$ imply $i < j$.
%\end{definition}


%\begin{lemma}
%	Let
%	\begin{equation*}
%	\begin{tikzpicture}[scale=.35]
%	\node at (9.5, 1){$\Gamma \, = $};
%	\draw (11,.5)--(12,1.5)--(12,2.5);
%	\draw (13,.5)--(12,1.5);
%	\node at (11,0){$\scriptstyle 1$};
%	\node at (12,0){$\scriptstyle \dots$};
%	\node at (13,0){$\scriptstyle r$};
%	\end{tikzpicture}
%	\end{equation*}
%	be the surjection-like graph representing the $(r-1)\th$ iterated Serre diagonal, and $x_1 \otimes \cdots \otimes x_n \in \cchains(\cube^n)$ a basis element.
%	If
%	\begin{equation*}	
%	\Gamma \big( x_1 \otimes \cdots \otimes x_n \big) =
%	\sum \pm \left( x_1^{(1)} \otimes \cdots \otimes x_n^{(1)} \right)
%	\otimes \cdots \otimes
%	\left( x_1^{(r)} \otimes \cdots \otimes x_n^{(r)} \right)
%	\end{equation*}
%	with each $x_i^{(k)}$ a basis element of $\cchains(\cube^1)$, then
%	\begin{enumerate}
%		\item For every $i \in \{1, \dots, n\}$ the element $x_i^{(1)} \otimes \cdots \otimes x_i^{(r)}$ is vertex-ordered.
%	
%	 	\item For every $k \in \{1, \dots, r\}$ if $x_1 \otimes \cdots \otimes x_n$ is vertex-ordered then $x_1^{(k)} \otimes \cdots \otimes x_n^{(k)}$ is vertex-ordered.
%	\end{enumerate}
%\end{lemma}

\bibliographystyle{alpha} % ieeetr
\bibliography{bibliography}

\end{document}