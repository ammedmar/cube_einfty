
\section{$\M$-bialgebra on cubes} \label{s:coaction}

Let $C$ be a chain complex, $\O$ an operad, and $\P$ a prop.
An $\O$-\textit{coalgebra} (resp. $\O$-\textit{algebra} or $\P$-\textit{bialgebra}) structure on $C$ is a structure preserving morphism $\O \to \End^C$ (resp. $\O \to \End_C$ or $\P \to \End_C^C$).

In this section we construct a natural $\M$-bialgebra structure on the chains of standard cubes $\cchains(\cube^n)$.
These are determined by three linear maps satisfying the relations in the presentation of $\mathcal M$.
For $n \in \mathbb{N}$, define: \vspace*{5pt} \\
(1) The counit $\epsilon \in \Hom(\cchains(\square^n), \Z)$ known as the \textit{augmentation} by
\begin{equation*}
\epsilon \left( x_1 \otimes \cdots \otimes x_d \right) = \epsilon(x_1) \cdots \, \epsilon(x_n),
\end{equation*}
where
\begin{equation*}
\epsilon([0]) = \epsilon([1]) = 1, \qquad \epsilon([0, 1]) = 0.
\end{equation*} \vspace*{-6pt} \\
(2) The coproduct $\Delta \in \Hom \left( \cchains(\square^n), \cchains(\square^n)^{\otimes 2} \right)$ known as the \textit{Serre diagonal} by
\begin{equation*}	
\Delta (x_1 \otimes \cdots \otimes x_n) = 	
\sum \pm \left( x_1^{(1)} \otimes \cdots \otimes x_n^{(1)} \right) \otimes 	
\left( x_1^{(2)} \otimes \cdots \otimes x_n^{(2)} \right),	
\end{equation*}	
where the sign is determined using the Koszul convention, and we are using Sweedler's notation
\begin{equation*}	
\Delta(x_i) = \sum x_i^{(1)} \otimes x_i^{(2)}
\end{equation*}
for the chain map $\Delta \colon \cchains(\square^1) \to \cchains(\square^1)^{\otimes 2}$ defined by
\begin{equation*}
\Delta([0]) = [0] \otimes [0], \quad \Delta([1]) = [1] \otimes [1], \quad \Delta([0, 1]) = [0] \otimes [0, 1] + [0, 1] \otimes [1].
\end{equation*}
Using that $\cchains(\square^n) = \cchains(\square^1)^{\otimes n}$, $\Delta$ is the composition
\begin{equation*}
\begin{tikzcd}
\cchains(\square^1)^{\otimes n} \arrow[r, "\Delta^{\otimes n}"] & \left( \cchains(\square^1)^{\otimes 2}  \right)^{\otimes n} \arrow[r, "sh"] & \left( \cchains(\square^1)^{\otimes n} \right)^{\otimes 2}
\end{tikzcd}
\end{equation*}
where $sh$ is the shuffle map that places tensor factors in odd position first. \vspace*{5pt} \\
(3) The product $\ast \in \Hom(\cchains(\square^n)^{\otimes 2}, \cchains(\square^n))$ by
\begin{align*}
(x_1 \otimes \cdots \otimes x_n) \ast (y_1 \otimes \cdots \otimes y_n) =
(-1)^{|x|} \sum_{i=1}^n x_{<i}\, \epsilon(y_{<i}) \otimes x_i \ast y_i \otimes \epsilon(x_{>i}) \, y_{>i},
\end{align*}
where
\begin{align*}
x_{<i} & = x_1 \otimes \cdots \otimes x_{i-1}, &
y_{<i} & = y_1 \otimes \cdots \otimes y_{i-1}, \\
x_{>i} & = x_{i+1} \otimes \cdots \otimes x_n, & 
y_{>i} & = y_{i+1} \otimes \cdots \otimes y_n,
\end{align*}
with the convention
\begin{equation*}
x_{<1} = y_{<1} = x_{>n} = y_{>n} = 1 \in \Z,
\end{equation*}
and the only non-zero values of $x_i \ast y_i$ are
\begin{equation*}
\ast([0] \otimes [1]) = [0, 1], \qquad  \ast([1] \otimes [0]) = -[0, 1].
\end{equation*}

We devote Section~\ref{s:proof} to the proof of the following statement, our main result.

\begin{theorem} \label{t:cubical chain bialgebra}
	For every $n \in \mathbb{N}$, the assignment
	\begin{equation*}
	\counit \mapsto \epsilon, \quad \coproduct \mapsto \Delta, \quad \product \mapsto \ast,
	\end{equation*}
	defines a natural $\mathcal M$-bialgebra structure on $\cchains(\square^n)$.
\end{theorem}

The category of bialgebras over a prop is in general not cocomplete, but those of algebras and coalgebras over operads are.
Using the forgetful functor $U$ we consider the previous construction as a functor $\cube \to \coAlg_{U(\M)}$ and use a Kan extension along the Yoneda embedding to define a lift of the functor of chains:
\begin{equation*}
\begin{tikzcd}
& \coAlg_{U(\M)} \arrow[d] \\
\cSet \arrow[r] \arrow[dashed, ur, "N", bend left]& \Ch.
\end{tikzcd}
\end{equation*}
Similarly, we can define a lift to $\Alg_{U(\M)}$ of the functor of cochains $\Hom(-, R) \circ \cchains$.

