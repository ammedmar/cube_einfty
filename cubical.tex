
\section{Conventions and preliminaries}

Throughout this article $R$ denotes a commutative and unital ring and we work over its associated symmetric monoidal category $(\Ch, \otimes, R)$ of chain complexes.
We denote the chain complex of $R$-linear maps between chain complexes $C$ and $C^\prime$ by $\Hom(C, C^\prime)$.
The functor $\Hom(-, R)$ is referred to as \textit{linear duality}.

Recall that a category is said to be \textit{small} if its objects and morphisms form sets. A category is said to be \textit{cocomplete} if any functor to it from a small category has a colimit.
If $\mathsf{A}$ is small and $\mathsf{C}$ cocomplete, then the (left) \textit{Kan extension of $g$ along $f$} exists for any pair of functors
\begin{equation*}
\begin{tikzcd}[column sep=normal, row sep=normal]
\mathsf{A} \arrow[d, "f"'] \arrow[r, "g"] & \mathsf{C} \\ 
\mathsf{B} \arrow[dashed, ur, bend right] & \quad .
\end{tikzcd}
\end{equation*}
Given categories $\mathsf{B}$ and $\C$, we denote the \textit{functor category} between them as $\Fun(\mathsf{B}, \C)$.

\section{Cubical topology}

Let $2^0$ and $2^1$ denote respectively the sets $\{0\}$ and $\{0,1\}$.
The \textit{cube category} $\cube$ is the subcategory of $\Set$ with objects $2^n$, together with the maps $f \colon 2^m \to 2^n$ which delete some coordinates and insert some 0’s and 1’s (without modifying the order of the remaining coordinates).
Explicitly, its morphisms are generated by the \textit{coface} and \textit{codegeneracy maps} defined by
\begin{align*}
\delta_i^\varepsilon & = \mathrm{id}_{2^{i-1}} \times \delta^\varepsilon \times \mathrm{id}_{2^{n-1-i}} \colon 2^{n-1} \to 2^n, \\
\sigma_i & = \mathrm{id}_{2^{i-1}} \times \, \sigma \times \mathrm{id}_{2^{n-i}} \colon 2^{n} \to 2^{n-1}.
\end{align*}
where $\varepsilon \in \{0,1\}$ and the maps
\begin{equation*}
\begin{tikzcd}
2^0 \arrow[r, bend left, "\delta^0"] \arrow[r, bend right, "\delta^1"'] & 2^1 \arrow[r, "\sigma"] & 2^0
\end{tikzcd}
\end{equation*}
are defined by
\begin{equation*}
\delta^0(0) = 0, \qquad \delta^1(0) = 1, \qquad \sigma(0) = \sigma(1) = 0.
\end{equation*}

We denote by $\cube_{\deg}$ the subcategory with the same objects as $\cube$ and morphisms of the form $\sigma_i \circ \tau$ for some morphisms $\tau$ of $\cube$.

The category of \textit{cubical sets} is the functor category $\cSet = \Fun(\cube^\op, \Set)$.
The \textit{standard $n$-cube} is the cubical set $\cube^n = \cube(-, 2^n)$, and the \textit{Yoneda embedding} $\cube \to \cSet$ is the functor induced by $2^n \mapsto \cube^n$.
For any cubical set $X$ we have
\begin{equation*}
X_n \cong \colim_{\cube^n \to X} \cube^n.
\end{equation*}

The functor of \textit{(normalized) chains} $\cchains \colon \cSet \to \Ch$ is the Kan extension along the Yoneda embedding of the functor $\cube \to \Ch$ assigning to $2^n$ in degree $m$ the $R$-module
\begin{equation*}
\frac{R\{\cube(2^m, 2^n)\}}{R\{\cube_{\deg}(2^m, 2^n)\}}
\end{equation*}
and whose boundary map is defined by
\begin{equation*}
\partial (\id_{2^n}) = \sum_{i=1}^{n} \ (-1)^i \
\big(\delta_i^1 - \delta_i^0 \big).
\end{equation*}
We remark that $\cchains(\cube^1)$ is isomorphic to the cellular chains on the interval with its standard CW structure
\begin{equation*}
C(\gcube) = R \big\{ [0], [1] \big\} \leftarrow R \big\{ [0,1] \big\}
\end{equation*}
and that $\cchains(\cube^n)$ is isomorphic to $C(\gcube)^{\otimes n}$. 