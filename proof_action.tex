
\section{Proof of \cref{t:cubical chain bialgebra}} \label{s:proof}

We need to show that the assignment
\begin{equation*}
\counit \mapsto \epsilon, \quad \coproduct \mapsto \Delta, \quad \product \mapsto \ast,
\end{equation*}
defined in \cref{s:action} is compatible with the relations
\begin{equation*}
\productcounit = 0,
\qquad
\leftcounitality = 0,
\qquad
\rightcounitality = 0,
\end{equation*}
and
\begin{equation*}
\partial\ \counit = 0,
\hspace*{.6cm}
\partial\ \coproduct = 0,
\hspace*{.6cm}
\partial\ \product = \ \boundary\,.
\end{equation*}
For the rest of this section let us consider two basis elements of $N_\bullet(\square^n) = N_\bullet(\square^1)^{\otimes n}$
\begin{align*}
x = x_1 \otimes \cdots \otimes x_n
\qquad \text{ and } \qquad
y = y_1 \otimes \cdots \otimes y_n.
\end{align*}
Since the degree of $\ast$ is $1$ and $\epsilon([0,1]) = 0$, we can verify the first relation easily:
\begin{align*}
\varepsilon(x \otimes y) & =
\sum (-1)^{|x|} \epsilon(y_{<i}) \epsilon(x_{<i}) \otimes \epsilon(x_i \ast y_i) \otimes \epsilon(x_{>i}) \epsilon(y_{>i}) = 0.
\end{align*}
For the second relation we want to show that $(\epsilon \otimes \id) \circ \Delta = \id$.
Since
\begin{gather*}
(\epsilon \otimes \id) \circ \Delta([0]) = \epsilon([0]) \otimes [0] = [0], \qquad
(\epsilon \otimes \id) \circ \Delta([1]) = \epsilon([1]) \otimes [1] = [1], \\
(\epsilon \otimes \id) \circ \Delta([0, 1]) = \epsilon([0]) \otimes [0, 1] + \epsilon([0, 1]) \otimes [1] = [0,1],
\end{gather*}
we have
\begin{align*}	
(\epsilon \otimes \id) \circ \Delta (x_1 \otimes \cdots \otimes x_n) &=
\sum \pm \left( \epsilon \big(x_1^{(1)}\big) \otimes \cdots \otimes \epsilon\big(x_n^{(1)}\big) \right) \otimes 	
\left( x_1^{(2)} \otimes \cdots \otimes x_n^{(2)} \right), \\ &=
x_1 \otimes \cdots \otimes x_n,
\end{align*}	
where the sign is obtained by noticing that the only non-zero term occurs when each factor $x_i^{(0)}$ is of degree $0$.
The third relation is verified analogously.
The fourth is precisely the fact that $\epsilon$ is a chain maps.
We will verify that $\Delta$ is a chain map recalling two facts: The map $\Delta$ is equal to the composition
\begin{equation*}
\begin{tikzcd}
\cchains(\square^1)^{\otimes n} \arrow[r, "\Delta^{\otimes n}"] & \left( \cchains(\square^1)^{\otimes 2}  \right)^{\otimes n} \arrow[r, "sh"] & \left( \cchains(\square^1)^{\otimes n} \right)^{\otimes 2},
\end{tikzcd}
\end{equation*}
and the tensor product of chain maps is a chain map.
Therefore, since
\begin{gather*}
\partial (\Delta)([0]) = \partial ([0] \otimes [0]) = 0, \qquad
\partial (\Delta)([1]) = \partial ([1] \otimes [1]) = 0, \\
\partial (\Delta)([0,1]) = \partial \big([0] \otimes [0,1] + [0,1] \otimes [1]\big) - [1] \otimes [1] + [0] \otimes [0] = 0,
\end{gather*}
the fifth relation follows.
To verify the sixth and final relation we need to show that
\begin{equation*}
\partial (x \ast y)\ +\ \partial x \ast y\ +\ (-1)^{|x|}x \ast \partial y\ =\ \epsilon(x) y \ -\ \epsilon(y) x.
\end{equation*}
We have
\begin{equation*}
x \ast y = \sum (-1)^{|x|} x_{<i} \, \epsilon(y_{<i}) \otimes x_i \ast y_i \otimes \epsilon(x_{>i})\, y_{>i}
\end{equation*}
and
\begin{align*}
\partial(x \ast y) & = 
\sum (-1)^{|x|} \, \partial x_{<i}\, \epsilon(y_{<i}) \otimes x_i \ast y_i \otimes \epsilon(x_{>i})\, y_{>i} \\ & +
\sum (-1)^{|x|+|x_{<i}|} \, x_{<i}\, \epsilon(y_{<i}) \otimes \partial (x_i \ast y_i) \otimes \epsilon(x_{>i}) \, y_{>i} \\ & -
\sum (-1)^{|x|+|x_{<i}|} \, x_{<i}\, \epsilon(y_{<i}) \otimes x_i \ast y_i \otimes \epsilon(x_{>i})\, \partial y_{>i}.
\end{align*}
Since
\begin{equation*}
|x| = |x_{<i}| + |x_i| + |x_{>i}|, \quad \epsilon(x_{>i}) \neq 0 \Leftrightarrow |x_{>i}| = 0, \quad \partial(x_i \ast y_i) \neq 0 \Rightarrow |x_i| = 0,
\end{equation*}
we have
\begin{equation} \label{e:boundary of product 1}
\begin{split}
\partial(x \ast y) & = 
\sum (-1)^{|x|} \, \partial x_{<i}\, \epsilon(y_{<i}) \otimes x_i \ast y_i \otimes \epsilon(x_{>i})\, y_{>i} \\ & +
\sum x_{<i} \, \epsilon(y_{<i}) \otimes \partial (x_i \ast y_i) \otimes \epsilon(x_{>i})\, y_{>i} \\ & -
\sum x_{<i} \, \epsilon(y_{<i}) \otimes x_i \ast y_i \otimes \epsilon(x_{>i})\, \partial y_{>i}.
\end{split}
\end{equation}
We also have
\begin{align*}
\partial x \ast y & = 
\sum (-1)^{|x|-1} \, \partial x_{<i}\, \epsilon(y_{<i}) \otimes x_i \ast y_i \otimes \epsilon(x_{>i}) \, y_{>i} \\ & +
\sum (-1)^{|x|-1+|x_{<i}|} \, x_{<i}\, \epsilon(y_{<i}) \otimes \partial x_i \ast y_i \otimes \epsilon(x_{>i}) \, y_{>i} \\ & +
\sum (-1)^{|x|-1+|x_{<i}|} \, x_{<i}\, \epsilon(y_{<i}) \otimes x_i \ast y_i \otimes \epsilon(\partial x_{>i}) \, y_{>i}.
\end{align*}
Since
\begin{equation*}
\epsilon(\partial x_{>i}) = 0, \quad \partial x_i \neq 0 \Leftrightarrow |x_i| = 1,
\end{equation*}
we have
\begin{equation} \label{e:boundary of product 2}
\begin{split}
\partial x \ast y & = 
\sum (-1)^{|x|-1} \, \partial x_{<i}\, \epsilon(y_{<i}) \otimes x_i \ast y_i \otimes \epsilon(x_{>i})\, y_{>i} \\ & +
\sum x_{<i}\, \epsilon(y_{<i}) \otimes \partial x_i \ast y_i \otimes \epsilon(x_{>i})\, y_{>i}.
\end{split}
\end{equation}
We also have
\begin{align*}
(-1)^{|x|} \, x \ast \partial y & = 
\sum x_{<i} \, \epsilon(\partial y_{<i}) \otimes x_i \ast y_i \otimes \epsilon(x_{>i})\, y_{>i} \\ & +
\sum (-1)^{|y_{<i}|} \, x_{<i}\, \epsilon(y_{<i}) \otimes x_i \ast \partial y_i \otimes \epsilon(x_{>i}) \, y_{>i} \\ & +
\sum (-1)^{|y_{<i}| + |y_i|} \, x_{<i}\, \epsilon(y_{<i}) \otimes x_i \ast y_i \otimes \epsilon(x_{>i}) \, \partial y_{>i},
\end{align*}
which is equivalent to
\begin{equation} \label{e:boundary of product 3}
\begin{split}
(-1)^{|x|} \, x \ast \partial y & = 
\sum x_{<i} \, \epsilon(y_{<i}) \otimes x_i \ast \partial y_i \otimes \epsilon(x_{>i})\, y_{>i} \\ & +
\sum x_{<i}\, \epsilon(y_{<i}) \otimes x_i \ast y_i \otimes \epsilon(x_{>i})\, \partial y_{>i}.
\end{split}
\end{equation}
Putting identities \eqref{e:boundary of product 1}, \eqref{e:boundary of product 2} and \eqref{e:boundary of product 3} together, we get
\begin{align*}
\partial (x \otimes y) \ +\ & \partial x \ast y\ +\, (-1)^{|x|}x \ast \partial y \\
& = \sum \epsilon(y_{<i})\, x_{<i} \otimes \big(\partial(x_i \ast y_i) + \partial x_i \ast y_i + x_i \ast \partial y_i\big) \otimes \epsilon(x_{>i})\, y_{>i}.
\end{align*}
Since
\begin{align*}
\partial(x_i \ast y_i)\ +\ \partial x_i \ast y_i\ +\ x_i \ast \partial y_i =
\epsilon(x_i)y_i\ -\ \epsilon(y_i)x_i,
\end{align*}
we have
\begin{align*}
\partial (x \ast y) \ +\ \partial x \ast y\ +\ & (-1)^{|x|}x \ast \partial y \\ = \ &
\sum \epsilon(y_{<i}) \, x_{<i} \otimes \epsilon(x_{\geq i}) y_{\geq i}\ -\
\epsilon(y_{\leq i}) \, x_{\leq i} \otimes \epsilon(x_{>i}) y_{>i} \\ = \ &
\epsilon(x)y - \epsilon(y)x,
\end{align*}
as desired, where the last equality follows from a telescopic sum argument.