
\section{Operads, props and \texorpdfstring{$E_\infty$}{E-infty}-structures} \label{s:operads and props}

In this section we start by reviewing the basic notions of operads and props.
For a more complete presentation we refer the reader to, for example, \cite{markl2008props}.
We then recall from \cite{medina2020prop1} the definition of the finitely presented prop $\M$ having the property that $\UM$ is an $E_{\infty}$-operad.
From the same reference we recall a combinatorial $\UM$-coalgebra structure on simplicial chains and, from \cite{medina2021cubical}, one on cubical chains.

\subsection{Symmetric (bi)modules}

Let $\S$ be the category whose objects are the natural numbers and whose set of morphisms between $m$ and $n$ is empty if $m \neq n$ and is otherwise the symmetric group $\S_n$.
A \textit{(left) $\S$-module} (resp. $\S$-\textit{bimodule}) is a functor from $\S$ (resp. $\S \times \S^\op$) to $\Ch$.
We respectively denote by $\smod$ and $\sbimod$ the categories of left $\S$-modules and of $\S$-bimodules with morphisms given by natural transformations.
We notice that the group homomorphisms $\S_n \to \S_n \times \S_1$ induce a forgetful functor $\U \colon \sbimod \to \smod$.
Explicitly, for a prop $\P$ we have $\U(\P)(r) = \P(1, r)$ for each $r \in \N$.

\subsection{Composition structures}

We can define \textit{operads} and \textit{props} by enriching $\S$-modules and $\S$-bimodules with certain composition structures.
For a complete presentation of these concepts we refer to Definition~11 and 54 of \cite{markl2008props}.
Intuitively, using examples defined in the next subsection, operads and props can be understood by abstracting the composition structure naturally present in the left $\S$-module $\End^C$, naturally an operad, and the $\S$-bimodule $\End^C_C$, naturally a prop.

We remark that if $\P$ is a prop, then its composition structure provides $\U(\P)$ with the structure of an operad.

\subsection{Representations}

Given a chain complex $C$ define
\begin{align*}
\End^C(r) &= \Hom(C, C^{\otimes r}), \\
\End^C_C(s, r) &= \Hom(C^{\otimes s}, C^{\otimes r}),
\end{align*}
with their natural operad and prop structures respectively.

Given an operad $\O$, an $\O$-\textit{structure} on $C$ is an operad morphism $\O \to \End^C$.
If equipped with an $\O$-\textit{structure} we say $C$ is an $\O$-coalgebra, denoting the category of these by $\coAlg_\O$.
For a prop $\P$, using $\End_C^C$ we define analogously the notion of $\P$-bialgebra and their category $\biAlg_\P$.

We remark that among these two categories only $\coAlg_\O$ is cocomplete in general.

\subsection{$E_{\infty}$-operads}

Recall that a \textit{free $\S_r$-resolution} of a chain complex $C$ is a quasi-isomorphism $R \to C$ from a chain complex of free $\k[\S_r]$-modules.

An $\S$-module $M$ is said to be $E_{\infty}$ if there exists a morphism of $\S$-modules $M \to \underline{\k}$ inducing for each $r \in \N$ a free $\S_r$-resolution $M(r) \to \k$.
For example, we can obtain one such $\S$-module by using the functor of singular chains and the set, parameterized by $r \in \N$, of maps to the terminal space from models of the universal bundle $\mathrm{E} \S_r$.

An operad is said to be $E_{\infty}$ if its underlying $\S$-module is $E_\infty$.

\subsection{Free constructions} \label{ss:free constructions}

The \textit{free prop} $\F(M)$ generated by an $\S$-bimodule $M$ is constructed using directed graphs with no directed loops that are enriched with a labeling described next.
We think of each directed edge as built from two compatibly directed half-edges.
For each vertex $v$ of a directed graph $G$, we have the sets $in(v)$ and $out(v)$ of half-edges that are respectively incoming to and outgoing from $v$.
Half-edges that do not belong to $in(v)$ or $out(v)$ for any $v$ are divided into the disjoint sets $in(G)$ and $out(G)$ of incoming and outgoing external half-edges.
For any positive integer $n$ let $\overline{n} = \{1, \dots, n\}$ and set $\overline{0} = \emptyset$.
For any finite set $S$, denote the cardinality of $S$ by $|S|$.
The labeling is given by bijections
\[
\overline{|in(G)|}\to in(G), \qquad
\overline{|out(G)|}\to out(G),
\]
and
\[
\overline{|in(v)|}\to in(v), \qquad
\overline{|out(v)|}\to out(v),
\]
for every vertex $v$.
We refer to the isomorphism classes of such labeled directed graphs with no directed loops as $(n,m)$\textit{-graphs} denoting the set of these by $\G(m,n)$.
We use graphs immersed in the plane to represent elements in $\G(m,n)$, please see \cref{f:immersion}.
We consider the right action of $\S_n$ and the left action of $\S_m$ on a $(n,m)$-graph given respectively by permuting the labels of $in(G)$ and $out(G)$.
This action defines the $\S$-bimodule structure on the free prop
\begin{equation} \label{e:free prop}
\F(M)(m,n) \ = \bigoplus_{\Gamma \in \G(m,n)} \bigotimes_{v \in Vert(\Gamma)} out(v) \otimes_{\S_q} M(p, q) \otimes_{\S_p} in(v),
\end{equation}
where we simplified the notation writing $p$ and $q$ for $\overline{|in(v)|}$ and $\overline{|out(v)|}$ respectively.
The composition structure is defined by (relabeled) grafting and disjoint union.

The \textit{free operad} generated by an $\S$-module is defined analogously using $(1,n)$-graphs only.

\begin{figure}
	\input{aux/immersion}
	\caption{Immersed graphs represent labeled directed graphs with the direction implicitly given from top to bottom and the labeling from left to right.}
	\label{f:immersion}
\end{figure}

\subsection{Coalgebras revisited}

We recall the definition of the operad $\As$ controlling coalgebras in the sense of \cref{ss:coalgebras}.
More precisely, $\As$ is such that there is a natural isomorphisms between $\coAlg_\As$ and $\coAlg$.

\begin{definition}
	Let $\As$ be the operad generated by
	\[
	\counit\,, \qquad \coproduct\,,
	\]
	both in homological degrees $0$ and boundaries
	\[
	\partial\ \counit = 0,
	\hspace*{.6cm}
	\partial\, \coproduct = 0,
	\]
	modulo the operad ideal generated by the relations
	\[
	\leftcounitality\,, \hspace*{.6cm} \rightcounitality\,, \hspace*{.5cm} \coassociativity\,.
	\]
\end{definition}

We remark that this operad is in any arity $r$ isomorphic to the chain complex $\k[\S_r]$ concentrated in degree~$0$.

\subsection{The prop $\M$} \label{ss:definition of M}

We review from \cite{medina2020prop1} the finitely presented $E_{\infty}$-prop $\M$ which, given its small number of generators and relations, is well suited to define $E_{\infty}$-structures.

\begin{definition}
	Let $\M$ be the prop generated by
	\begin{equation} \label{e:generators of M}
	\counit\,, \hspace*{.6cm} \coproduct\,, \hspace*{.6cm} \product,
	\end{equation}
	in degrees $0$, $0$ and $1$ respectively, and boundaries
	\begin{equation} \label{e:boundary of M}
	\partial\ \counit = 0,
	\hspace*{.6cm}
	\partial\, \coproduct = 0,
	\hspace*{.6cm}
	\partial \product = \ \boundary\,,
	\end{equation}
	modulo the prop ideal generated by
	\begin{equation} \label{e:relations of M}
	\leftcounitality\,, \hspace*{.6cm} \rightcounitality\,, \hspace*{.5cm} \coassociativity\,, \hspace*{.6cm} \productcounit.
	\end{equation}
\end{definition}

Explicitly, any element in $\M(m,n)$ can be written as a linear combination of the $(m,n)$-graphs generated by those in \eqref{e:generators of M} via grafting, disjoint union and relabeling, modulo the prop ideal generated by the relations in \eqref{e:relations of M}, and its boundary is determined, using \eqref{e:free prop}, by \eqref{e:boundary of M}.

Originally this prop was defined without imposing the relation \ \coassociativity \,.
This is advantageous since in that case the associated reduced operad is a cofibrant resolution of the terminal operad.
Since in this work we are interested in extending the Alexander--Whitney and Serre coalgebras, which are coassociative, we find it convenient to impose this relation, so obtaining an inclusion $\As \to \UM$ and a forgetful functor $\coAlg_\UM \to \coAlg$.

The same proof given in \cite[Theorem 3.3]{medina2020prop1} establishes the following.

\begin{proposition}
	The operad $\UM$ is $E_{\infty}$.
\end{proposition}

We remark that, as proven in \cite{medina2018prop2}, this prop is obtained from applying the functor of cellular chains to a finitely presented prop over the category of CW-complexes.

\subsection{Simplicial $E_{\infty}$-structure} \label{ss:e-infty on simplicial}

We review from \cite{medina2020prop1} a natural $\mathcal M$-structure on the chains of standard simplices.
These define, for any simplicial set, a natural $\UM$-structure on its chains generalizing the $E_{\infty}$-coalgebra structures constructed by McClure--Smith \cite{mcclure2003multivariable} and Berger--Fresse \cite{berger2004combinatorial}.

An $\M$-structure is specified by three linear maps, the images of the generators
\[
\counit, \quad \coproduct, \quad \product,
\]
satisfying the relations in the presentation of $\mathcal M$.
For the chains on standard simplices, the first two generators are send respectively to the counit and coproduct of the Alexander--Whitney coalgebra as defined in \cref{ss:aw coalgebra}, and the third generator to an algebraic version of the join $\ast \colon \chains(\simplex^n)^{\otimes 2} \to \chains(\simplex^n)$ defined by
\[
\left[v_0, \dots, v_p \right] \ast \left[v_{p+1}, \dots, v_q\right] = \begin{cases} (-1)^{p+|\pi|} \left[ v_{\pi(0)}, \dots, v_{\pi(q)} \right] & \text{ if } v_i \neq v_j \text{ for } i \neq j, \\
0 & \text{ otherwise}, \end{cases}
\]
where $\pi$ is the permutation that orders the totally ordered set of vertices, and $(-1)^{|\pi|}$ its sign.

The same proof given in \cite[Theorem 4.2]{medina2020prop1} establishes the following.

\begin{proposition} \label{p:simplicial chain bialgebra}
	For every $n \in \mathbb{N}$, the assignment
	\[
	\counit \mapsto \epsilon, \quad \coproduct \mapsto \Delta, \quad \product \mapsto \ast,
	\]
	defines a natural $\mathcal M$-structure on $\chains(\simplex^n)$ extending the Alexander--Whitney coalgebra.
\end{proposition}

The chains on general simplicial sets are not equipped with an $\M$-structure, but using the forgetful functor from $\biAlg_{\M}$ to the cocomplete category $\coAlg_\UM$ allows us to Kan extend the induced natural $\UM$-structures on standard simplices to all simplicial sets.
Specifically, we obtain a lift:
\[
\begin{tikzcd}[column sep=normal, row sep=small]
& \coAlg_\UM \arrow[d] \\
& \coAlg \arrow[d] \\
\sSet \arrow[r, "\schains"]
\arrow[ur, "\schainsAs", out=45, in=180]
\arrow[uur, "\schainsUM", out=90, in=180]
& \Ch.
\end{tikzcd}
\]

\subsection{Cubical $E_\infty$-structure} \label{ss:e-infty on cubical}

We follow the presentation of \cref{ss:e-infty on simplicial} closely to review from \cite{medina2021cubical} a natural $\mathcal M$-structure on the chains of standard cubes leading to a natural $\UM$-structure on the chains of any cubical set.

An $\M$-structure is specified by three linear maps, the images of the generators
\[
\counit, \quad \coproduct, \quad \product,
\]
satisfying the relations in the presentation of $\mathcal M$.
For the chains on standard cubes, the first two generators are send respectively to the counit and coproduct of the Serre coalgebra as defined in \cref{ss:serre coalgebra}, and the third generator to a degree one map $\ast \colon \schains(\simplex^n)^{\otimes 2} \to \schains(\simplex^n)$ defined by
\begin{align*}
(x_1 \otimes \dots \otimes x_n) \ast (y_1 \otimes \dots \otimes y_n) =
(-1)^{|x|} \sum_{i=1}^n x_{<i} \, \epsilon(y_{<i}) \otimes x_i \ast y_i \otimes \epsilon(x_{>i}) \, y_{>i},
\end{align*}
where
\begin{align*}
x_{<i} & = x_1 \otimes \dots \otimes x_{i-1}, &
y_{<i} & = y_1 \otimes \dots \otimes y_{i-1}, \\
x_{>i} & = x_{i+1} \otimes \dots \otimes x_n, &
y_{>i} & = y_{i+1} \otimes \dots \otimes y_n,
\end{align*}
with the convention
\[
x_{<1} = y_{<1} = x_{>n} = y_{>n} = 1 \in \k,
\]
and the only non-zero values of $x_i \ast y_i$ are
\[
\ast([0] \otimes [1]) = [0, 1], \qquad \ast([1] \otimes [0]) = -[0, 1].
\]

The same proof given in \cite{medina2020prop1} establishes the following.

\begin{proposition} \label{thm: cubical chain bialgebra}
	For every $n \in \mathbb{N}$, the assignment
	\[
	\counit \mapsto \epsilon, \quad \coproduct \mapsto \Delta, \quad \product \mapsto \ast,
	\]
	defines a natural $\mathcal M$-structure on $\chains(\cube^n)$ extending the Serre coalgebra.
\end{proposition}

Kan extending the induced natural $\UM$-structures on standard simplices to all simplicial sets defines a lift of the functor of cubical chains to $\UM$-coalgebras:
\begin{equation} \label{e:lift of cubical chains to UM-coalgs}
\begin{tikzcd}[column sep=normal, row sep=small]
& \coAlg_\UM \arrow[d] \\
& \coAlg \arrow[d] \\
\cSet \arrow[r, "\cchains"]
\arrow[ur, "\cchainsAs", out=45, in=180]
\arrow[uur, "\cchainsUM", out=90, in=180]
& \Ch.
\end{tikzcd}
\end{equation}
In the next section we will show that $\cchainsUM$ is a monoidal functor.